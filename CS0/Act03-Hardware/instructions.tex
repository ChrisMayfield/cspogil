\model{Machine Instructions}

\begin{multicols}{2}
\includegraphics[width=\linewidth]{opcode1.png}

\columnbreak

\includegraphics[width=\linewidth]{opcode2.png}
\end{multicols}

\begin{center}
\color{gray}\scriptsize Image credit:
Brookshear and Brylow. (2015). \textit{Computer Science: An Overview}. Pearson.
\end{center}


\quest{8 min}


\Q How many bits is the op-code? How many possible op-codes can the machine support?

\begin{answer}[3em]
With 4-bit op-codes, the machine can have at most 16 different op-codes.
\end{answer}


\Q The op-code for loading data from memory into a register is 1.
Write an instruction (in hex) for loading data from address 3D into register 4.

\begin{answer}[2em]
143D
\end{answer}


\Q Why is the instruction register in \ref{architecture.tex} twice as large as the other registers?
How many memory cells are needed to store a single instruction?

\begin{answer}
Instructions are 16 bits, and memory cells are 8 bits; one instruction takes up two cells.
Note the instructions need to be large enough for their operand to represent a memory address.
\end{answer}
