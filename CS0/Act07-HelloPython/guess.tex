\model{Guessing Game}

Create a new file named \texttt{guess.py} and enter the following code.
Replace the name in Line~2 with your own name.
Be careful to type the code \emph{exactly} as shown.

\begin{pythbox}
name = raw_input("What is your name? ")
if name == "Taylor":
    print name, "is a great name!"
else:
    print name, "is an okay name."
\end{pythbox}

Note: \verb|raw_input| is a \textbf{function} that displays a \textbf{prompt} on the screen and reads a line from the keyboard.
In this program, the result of \verb|raw_input| is stored in the \textbf{variable} \pyth{name}.


\quest{15 min}


\Q What is the prompt? Why is there a space at the end of it?

\begin{answer}[3em]
The prompt is \str{What is your name? }.
The space at the end makes it so that the input isn't ``touching'' the question mark when the user types.
\end{answer}


\Q Run the program a few times, entering a different name each time. Feel free to modify the messages as you see fit.

\vspace{1em}


\Q Enter each of these lines into the IDLE shell, and explain where the syntax error occurs.

\begin{enumerate}[itemsep=1ex]
\item \verb|name? = raw_input("What is your name?")|
\ans{question mark}

\item \verb|your name = raw_input("What is your name?")|
\ans{space between your and name}

\item \verb|1st_name = raw_input("What is your name?")|
\ans{the word 1st}

\item \verb|from = raw_input("Where were you born?")|
\ans{equals sign (from is a keyword)}
\end{enumerate}


\Q Based on the errors in the previous question and the following correct examples, describe three rules that need to be followed when naming a variable.

\begin{pythlst}
    name2 = raw_input("What is your name?")
    your_name = raw_input("What is your name?")
    firstName = raw_input("What is your name?")
\end{pythlst}

\begin{answer}
Answers may include: it can't have punctuation or other symbols, it has to be one word, it can't start with a number, and it can't be a keyword.
\end{answer}


\Q \label{vars} At the end of your \texttt{guess.py} program, create two new variables named \pyth{number} and \pyth{guess}.
Set the value of \pyth{number} to be an integer between 1 and 100 (of your choice).
Ask the user to guess your number, and store the result in \pyth{guess}.
When asking for numbers, use \pyth{input} instead of \pyth{raw_input}.
Write your two statements in the space below.

\begin{answer}[3em]
\begin{pythans}
number = 74  # or some other value
guess = input("Guess my number: ")
\end{pythans}
\end{answer}


\Q Add the following logic to your program:
If the guess is too high, display the message \str{Too high!};
if the guess is too low, display the message \str{Too low!};
if the user guessed the number, display \str{You got it!}.
Write your statements in the space below.

\begin{answer}[8em]
\begin{pythans}
    if guess < number:
        print "Too low!"
    if guess > number:
        print "Too high!"
    if guess == number:
        print "You got it!"
\end{pythans}
\end{answer}


\Q What is the difference between \pyth{=} and \pyth{==} in the programs you have written today?

\begin{answer}[3em]
The \pyth{=} operator is used to \emph{assign} values to variables, whereas the \pyth{==} operator is used to \emph{compare} values for equality.
\end{answer}


\Q At this point, you should have a program that allows the user to make only one guess.
Rather than run this program over and over again, you can use a \pyth{while} loop to make it repeat the guessing part.
Insert the following two lines before the \pyth{input} line you wrote in \ref{vars}.

\begin{pythlst}
    guess = -1
    while guess != number:
\end{pythlst}

\vspace{1em}


\Q What did you have to do after inserting the \pyth{while} loop to make it work?
In other words, how did you make the \pyth{input} and \pyth{if} statements part of the \pyth{while} loop?

\begin{answer}
You need to indent the corresponding lines of code underneath it.
Select the rest of the program lines, and then press the Tab key.
\end{answer}


\Q Rather than guess the same number every time, you can have the computer select a random number for you:
\begin{itemize}[itemsep=2pt]
\item At the top of your program, add the line ``\pyth{import random}'' (without the quotes).
\item Then change the line where you set value of \pyth{number} to use this example instead:
\begin{pythlst}
number = random.randint(1, 100)
\end{pythlst}
\end{itemize}
