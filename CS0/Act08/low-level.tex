\model{Low-Level Languages}

The following program, shown in three different languages, calculates the sum of numbers from 1 to 10.
In other words, it adds $1 + 2 + \ldots + 10 = 55$.

\begin{center}
\begin{minipage}[t]{168pt}
%\hrulefill

\textbf{Machine Code} \\
(1st Generation)

\begin{verbatim}
0x000:
0x000: 700001000000000000

0x100:
0x100:
0x100: 30f00b00000000000000
0x10a: 30f30100000000000000
0x114: 30f10200000000000000
0x11e: 30f20100000000000000

0x128: 2017
0x12a: 6107
0x12c: 734601000000000000

0x135:
0x135: 6013
0x137: 6021

0x139: 2017
0x13b: 6107
0x13d: 743501000000000000

0x146:
0x146: 00
\end{verbatim}
\end{minipage}
\hfill\vline\hfill
\begin{minipage}[t]{144pt}
%\hrulefill

\textbf{Y86-64 Assembly} \\
(2nd Generation)

\begin{verbatim}
.pos 0 code
    jmp _start

.pos 0x100 code
_start:
    irmovq  $0xb,  %rax
    irmovq  $0x1,  %rbx
    irmovq  $0x2,  %rcx
    irmovq  $0x1,  %rdx

    rrmovq  %rcx,  %rdi
    subq    %rax,  %rdi
    je      done

loop:
    addq    %rcx,  %rbx
    addq    %rdx,  %rcx

    rrmovq  %rcx,  %rdi
    subq    %rax,  %rdi
    jne     loop

done:
    halt
\end{verbatim}
\end{minipage}
\hfill\vline\hfill
\begin{minipage}[t]{148pt}
%\hrulefill

\textbf{Standard C} \\
(3rd Generation)

\begin{verbatim}



int main()
{
    int upper = 11;
    int sum = 1;
    int val = 2;


    while (val < upper)
    {
        sum = sum + val;
        val++;
    }
}
\end{verbatim}
\end{minipage}
\end{center}


\quest{15 min}


\Q Compare the length of each program. Do not count labels (e.g, \verb|0x000:|, \verb|.pos 0 code|) or punctuation (e.g., \verb|{|, \verb|}|).

\begin{enumerate}
\item How many instructions of machine code?  \ans{14}
\item How many instructions of assembly code?  \ans{14}
\item How many non-blank, non-brace lines of C code?  \ans{7}
\end{enumerate}


\Q All data values for this program are stored in registers named \verb|%rax|, \verb|%rbx|, etc.

\begin{enumerate}
\item In which register is the sum stored?  \ans{\tt \%rbx}
\item In which register is the next value to add stored?  \ans{\tt \%rcx}
%\item What constant value is stored in register \verb|%rdx|?  \ans{1}
\end{enumerate}


\Q The instruction \verb|irmovq| means ``move immediate value to register''. Immediate values begin with a dollar sign (\verb|$|), and registers begin with a percent sign (\verb|%|).

\begin{enumerate}
\item What is the value 11 in assembly code?  \ans{\tt \$0xb}
\item Does assembly use decimal or hexadecimal?  \ans{hexadecimal}
\item Does Standard C use decimal or hexadecimal?  \ans{decimal}
\end{enumerate}


\Q In terms of the machine, what does an assignment statement do?
As part of your answer, name the instructions in \ref{\currfilename} that perform assignment.

\begin{answer}
Assignment updates the register (or memory location) that corresponds to the variable being assigned.
In this program, the \verb|irmovq| and \verb|addq| perform assignment.
\end{answer}


\Q Consider the line ``\verb|rrmovq  %rcx,  %rdi|''.
The instruction \verb|rrmovq| means ``move (copy) register to register''.

\begin{enumerate}
\item What is stored in register \verb|%rcx|? \ans{The next value to be added}
\item Where is this value copied to? \ans{Register {\tt \%rdi}}
\end{enumerate}


\Q The instruction \verb|subq| means ``subtract''. Given two registers $R$ and $T$, \verb|subq| performs $R-T$ and stores the result in $T$.

\begin{enumerate}
\item What is stored in register \verb|%rax|? \ans{The value 11, which is the upper limit.}
\item In what case would \verb|%rax| $-$ \verb|%rdi| be zero? \ans{When their values are equal.}
\end{enumerate}


\Q The instruction \verb|je| means ``jump if the last operation's result equals 0'', and the instruction \verb|jne| means ``jump if the last operation's result does not equal 0''. Circle the portion of assembly code that corresponds to the \verb|while| loop in C.

\begin{answer}[2em]
Eight instructions should be circled, starting from the first {\tt rrmovq} and ending with {\tt jne}.
\end{answer}


% Discussion questions:
%
% Why would you chose to write in C vs Y86-64?
% What are the pros and cons of each language?
