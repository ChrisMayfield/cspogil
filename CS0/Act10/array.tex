\model{Array}

% http://www.the-horoscope.com/general-horoscope/leo-lucky-numbers.asp

Consider a list of numbers:

\begin{quote}
\begin{pythlst}
lucky = [29, 16, 23, 47, 37]
\end{pythlst}
\end{quote}

The easiest way to store these numbers in memory is to put them side by side.
For example, if the list were to begin at memory address 40:

\setlength{\defaultwidth}{1em}

\begin{center}
\renewcommand{\arraystretch}{1.8}
\begin{tabular}{rcccccccccccc}
\textit{Address:~~} &
& 40 & 41 & 42 & 43 & 44 & 45 & 46 & 47 & 48 & 49 &
\\ \cline{2-13}
\textit{Value:~~} &
\multicolumn{1}{c|}{~~$\cdots$~~} &
\multicolumn{1}{c|}{29} &
\multicolumn{1}{c|}{16} &
\multicolumn{1}{c|}{23} &
\multicolumn{1}{c|}{47} &
\multicolumn{1}{c|}{37} &
\multicolumn{1}{c|}{} &
\multicolumn{1}{c|}{} &
\multicolumn{1}{c|}{} &
\multicolumn{1}{c|}{} &
\multicolumn{1}{c|}{} &
~~$\cdots$~~
\\ \cline{2-13}
\textit{Index:~~} &
& \ans{0}
& \ans{1}
& \ans{2}
& \ans{3}
& \ans{4}
& \ans{}
& \ans{}
& \ans{}
& \ans{}
& \ans{}
&
\end{tabular}
\end{center}

When programming, we can access individual numbers by their \emph{index}.
For example, the value of \pyth{lucky[1]} is \pyth{16}, and \pyth{lucky[5]} is out of range.


\quest{10 min}


\Q In the diagram above, write the index below each value.

\vspace{1em}


\Q We call the beginning of the list the \emph{head} and the end of the list the \emph{tail}.

\begin{multicols}{2}
\begin{enumerate}
\item What is the address of the head? \ans{40}
\item What is the address of the tail? \ans{44}
\item What is the index of the head? \ans{0}
\item What is the index of the tail? \ans{4}
\end{enumerate}
\end{multicols}


\Q What is the relationship between the index and the corresponding memory address?

\begin{answer}
The index is relative to the first address.
It counts the number of cells over from the head.
\end{answer}


\Q Based on your answer to the previous question, why do indexes start at 0 instead of 1?

\begin{answer}
Starting at zero makes it easier to compute the memory address of list items.
We simply add \pans{head + index}.
\end{answer}


\Q There are (currently) five numbers in the list. Why is \pyth{lucky[5]} out of range?

\begin{answer}
The indexes range from 0 to 4.
\end{answer}


\Q The statement \pyth{lucky.insert(2, 13)} changes the list to \pyth{[29, 16, 13, 23, 47, 37]}.
What is the address of the head and tail after inserting 13?

\begin{answer}
The head is 40 (unchanged), and the tail is 45.
\end{answer}


\Q In terms of memory operations, what does it take to insert values in the middle of an array?

\begin{answer}
All subsequent values in the array need to be shifted over by one memory cell, resulting in multiple LOAD and STORE operations.
\end{answer}
