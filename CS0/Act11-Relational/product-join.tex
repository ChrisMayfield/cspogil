\model{Product and Join}

Mathematically speaking, we combine tables by ``multiplying'' them.
Every row in the right table is appended to every row in the left table:

\begin{multicols}{3}
\centering

\begin{tabular}{|l|}
\multicolumn{1}{l}{{\bf A}} \\
\hline
\tr let  \\ \hline
\hline
A        \\ \hline
B        \\ \hline
C        \\ \hline
\end{tabular}

\columnbreak

\begin{tabular}{|l|}
\multicolumn{1}{l}{{\bf B}} \\
\hline
\tr num  \\ \hline
\hline
1        \\ \hline
2        \\ \hline
\end{tabular}

\columnbreak

A $\times$ B

\vspace*{1ex}
\begin{tabular}{|l|l|}
\hline
\tr let  & \tr num  \\ \hline
\hline
A        & 1        \\ \hline
A        & 2        \\ \hline
B        & 1        \\ \hline
B        & 2        \\ \hline
C        & 1        \\ \hline
C        & 2        \\ \hline
\end{tabular}

\end{multicols}

In relational databases, a \emph{join} operation is a product followed by a condition.
The condition is used to specify which of the combined rows should be part of the result.

\begin{multicols}{3}
\centering
\small

\begin{tabular}{|l|l|l|}
\multicolumn{3}{l}{{\bf course}} \\
\hline
\tr cid  & \tr dept  & \tr num  \\ \hline
\hline
13466    & CS        & 101      \\ \hline
13468    & CS        & 149      \\ \hline
56482    & MATH      & 231      \\ \hline
%67606    & SCOM      & 123      \\ \hline
\end{tabular}

\columnbreak

\begin{tabular}{|l|l|}
\multicolumn{2}{l}{{\bf teach}} \\
\hline
\tr cid  & \tr pid  \\ \hline
\hline
13466    & 2774       \\ \hline
13468    & 2774       \\ \hline
13466    & 9036       \\ \hline
13468    & 9036       \\ \hline
%56482    & 1158       \\ \hline
\end{tabular}

\columnbreak

\begin{tabular}{|l|l|l|}
\multicolumn{3}{l}{{\bf professor}} \\
\hline
\tr pid  & \tr dept  & \tr name  \\ \hline
\hline
2774    & CS        & Mayfield  \\ \hline
9036    & CS        & Stewart   \\ \hline
1158    & MATH      & Taalman   \\ \hline
5241    & SCOM      & Hazard    \\ \hline
\end{tabular}

\end{multicols}

\vspace{2pt}

\begin{multicols}{2}
\centering
\small

course $\times$ teach

\vspace*{1ex}
\begin{tabular}{|l|l|l|l|l|}
\hline
\tr cid  & \tr dept  & \tr num  & \tr cid  & \tr pid  \\ \hline
\hline
13466    & CS        & 101      & 13466    & 2774     \\ \hline
13466    & CS        & 101      & 13468    & 2774     \\ \hline
13466    & CS        & 101      & 13466    & 9036     \\ \hline
13466    & CS        & 101      & 13468    & 9036     \\ \hline
13468    & CS        & 149      & 13466    & 2774     \\ \hline
13468    & CS        & 149      & 13468    & 2774     \\ \hline
13468    & CS        & 149      & 13466    & 9036     \\ \hline
13468    & CS        & 149      & 13468    & 9036     \\ \hline
56482    & MATH      & 231      & 13466    & 2774     \\ \hline
56482    & MATH      & 231      & 13468    & 2774     \\ \hline
56482    & MATH      & 231      & 13466    & 9036     \\ \hline
56482    & MATH      & 231      & 13468    & 9036     \\ \hline
\end{tabular}

\columnbreak

JOIN \sub{course.cid = teach.cid} (course, teach)

\vspace*{1ex}
\begin{tabular}{|l|l|l|l|l|}
\hline
\tr cid  & \tr dept  & \tr num  & \tr cid  & \tr pid  \\ \hline
\hline
13466    & CS        & 101      & 13466    & 2774     \\ \hline
13466    & CS        & 101      & 13466    & 9036     \\ \hline
13468    & CS        & 149      & 13468    & 2774     \\ \hline
13468    & CS        & 149      & 13468    & 9036     \\ \hline
\end{tabular}

\vfill
\null

\end{multicols}


\quest{10 min}


\Q How many rows and columns are in:

\begin{enumerate}[itemsep=1ex]
\item the course table? \ans{3 rows, 3 cols}
\item the teach table? \ans{4 rows, 2 cols}
\item course $\times$ teach? \ans{12 rows, 5 cols}
\end{enumerate}


\Q Consider a table with $i$ rows and $j$ columns, and another table with $k$ rows and $l$ columns.

\begin{enumerate}[itemsep=1ex]
\item how many rows will be in the product? \ans{$i * k$}
\item how many columns will be in the product? \ans{$j + l$}
\end{enumerate}


\Q Discuss how the results of ``course $\times$ teach'' are different from the JOIN operation.
Then in \ref{\currfilename}, draw an arrow from each result in the JOIN to the corresponding row in the product.

\begin{answer}[3em]
There should be four arrows: \\ row 1 $\leftarrow$ row 1, row 3 $\leftarrow$ row 2, row 6 $\leftarrow$ row 3, row 8 $\leftarrow$ row 4.
\end{answer}


\Q What is the result of JOIN \sub{teach.pid = professor.pid} (teach, professor)?
Don't forget to include the column names.

\begin{answer}[12em]
\begin{quote}
\begin{tabular}{|l|l|l|l|l|}
\hline
\tr cid  & \tr pid  & \tr pid  & \tr dept  & \tr name  \\ \hline
\hline
13466    & 2774        & 2774      & CS    & Mayfield     \\ \hline
13468    & 2774        & 2774      & CS    & Mayfield     \\ \hline
13466    & 9036        & 9036      & CS    & Stewart     \\ \hline
13468    & 9036        & 9036      & CS    & Stewart     \\ \hline
\end{tabular}
\end{quote}
\end{answer}


\Q Describe what relational operations you would have to use to find the names of all professors who teach CS 101.
(The results should have 2 rows and 1 column.)

\begin{answer}
You would first have to join the course, teach, and professor tables using the conditions ``cid = cid'' and ``pid = pid''.
Then you would need to apply a selection on the rows with ``CS 101''.
Finally, you would need to project the ``name'' column.
\end{answer}
