\model{Select and Project}

% subscript for relational algebra
\newcommand{\sub}[1]{$_{\mbox{\small #1}}$}

In relational databases, \emph{data} is organized as tables.
We use \emph{SELECT} to work with rows and \emph{PROJECT} to work with columns.
The names of the columns are called the \emph{schema}.

% http://vendinghow.com/article/what-are-the-most-popular-items-to-stock-in-a-vending-machine

\vspace{-1ex}
\begin{center}
\begin{tabular}{|l|l|l|l|}
\multicolumn{4}{l}{{\bf snacks}} \\
\hline
\tr name      & \tr owner & \tr calories & \tr price \\ \hline
\hline
Snickers      & Mars      & 215          & 1.25      \\ \hline
Peanut M\&M's & Mars      & 250          & 1.00      \\ \hline
Twix          & Mars      & 286          & 1.25      \\ \hline
Reeses Pieces & Hershey   & 234          & 1.00      \\ \hline
Butterfinger  & Nestle    & 275          & 1.25      \\ \hline
Milk Duds     & Hershey   & 218          & 1.50      \\ \hline
Milky Way     & Mars      & 264          & 1.25      \\ \hline
Baby Ruth     & Ferrero   & 275          & 1.50      \\ \hline
Doritos       & Frito-Lay & 140          & 0.75      \\ \hline
Cheetos       & Frito-Lay & 160          & 0.75      \\ \hline
\end{tabular}
\end{center}

\emph{Examples:}
\vspace{1ex}

\begin{multicols}{2}
\centering

SELECT \sub{price $\geq$ 1.50} (snacks)

\vspace*{1ex}
\begin{tabular}{|l|l|l|l|}
\hline
\tr name      & \tr owner & \tr calories & \tr price \\ \hline
\hline
Milk Duds     & Hershey   & 218          & 1.50      \\ \hline
Baby Ruth     & Ferrero   & 275          & 1.50      \\ \hline
\end{tabular}

\vspace{2em}

SELECT \sub{price $<$ 0} (snacks)

\vspace*{1ex}
\begin{tabular}{|l|l|l|l|}
\hline
\tr name      & \tr owner & \tr calories & \tr price \\ \hline
\hline
\end{tabular}

\vspace{2em}

PROJECT \sub{name} $\big($SELECT \sub{price = 0.75} (snacks)$\big)$

\vspace*{1ex}
\begin{tabular}{|l|}
\hline
\tr name \\ \hline
\hline
Doritos  \\ \hline
Cheetos  \\ \hline
\end{tabular}

\columnbreak

PROJECT \sub{owner, calories} (snacks)

\vspace*{1ex}
\begin{tabular}{|l|l|}
\hline
\tr owner & \tr calories \\ \hline
\hline
Mars      & 215          \\ \hline
Mars      & 250          \\ \hline
Mars      & 286          \\ \hline
Hershey   & 234          \\ \hline
Nestle    & 275          \\ \hline
Hershey   & 218          \\ \hline
Mars      & 264          \\ \hline
Ferrero   & 275          \\ \hline
Frito-Lay & 140          \\ \hline
Frito-Lay & 160          \\ \hline
\end{tabular}

\end{multicols}


\quest{10 min}


\Q How many rows and columns are in:

\begin{enumerate}[itemsep=1ex]
\item the original snacks table? \ans{10 rows, 4 cols}
\item selecting price $\geq$ 1.50? \ans{2 rows, 4 cols}
\item selecting price $<$ 0? \ans{0 rows, 4 cols}
\item projecting owner and calories? \ans{10 rows, 2 cols}
\end{enumerate}


\Q Which operation (SELECT or PROJECT) affects the schema? Justify your answer.

\begin{answer}
The PROJECT operation reduces the table to fewer columns.
\end{answer}


\Q The bottom-left example in \ref{\currfilename} uses both SELECT and PROJECT.
Describe the data source of each operation (the part in parentheses):

\begin{enumerate}[itemsep=1ex]
\item SELECT \sub{\ldots} (which data?) \ans{the original snacks table}
\item PROJECT \sub{\ldots} (which data?) \ans{the result of the SELECT}
\end{enumerate}


\Q In addition to the data source, what other information (the part in \sub{subscript}) is required for:

\begin{enumerate}[itemsep=1ex]
\item a SELECT operation?  \ans[20em]{a condition for which rows to select}
\item a PROJECT operation? \ans[20em]{a list of columns to project (retain)}
\end{enumerate}


\Q Explain what is wrong with this example: ~ SELECT \sub{price = 0.75} $\big($PROJECT \sub{name} (snacks)$\big)$

\begin{answer}
The PROJECT operation removes all columns except for name, so you can't SELECT based on price anymore.
\end{answer}


\Q Write the following \emph{queries} using SELECT and/or PROJECT:

\begin{enumerate}
\item List the name and price of all snacks.
\ans[20em]{PROJECT \sub{name, price} (snacks)}

\item Find snacks with less than 200 calories.
\ans[20em]{SELECT \sub{calories $<$ 200} (snacks)}

\item Which company makes Twix?
\ans[23em]{PROJECT \sub{owner} $\big($SELECT \sub{name = \qs{Twix}\qs} (snacks)$\big)$}
\end{enumerate}
