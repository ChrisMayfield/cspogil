\model{Assignment}

Declaring a variable instructs the computer to reserve space for it in memory.

\vspace{1em}
\hfill
\begin{minipage}[t]{0.5\textwidth}

\textbf{Java code}
\vspace{1ex}

\begin{javanum}
int dollars;
int cents;
\end{javanum}

\end{minipage}
\hfill
\begin{minipage}[t]{0.4\textwidth}

\textbf{Computer memory}
\vspace{1em}

\var{dollars}{}
\par\vspace{1em}
\var{cents}{}

\end{minipage}


\vspace{2em}
Variables cannot be used until they are \textit{initialized} (assigned for the first time).

\vspace{1em}
\hfill
\begin{minipage}[t]{0.5\textwidth}

\vspace{-1em}
\lstset{firstnumber=3}
\begin{javanum}
dollars = 2;
System.out.println(dollars);  // OK
System.out.println(cents);    // error
\end{javanum}

\end{minipage}
\hfill
\begin{minipage}[t]{0.4\textwidth}

\var{dollars}{2}
\par\vspace{1em}
\var{cents}{}

\end{minipage}


\vspace{2em}
Each time you assign a variable, you are \emph{updating} its value stored in memory.

\vspace{1em}
\hfill
\begin{minipage}[t]{0.5\textwidth}

\vspace{-1em}
\lstset{firstnumber=6}
\begin{javanum}
dollars = 3;
dollars = 4;
cents = 49;
\end{javanum}

\end{minipage}
\hfill
\begin{minipage}[t]{0.4\textwidth}

\var{dollars}{4}
\par\vspace{1em}
\var{cents}{49}

\end{minipage}
\vspace{1ex}


\quest{10 min}


\Q How many times is each variable in \ref{assignment.tex} assigned?

\begin{answer}
The variable \java{dollars} is assigned three times, but \java{cents} is assigned only once.
\end{answer}


\Q What is the error in the second \java{System.out.println} statement? (Don't just repeat the text in \ref{\currfilename}; explain in your own words what the problem is.)

\begin{answer}
The variable \java{cents} is not initialized, so Java does not know what value to print.
\end{answer}


\Q What is the value of \java{dollars} right before it's assigned for the last time?
What is the value of \java{cents} before it's assigned for the last time?

\begin{answer}
Just before the ~\java{dollars = 4;} statement, ~\java{dollars} is 3.
And before the ~\java{cents = 49;} statement, ~\java{cents} is uninitialized.
\end{answer}


% The following Qs are based on "Activity 02 - Declaration and Assignments" by Helen Hu


\Q Consider the statement: ~\java{cents = dollars;}

\begin{enumerate}

\item Compare this code to lines 6--8 in \ref{assignment.tex}.
What value do you think \java{cents} and \java{dollars} will have after running this statement?

\ans{The variable \java{cents} will be 4, and \java{dollars} will remain unchanged.}

\item Which side of the equals sign (left or right) was assigned a new value?

\ans{The left side.}

\end{enumerate}


\Q In Java, the \java{+} and \java{-} symbols are used to perform addition and subtraction. For example, the statement ~\java{dollars = dollars + 1;} adds one to the current value of \java{dollars}.

\begin{enumerate}

\item What is the value of \java{dollars} (in memory) after running this statement?
\ans{5}

\item Do you consider the equals sign in Java an operation to be performed? (like \java{+})
\\ If so, explain the operation. If not, explain why not.

\ans{Yes; it executes the assignment operation which stores a value in memory.}

\item Do you consider the equals sign in mathematics an operation to be performed?
\\ If so, explain the operation. If not, explain why not.

\ans{No; it simply states the proposition that two values are equal.}

\end{enumerate}


\Q \label{xgets}
In your own words, explain how you should read the \java{=} sign in Java.
For example, the Java statement ~\java{x = a + b;} should be read as ``x \_\_\_\_\_\_\_\_ a plus b.''

\begin{answer}
Answers may include ``x \emph{gets} a plus b'', ``x \emph{becomes} a plus b'', etc.
\end{answer}
