\model{Team Roles}

Decide who will be what role for today; we will rotate the roles each week.
If you have only three people, one should have two roles.
If you have five people, two may share the same role.

\begin{table}[h!]
\renewcommand{\arraystretch}{1.6}
\begin{tabular}{|p{0.45\linewidth}|p{0.50\linewidth}}
\cline{1-1}

Manager:
\hfill \ans[13em]{Helen Hu}       & ~~ \ans[18em]{keeps track of time, all voices are heard}
\\ \cline{1-1}

Presenter:
\hfill \ans[13em]{Clif Kussmaul}  & ~~ \ans[18em]{asks questions, gives the team's answers}
\\ \cline{1-1}

Recorder:
\hfill \ans[13em]{Chris Mayfield} & ~~ \ans[18em]{quality control and consensus building}
\\ \cline{1-1}

Reflector:
\hfill \ans[13em]{Aman Yadav}     & ~~ \ans[18em]{team dynamics, suggest improvements}
\\ \cline{1-1}

\end{tabular}
\end{table}


\quest{15 min}


\Q What is the difference between \textbf{bold} and \textit{italics} on the role cards?

\begin{answer}
The bold points describe what the responsibilities are.
Examples of what that person would say are in italics.
\end{answer}


\Q Manager: invite each person to explain their role to the team.
Recorder: take notes of the discussion by writing down key phrases next to the table above.

\vspace{1ex}


\Q What responsibilities do two or more roles have in common?

\begin{answer}
Both the presenter and the recorder help the team reach consensus.
The manager and reflector both monitor how the team is working.
\end{answer}


\Q For each role, give an example of how someone observing your team would know that a person is \underline{not} doing their job well.

\begin{itemize}

\item Manager:
\hfill \ans[34em]{The team is constantly getting behind.}

\item Presenter:
\hfill \ans[34em]{The student doesn't know what to say.}

\item Recorder:
\hfill \ans[34em]{Some team members aren't taking good notes.}

\item Reflector:
\hfill \ans[34em]{The student never comments on team dynamics.}

\end{itemize}
