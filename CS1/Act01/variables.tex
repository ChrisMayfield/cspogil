\model{Variables}
% Based on Model 1 of "Activity 02 - Declaration and Assignments" by Helen Hu

Most programs store and manipulate data values, and we use \emph{variables} to give them meaningful names.
The following code \emph{declares} and \emph{assigns} three variables.
Each variable is stored in the computer's memory, represented by the boxes on the right.

\vspace{1em}
\hfill
\begin{minipage}[t]{0.5\textwidth}

\textbf{Java code}
\vspace{1ex}

\begin{javalst}
int dollars;
int cents;
double grams;

dollars = 1;
cents = 90;
grams = 3.5;
\end{javalst}

\end{minipage}
\hfill
\begin{minipage}[t]{0.4\textwidth}

\textbf{Computer memory}
\vspace{1em}

\var{dollars}{1}
\par\vspace{1em}
\var{cents}{90}
\par\vspace{1em}
\var{grams}{3.5}

\end{minipage}


\quest{10 min}


\Q Identify the Java \emph{keyword} used in a variable declaration to indicate

\begin{enumerate}
\item an integer: \ans{int}
\item a real number: \ans{double}
\end{enumerate}


\Q Consider numbers of dollar bills, cents, and grams.
Which of these units only makes sense as an integer, and why?

\begin{answer}
Cents makes sense (ha ha) only as an integer, because at the end of the day, you can't pay with a fractional amount. (It is possible to make a similar argument for dollars, but not for grams.)
\end{answer}


\Q What would you expect the following statements to print out?

\begin{enumerate}
\item \java{System.out.println(dollars);} \ans{1}
\item \java{System.out.println(cents);} \ans{90}
\item \java{System.out.println(grams);} \ans{3.5}
\end{enumerate}


\Q In the previous question, how does the third printed line (c) differ from the first two?

\begin{answer}
The third line prints a double, and the first two print an integer.
\end{answer}


\Q \label{vardecl}
What do you think is the purpose of a variable declaration?

\begin{answer}
It tells the computer how to interpret and display the value.
\end{answer}


\Q What is output by the following code, and why?

\begin{minipage}[t]{0.33\linewidth}

\vspace{-2ex}
\begin{javalst}
double one;
one = 1;
System.out.println(one);
\end{javalst}

\end{minipage}
\hfill
\begin{minipage}[t]{0.66\linewidth}

\begin{answer}
The output is 1.0, because {\tt one} is a {\tt double}.
The type of variable determines the output format.
\end{answer}

\end{minipage}
