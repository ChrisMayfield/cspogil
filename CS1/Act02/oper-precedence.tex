% Based on Model 2 of "Activity 02 Declaration" by Helen Hu

\model{Order of Operations}

The Java language defines a specific order of execution for math and other operations. For example, multiplication and division take \textbf{precedence} over addition and subtraction. Using parentheses, you can override the order of operations.
The following table lists some Java operators from highest precedence to lowest precedence.

\begin{center}
\begin{tabular}{|L{3in}|L{1in}|}
\hline
Parenthesis
& \tt ( ) \\
\hline
Unary (positive or negative signs)
& \tt + - \\
\hline
Multiplicative
& \tt * / \\
\hline
Additive
& \tt + - \\
\hline
Assignment
& \tt = \\
\hline
\end{tabular}
\end{center}

For the following questions, assume you have these two variables:

\begin{javalst}
    int x;
    double y;
\end{javalst}


\quest{10 min}


\Q What operator has the lowest precedence?
Why do you think Java is designed that way?

\begin{answer}
Assignment has the lowest precedence so that all other operations happen first (before the final value is stored in memory).
\end{answer}


\Q The \java{+} and \java{-} operators show up twice in the table of operator precedence.
For the Java expression ~ \java{x = 5 * -3;} ~ explain how you know whether the \java{-} operator is being used as an unary or binary operator in this expression.

\begin{answer}
It matters what is to the left or right of the operator.
In this example, the \java{-} is preceded by a \java{*}, so it must be unary.
\end{answer}


\Q Give the order of operations in the Java expression: ~ \java{x = 5 * -3;}

\begin{enumerate}
\item First operator to be evaluated: \ans{\java{-}}
\item Second operator: \ans{\java{*}}
\item Third operator: \ans{\java{=}}
\end{enumerate}


\Q Give the order of operations in the Java expression: ~ \java{y = 9 / 2;}

\begin{enumerate}
\item First operator to be evaluated: \ans{\java{/}}
\item Second operator: \ans{\java{=}}
\end{enumerate}


\Q Based on your answer to the previous question, explain why the variable \java{y} would be assigned 4.0 (as opposed to 4 or 4.5).

\begin{answer}
If \java{y} is a floating-point variable, the integer result 4 would be stored as 4.0 in memory.
\end{answer}


\Q Rewrite the assignment for \java{y} so that it would be set correctly to 4.5. (Hint: you'll need to recall what you learned about division in \ref{dividing-numbers}.)

\begin{answer}
\java{y = 9.0 / 2.0;}
\end{answer}
