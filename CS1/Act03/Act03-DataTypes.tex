% comment out for student version
\ifdefined\Student\relax\else\def\Teacher{}\fi

\documentclass[12pt]{article}

\title{Data Types}
\author{Chris Mayfield and Stoney Jackson}
\date{Summer 2021}

%\ProvidesPackage{cspogil}

% fonts
\usepackage[utf8]{inputenc}
\usepackage[T1]{fontenc}
\usepackage{mathpazo}

% spacing
\usepackage[margin=2cm]{geometry}
\renewcommand{\arraystretch}{1.4}
\setlength{\parindent}{0pt}

% orphans and widows
\clubpenalty=10000
\widowpenalty=10000
\pagestyle{empty}

% figures and tables
\usepackage{graphicx}
\usepackage{multicol}
\usepackage{tabularx}

% fixed-width columns
\usepackage{array}
\newcolumntype{L}[1]{>{\raggedright\let\newline\\\arraybackslash\hspace{0pt}}m{#1}}
\newcolumntype{C}[1]{>{\centering\let\newline\\\arraybackslash\hspace{0pt}}m{#1}}
\newcolumntype{R}[1]{>{\raggedleft\let\newline\\\arraybackslash\hspace{0pt}}m{#1}}

% include paths
\makeatletter
\def\input@path{{Models/}{../../Models/}}
\graphicspath{{Models/}{../../Models/}}
\makeatother

% colors
\usepackage[svgnames,table]{xcolor}
\definecolor{bgcolor}{HTML}{FAFAFA}
\definecolor{comment}{HTML}{007C00}
\definecolor{keyword}{HTML}{0000FF}
\definecolor{strings}{HTML}{B20000}

% table headers
\newcommand{\tr}{\bf\cellcolor{Yellow!10}}

% syntax highlighting
\usepackage{textcomp}
\usepackage{listings}
\lstset{
    basicstyle=\ttfamily,
    backgroundcolor=\color{bgcolor},
    numberstyle=\scriptsize\color{comment},
    commentstyle=\color{comment},
    keywordstyle=\color{keyword},
    stringstyle=\color{strings},
    columns=fullflexible,
    keepspaces=true,
    showstringspaces=false,
    upquote=true
}

% code environments
\newcommand{\java}[1]{\lstinline[language=java]{#1}}%[
\lstnewenvironment{javalst}{\lstset{language=java,backgroundcolor=}}{}
\lstnewenvironment{javabox}{\lstset{language=java,frame=single,numbers=left}\quote}{\endquote}

% PDF properties
\usepackage[pdftex]{hyperref}
\urlstyle{same}
\makeatletter
\hypersetup{
  pdftitle={\@title},
  pdfauthor={\@author},
  pdfsubject={\@date},
  pdfkeywords={},
  bookmarksopen=false,
  colorlinks=true,
  citecolor=black,
  filecolor=black,
  linkcolor=black,
  urlcolor=blue
}
\makeatother

% titles
\makeatletter
\renewcommand{\maketitle}{\begin{center}\LARGE\@title\end{center}}
\makeatother

% boxes
\newcommand{\emptybox}[1][10em]{
\vspace{1em}
\begin{tabularx}{\linewidth}{|X|}
\hline\\[#1]\hline
\end{tabularx}}

% models
\newcommand{\model}[1]{\section{#1}\nopagebreak}
\renewcommand{\thesection}{Model~\arabic{section}}

% questions
\newcommand{\quest}[1]{\subsection*{Questions~ (#1)}}
\newcounter{question}
\newcommand{\Q}{\vspace{1em}\refstepcounter{question}\arabic{question}.~ }
\renewcommand{\thequestion}{\#\arabic{question}}

% sub-question lists
\usepackage{enumitem}
\setenumerate[1]{label=\alph*)}
\setlist{itemsep=1em,after=\vspace{1ex}}

% inline answers
\definecolor{answers}{HTML}{C0C0C0}
\newcommand{\ans}[1]{%
\ifdefined\Student
    \phantom{~~\textcolor{answers}{#1}}
\else
    ~~\textcolor{answers}{#1}
\fi}

% longer answers [optional height]
\newsavebox{\ansbox}
\newenvironment{answer}[1][4em]{
\nopagebreak
\begin{lrbox}{\ansbox}
\begin{minipage}[t][#1]{\linewidth}
\color{answers}
}{
\end{minipage}
\end{lrbox}
\ifdefined\Student
    \phantom{\usebox{\ansbox}}%
\else
    \usebox{\ansbox}%
\fi}


\begin{document}

\maketitle

Java supports two main types of data: \emph{primitive types} like \java{int} and \java{double} that represent a single value, and \emph{reference types} like \java{String} and \java{Scanner} that represent more complex data.

\rolenames

\guide{
  \item Explain how using roles improves the team's success.
  \item Name Java's primitive data types and give examples of each one.
  \item Identify illegal assignment statements, and explain why they are illegal.
  \item Describe what it means for variables to store a reference to an object.
}{
  \item Providing feedback on how well other team members are working. (Teamwork)
}{
The meta activity reinforces the importance of roles.
Successful teams are able to accomplish all the tasks outlined on the \github{Handouts/role-cards-mayfield.pdf}{Role Cards}, and there's too many things for one person to keep track.
Ask the reflectors to pay special attention to their role during today's activity, and invite them at some point to report to share what they have observed.

On \ref{literals}, students may struggle with letters used in primitive values (i.e., F and L).
Rather than just say what everything means, direct students back to \ref{primitive-types.tex} and have them read out loud the type names.
%Explain that \texttt{float} is a floating-point data type, and ask students to guess where the name \texttt{double} comes from.
\ref{allowed} is good for report-out, allowing students to collectively develop a rule for when an assignment is allowed.
The solution files \sol{Act03}{Assign1.java} and \sol{Act03}{Assign2.java} are available for demonstration.
When reporting out, introduce the term \emph{literal}.

%It might be helpful to have a couple of examples handy to challenge their assumptions.
%But be careful, examples like \texttt{short x = 3; byte y = 3; char c = 3;} may open a can of worms.
%If these issues come up, you'll need to explain that \texttt{javac} will check the range of the value, and if it is acceptable, allow the assignment (i.e., \texttt{byte z = 255;} will fail).

\ref{reference-types.tex} introduces a technique for drawing memory diagrams to show the difference between primitive and reference types.
On \ref{twostrs}, have multiple teams draw their diagrams on the board.
Address misconceptions that arise (e.g., drawing multiple string objects).
It might be helpful at this point to direct students to \href{https://cscircles.cemc.uwaterloo.ca/java_visualize/}{Java Visualizer}, which draws similar diagrams.
Select the option ``Show String/Integer/etc objects, not just values'' for it to display references as arrows.

Key questions: \ref{allowed}, \ref{varval}, \ref{twostrs}

Source files: \src{Act03}{TaylorSwift.java}
}

\model{Team Disruptions}
% Based on Model 4 of "We Are a Learning Team!" by Urik Halliday

Common disruptions to learning in teams include:
  talking about topics that are off-task,
  teammates answering questions on their own,
  entire teams working alone,
  limited or no communication between teammates,
  arguing or being disrespectful,
  rushing to complete the activity,
  not being an active teammate,
  not coming to a consensus about an answer,
  writing incomplete answers or explanations,
  ignoring ideas from one or more teammates.


\quest{10 min}


\Q Pick four of the disruptions listed above.
For each one, find something from the role cards that could help improve the team's success.
Use a different role for each disruption.

\begin{enumerate}
\item Manager: \ans{limited communication between teammates}
\vspace{2em}
\item Presenter: \ans{ignoring ideas from one or more teammates}
\vspace{2em}
\item Recorder: \ans{writing incomplete answers or explanations}
\vspace{2em}
\item Reflector: \ans{teammates answering questions on their own}
\vspace{2em}
\end{enumerate}

\begin{center}
\includegraphics[height=2.85in]{disrupt1.png}
\end{center}

\newpage
\model{Primitive Types}

\vspace{-1ex}
\begin{table}[h!]
\begin{tabularx}{\linewidth}{|X|X|X|X|l|}
\hline
\tr Keyword    & \tr Size & \tr Min Value          & \tr Max Value         & \tr Example              \\
\hline
\java{byte}    & 1 byte   & $-128$                 & $127$                 & \java{(byte) 123}        \\
\hline
\java{short}   & 2 bytes  & $-32,768$              & $32,767$              & \java{(short) 12345}     \\
\hline
\java{int}     & 4 bytes  & $-2^{31}$              & $2^{31}-1$            & \java{1234567890}        \\
\hline
\java{long}    & 8 bytes  & $-2^{63}$              & $2^{63}-1$            & \java{123456789012345L}  \\
\hline
\java{float}   & 4 bytes  & $-3.4 \times 10^{38}$  & $3.4 \times 10^{38}$  & \java{3.14159F}          \\
\hline
\java{double}  & 8 bytes  & $-1.8 \times 10^{308}$ & $1.8 \times 10^{308}$ & \java{3.141592653589793} \\
\hline
\java{boolean} & 1 byte   & N/A                    & N/A                   & \java{true}              \\
\hline
\java{char}    & 2 bytes  & 0                      & 65,535                & \java{'A'}               \\
\hline
\end{tabularx}
\end{table}

Note that 1 byte is 8 bits, i.e., eight ``ones and zeros'' in computer memory.
Since there are only two possible values for each bit, you can represent $2^8 = 256$ possible values with 1 byte.


\quest{15 min}


\Q Which of the primitive types are integers? Which are floating-point?

\begin{answer}
Integers: byte, short, int, long.
Floating-point: float, double.
\end{answer}


\Q Why do primitive types have ranges of values? What determines the range of the data type?

\begin{answer}
The range of values depends on the size, i.e., how many bytes are used to store the value.
\end{answer}


\Q Why can't computers represent every possible number in mathematics? Will they ever be able to do so?

\begin{answer}
Computers have finite memory, but there are an infinite number of numbers.
There will always be a number larger than what computers can store.
\end{answer}


\Q Since a \java{byte} can represent 256 different numbers, why is its max value 127 and not 128?

\begin{answer}
One of the 256 values is the number zero. So 128 negatives, plus 1 zero, plus 127 positives equals 256 values.
\end{answer}


\newpage

\Q \label{literals}
What is the data type for each of the following values?

\begin{quote}
\begin{tabular}{R{4em}L{6em}R{4em}L{6em}R{4em}L{6em}}

1.14159   & \ans[5em]{double}  &
7.2E-4    & \ans[5em]{double}  &
-128      & \ans[5em]{int}     \\

0         & \ans[5em]{int}     &
0.0       & \ans[5em]{double}  &
\qs{0}\qs & \ans[5em]{char}    \\

-1.0F     & \ans[5em]{float}   &
-13L      & \ans[5em]{long}    &
false     & \ans[5em]{boolean} \\

123       & \ans[5em]{int}     &
\qs{H}\qs & \ans[5em]{char}    &
true      & \ans[5em]{boolean} \\

\end{tabular}
\end{quote}


\Q \label{allowed}
Based on the examples below, when does Java allow you to assign one type of primitive variable to another?

% Based on example from Stoney Jackson
\begin{multicols}{2}
\begin{javalst}
    int int_ = 3;
    long long_ = 3L;
    float float_ = 3.0F;
    double double_ = 3.0;
    
    int_ = int_;
    int_ = long_;      // illegal
    int_ = float_;     // illegal
    int_ = double_;    // illegal
    
    long_ = int_;
    long_ = long_;
    long_ = float_;    // illegal
    long_ = double_;   // illegal

    float_ = int_;
    float_ = long_;
    float_ = float_;
    float_ = double_;  // illegal
    
    double_ = int_;
    double_ = long_;
    double_ = float_;
    double_ = double_;
    
    int_ = '0';
    int_ = false;      // illegal
    double_ = '0';
    double_ = false;   // illegal
\end{javalst}
\end{multicols}

\vspace{-1em}
\begin{answer}
The types have to be compatible (e.g., you can't assign numeric to boolean), and you can only assign from smaller to larger (e.g., from float to double, or int to double).
\end{answer}


\Q Given the following variable declarations, which of the assignments are not allowed?

\begin{quote}
\begin{multicols}{3}

\begin{javalst}
byte miles;
short minutes;
int checking;
long days;
float total;
double sum;
boolean flag;
char letter;
\end{javalst}

\columnbreak

\begin{javalst}
checking = 56000;
total = 0;
sum = total;
total = sum;
checking = miles;
sum = checking;
flag = minutes;
days = '0';
\end{javalst}

\columnbreak

\begin{answer}[10em]
All are okay except:
\\ \hspace*{1em} \texttt{total = sum;}
\\ \hspace*{1em} \texttt{flag = minutes;}
%\\ \hspace*{1em} \texttt{days = \chr{0};}

\medskip
Note that assigning \chr{0} to \java{days} is legal,
but the value is actually stored as 48L (Unicode for the digit zero character).
\end{answer}

\end{multicols}
\end{quote}


%\Q Based on your answer to \ref{allowed}, list all possible assignments for \java{int}, \java{long}, \java{float}, and \java{double} values in this format: \java{long} $\gets$ \java{int}
%
%\begin{answer}[5em]
%\setlength{\tabcolsep}{12pt}
%\begin{tabular}{llll}
%&  long   $\gets$ int  &                       &                        \\
%&  float  $\gets$ int  &  float  $\gets$ long  &                        \\
%&  double $\gets$ int  &  double $\gets$ long  &  double $\gets$ float  \\
%\end{tabular}
%\end{answer}

\newpage
\model{Reference Types}


\quest{10 min}


\Q 

\begin{answer}
\end{answer}


\end{document}
