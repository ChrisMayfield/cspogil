\documentclass[12pt]{article}

\title{Activity 4: Boolean Logic}
\author{Chris Mayfield and Helen Hu}
\date{July 2017}

%\ProvidesPackage{cspogil}

% fonts
\usepackage[utf8]{inputenc}
\usepackage[T1]{fontenc}
\usepackage{mathpazo}

% spacing
\usepackage[margin=2cm]{geometry}
\renewcommand{\arraystretch}{1.4}
\setlength{\parindent}{0pt}

% orphans and widows
\clubpenalty=10000
\widowpenalty=10000
\pagestyle{empty}

% figures and tables
\usepackage{graphicx}
\usepackage{multicol}
\usepackage{tabularx}
\usepackage{wrapfig}

% fixed-width columns
\usepackage{array}
\newcolumntype{L}[1]{>{\raggedright\let\newline\\\arraybackslash\hspace{0pt}}m{#1}}
\newcolumntype{C}[1]{>{\centering\let\newline\\\arraybackslash\hspace{0pt}}m{#1}}
\newcolumntype{R}[1]{>{\raggedleft\let\newline\\\arraybackslash\hspace{0pt}}m{#1}}

% include paths
\makeatletter
\def\input@path{{Models/}{../../Models/}}
\graphicspath{{Models/}{../../Models/}}
\makeatother

% colors
\usepackage[svgnames,table]{xcolor}
\definecolor{bgcolor}{HTML}{FAFAFA}
\definecolor{comment}{HTML}{007C00}
\definecolor{keyword}{HTML}{0000FF}
\definecolor{strings}{HTML}{B20000}

% table headers
\newcommand{\tr}{\bf\cellcolor{Yellow!10}}

% syntax highlighting
\usepackage{textcomp}
\usepackage{listings}
\lstset{
    basicstyle=\ttfamily\color{black},
    backgroundcolor=\color{bgcolor},
    numberstyle=\scriptsize\color{comment},
    commentstyle=\color{comment},
    keywordstyle=\color{keyword},
    stringstyle=\color{strings},
    columns=fullflexible,
    keepspaces=true,
    showlines=true,
    showstringspaces=false,
    upquote=true
}

% code environments
\newcommand{\java}[1]{\lstinline[language=java]{#1}}%[
\lstnewenvironment{javalst}{\lstset{language=java,backgroundcolor=}}{}
\lstnewenvironment{javabox}{\lstset{language=java,frame=single,numbers=left}\quote}{\endquote}

% PDF properties
\usepackage[pdftex]{hyperref}
\urlstyle{same}
\makeatletter
\hypersetup{
  pdftitle={\@title},
  pdfauthor={\@author},
  pdfsubject={\@date},
  pdfkeywords={},
  bookmarksopen=false,
  colorlinks=true,
  citecolor=black,
  filecolor=black,
  linkcolor=black,
  urlcolor=blue
}
\makeatother

% titles
\makeatletter
\renewcommand{\maketitle}{\begin{center}\LARGE\@title\end{center}}
\makeatother

% boxes [optional height]
\newcommand{\emptybox}[1][10em]{
\vspace{1em}
\begin{tabularx}{\linewidth}{|X|}
\hline\\[#1]\hline
\end{tabularx}}

% models
\newcommand{\model}[1]{\section{#1}\nopagebreak}
\renewcommand{\thesection}{Model~\arabic{section}}

% questions
\newcommand{\quest}[1]{\subsection*{Questions~ (#1)}}
\newcounter{question}
\newcommand{\Q}{\vspace{1em}\refstepcounter{question}\arabic{question}.~ }
\renewcommand{\thequestion}{\#\arabic{question}}

% sub-question lists
\usepackage{enumitem}
\setenumerate[1]{label=\alph*)}
\setlist{itemsep=1em,after=\vspace{1ex}}

% inline answers
\definecolor{answers}{HTML}{C0C0C0}
\newcommand{\ans}[1]{%
\ifdefined\Student
    \leavevmode\phantom{~~\textcolor{answers}{#1}}
\else
    ~~\textcolor{answers}{#1}
\fi}

% longer answers [optional height]
\newsavebox{\ansbox}
\newenvironment{answer}[1][4em]{
\nopagebreak
\begin{lrbox}{\ansbox}
\begin{minipage}[t][#1]{\linewidth}
\color{answers}
}{
\end{minipage}
\end{lrbox}
\ifdefined\Student
    \phantom{\usebox{\ansbox}}%
\else
    \usebox{\ansbox}%
\fi}


\begin{document}

\maketitle

The primitive data type \java{boolean} has two values: \java{true} and \java{false}.
Boolean expressions are built using \emph{relational operators} and \emph{conditional operators}.

\guide{
  \item Recognize the value of developing process skills.
  \item Evaluate boolean expressions with relational operators (<, >, <=, >=, ==, !=).
  \item Explain the difference between assignment (=) and equality (==) operators.
  \item Evaluate boolean expressions that involve comparisons with \&\&, ||, and !.
}{
  \item Evaluating complex logic expressions based on operator precedence. (Critical Thinking)
}{
\ref{employers.tex} is ultimately about process skills and should help with student buy-in.
If you are using the \github{Handouts/role-cards-mayfield.pdf}{Role Cards}, have students look at the definitions on the reverse side.
Each POGIL activity targets specific ``process skill goals'' from these categories.

\ref{relational.tex} mentions DrJava, but it can be replaced with another IDE or \href{http://www.javarepl.com/}{Java REPL}.
Give students about three minutes to fill in the table without using a computer.
Then show them the actual results interactively (or in an example program) on the projector.

When reporting out, ask students to explain what \emph{expressions} are and how they differ from \emph{statements}.
Reinforce what it means to \emph{evaluate} an expression (i.e., compute a single value) versus \emph{execute} a statement (i.e., run an entire line of code).

During \ref{conditional.tex}, explain that the variables $p$ and $q$ are often used to represent logic values in discrete math.
Make sure students understand that ! is a \emph{unary} operator, and that \java{&&} and \java{||} are \emph{binary} operators.
}

\model{What Employers Want}

``What do employers look for when they are seeking new college graduates to take on jobs?
According to NACE's \textit{Job Outlook 2016} survey, they are looking for leaders who can work as part of a team.''
{\footnotesize \url{http://www.naceweb.org/s11182015/employers-look-for-in-new-hires.aspx}}

\begin{table}[h!]
\centering

{\bf Attributes employers seek on a candidate's resume}
\vspace{2pt}

\renewcommand{\arraystretch}{1.0}
\begin{tabular}{|l|l|c|}
\hline
\tr & \tr Attribute   & \tr \% of respondents \\
\hline
1.  & Leadership                     & 80.1\% \\
\hline
2.  & Ability to work in a team      & 78.9\% \\
\hline
3.  & Communication skills (written) & 70.2\% \\
\hline
4.  & Problem-solving skills         & 70.2\% \\
\hline
5.  & Communication skills (verbal)  & 68.9\% \\
\hline
6.  & Strong work ethic              & 68.9\% \\
\hline
7.  & Initiative                     & 65.8\% \\
\hline
8.  & Analytical/quantitative skills & 62.7\% \\
\hline
9.  & Flexibility/adaptability       & 60.9\% \\
\hline
10. & Technical skills               & 59.6\% \\
\hline
\end{tabular}

%\vspace{1ex}
%\footnotesize
%Source: \textit{Job Outlook 2016}, National Association of Colleges and Employers
\end{table}
\vspace{-1em}


\quest{10 min}


\Q What is the relationship between the top two attributes employers seek?

\begin{answer}[3em]
Leadership implies working with other people, most likely as part of a team.
Working in teams is essential to developing leadership skills.
\end{answer}


\Q How is communication (written and verbal) related to problem-solving?

\begin{answer}[3em]
Solving problems in teams involves talking to other people and trying different approaches.
Writing solutions down is necessary to solidify the details and share them with others.
\end{answer}


\Q As a team, come up with a short description/example of each attribute.

\begin{quote}
\begin{multicols}{2}
1. \ans[16em]{leading a group of people} \\[1ex]
2. \ans[16em]{geting along well with others} \\[1ex]
3. \ans[16em]{writing and reading effectively} \\[1ex]
4. \ans[16em]{finding solutions creatively} \\[1ex]
5. \ans[16em]{speaking and listening effectively}

6.~ \ans[16em]{self-motivated to work hard} \\[1ex]
7.~ \ans[16em]{acting or taking charge early} \\[1ex]
8.~ \ans[16em]{analyzing data and reasoning} \\[1ex]
9.~ \ans[16em]{being able to handle change} \\[1ex]
10. \ans[16em]{computer/technology literacy}
\end{multicols}
\end{quote}

\vspace{-1ex}


\Q Which of these skills do you expect to develop in this course? Why?

\begin{answer}[3em]
Ideally, all of them. Students will definitely learn technical computer programming skills.
But working in teams provides the opportunity to develop many other employable skills.
\end{answer}

% based on Model 2 of Activity 4 Boolean by Helen Hu

\model{Relational Operators}

When you omit the semicolon in the Interactions pane, DrJava will display the \emph{value} of the expression you have entered.
In the table below, predict what values DrJava will display and identify the relational operator (the first four rows are completed for you).

%TODO try to rework w/o DrJava; Interactions w/o println confuses students

\begin{center}
\begin{tabular}{|L{200pt}|C{110pt}|C{110pt}|}
\hline
\tr Interactions & \tr Value displayed & \tr Relational operator \\
\hline
\java{int three = 3}                   & none        & none            \\
\hline
\java{int four = 4}                    & none        & none            \\
\hline
\java{System.out.println(four)}        & 4           & none            \\
\hline
\java{three > four}                    & false       & \java{>}        \\
\hline
\java{boolean isLarger = three > four} & \ans{none}  & \ans{\java{>}}  \\
\hline
\java{System.out.println(isLarger)}    & \ans{false} & \ans{none}      \\
\hline
\java{three == four}                   & \ans{false} & \ans{\java{==}} \\
\hline
\java{three < four}                    & \ans{true}  & \ans{\java{<}}  \\
\hline
\java{three <= four}                   & \ans{true}  & \ans{\java{<=}} \\
\hline
\java{three = four}                    & \ans{4}     & \ans{none}      \\
\hline
\java{three == four}                   & \ans{true}  & \ans{\java{==}} \\
\hline
\end{tabular}
\end{center}


\quest{10 min}

\Q List the four unique \emph{boolean expressions} used in the model.

\begin{answer}[3em]
\begin{javaans}
    three > four    three == four    three < four    three <= four
\end{javaans}
\end{answer}


\Q Examine the fifth line of Java code in the above model.

\begin{enumerate}

\item What three actions are performed in this single line of code?
\ans{It declares the variable \java{isLarger}, compares the values of \java{three} and \java{four}, and assigns the result to \java{isLarger}.}

\item Write two lines of code, ending with semicolons, that would perform these same actions (but in two lines instead of a single line).

\ans{\tt boolean isLarger;} \\
\ans{\tt isLarger = three > four;}

\end{enumerate}


\Q The \java{!=} operator means ``not equals''.
Give an example of a boolean expression that uses \java{!=} and evaluates to false.

\begin{answer}[3em]
\java{5 != 5} is false (because they \emph{are} equal)
\end{answer}


\Q Explain why the same boolean expression \java{three == four} resulted with two different boolean values in the table.

\begin{answer}
The line \java{three = four} assigned the value of four to three, making the two variables equal.
They started out not being equal, but they ended up with the same value.
\end{answer}


\Q What is the difference between \java{=} and \java{==} in Java?

\begin{answer}
The \java{=} operator assigns a value to a variable, and the \java{==} operator compares two values.
\end{answer}


\Q List the six relational operators that can be used in a boolean expression.
(Five have been used so far, but you should be able to guess the sixth.)
Explain briefly what each one means.

\begin{answer}
\begin{tabular}{lll}
\java{<} is less than              & \java{>} is greater than              & \java{==} is equal to     \\
\java{<=} is less than or equal to & \java{>=} is greater than or equal to & \java{!=} is not equal to \\
\end{tabular}
\end{answer}

\model{Conditional Operators}
% based on Model 3 of Activity 4 Boolean by Helen Hu

Boolean expressions, like \java{written > problem} and \java{teamwork < 75.0}, can be combined using the \emph{conditional operators}:

\begin{center}
\begin{tabular}{|C{100pt}|C{100pt}|}
\hline
\tr Operator & \tr Meaning \\
\hline
\java{!}  & not \\
\hline
\java{&&} & and \\
\hline
\java{||} & or \\
\hline
\end{tabular}
\end{center}

For example, \java{written > problem && teamwork < 75.0} is false, because \java{teamwork} is not less than \java{75.0}.
(Both conditions need to be true in order for \java{&&} to be true.)

\bigskip

The following table summarizes the result of \java{&&}, \java{||}, and \java{!} for all possible inputs.
The variables \java{p} and \java{q} represent conditions like \java{written > problem} and \java{teamwork < 75.0}.

\begin{center}
\begin{tabular}{|C{50pt}|C{50pt}||C{50pt}|C{50pt}|C{50pt}|}
\hline
\tr \java{p} & \tr \java{q} & \tr \java{p && q} & \tr \java{p || q} & \tr \java{!p} \\
\hline
false & false & false & false & true  \\ \hline  % \multirow{2}{*}{true}  \\ \cline{1-4}
false & true  & false & true  & true  \\ \hline  %                        \\ \hline
true  & false & false & true  & false \\ \hline  % \multirow{2}{*}{false} \\ \cline{1-4}
true  & true  & true  & true  & false \\ \hline  %                        \\ \hline
\end{tabular}
\end{center}


\quest{20 min}


\Q Consider the following variables:

\begin{quote}
\begin{javalst}
double initiative = 74.2;
double analytical = 71.9;
double workEthic  = 70.8;
boolean hired = true;
boolean fired = false;
\end{javalst}
\end{quote}

What are the results (true or false) of the following expressions?

\begin{quote}
\begin{tabular}{|l|l|}
\hline
\tr Expression & \tr Result \\
\hline
\java{!fired} & \ans[3em]{true} \\
\hline
\java{!(workEthic < initiative)} & \ans[3em]{false} \\
\hline
\java{workEthic < 71.0 && 71.0 < initiative} & \ans[3em]{true} \\
\hline
\java{initiative < 70.0 || workEthic > 70.0} & \ans[3em]{true} \\
\hline
\java{fired || workEthic < 50.0} & \ans[3em]{false} \\
\hline
\java{analytical < initiative && fired} & \ans[3em]{false} \\
\hline
\java{hired && !fired} & \ans[3em]{true} \\
\hline
\end{tabular}
\end{quote}


\Q Write a boolean expression that \ldots

\begin{enumerate}
\item uses \java{initiative}, \java{analytical}, and \java{!}, and evaluates to false. \ans[12em]{!(analytical < initiative)}
\item uses \java{analytical}, \java{workEthic}, and \java{!}, and evaluates to true. \ans[12em]{!(workEthic > analytical)}
\item uses any variable(s), and evaluates to false. \ans[12em]{fired}
\item uses any variable(s), and evaluates to true. \ans[12em]{hired}
\end{enumerate}


\Q Using your answers to the previous question, write a boolean expression ``\java{p && q}'' where \java{p} is your answer to part a) and \java{q} is your answer to part b).

\begin{enumerate}
\item Your expression: \ans[30em]{\java{!(analytical < initiative) && !(workEthic > analytical)}}
\item Result of ~\java{p && q}: \ans{false (no matter what)}
\end{enumerate}


\comment{
Relational operators (\java{<}, \java{>}, and \java{==}) are evaluated before conditional operators (\java{!}, \java{&&}, and \java{||}).
When multiple conditional operators are used, Java evaluates \java{!} first, then \java{&&}, and finally \java{||}.
}


%\Q \label{truthtable} Complete the following table, which explores all possible values for \java{p} and \java{q}:
%
%\begin{center}
%\begin{tabular}{|C{50pt}|C{50pt}|C{50pt}|C{50pt}|C{50pt}|}
%\hline
%\tr \java{p} & \tr \java{q} & \tr \java{p && q} & \tr \java{p || q} & \tr \java{!p} \\
%\hline
%false & false & \ans[3em]{false} & \ans[3em]{false} & \ans[3em]{true}  \\
%\hline
%false & true  & \ans[3em]{false} & \ans[3em]{true}  & \ans[3em]{true}  \\
%\hline
%true  & false & \ans[3em]{false} & \ans[3em]{true}  & \ans[3em]{false} \\
%\hline
%true  & true  & \ans[3em]{true}  & \ans[3em]{true}  & \ans[3em]{false} \\
%\hline
%\end{tabular}
%\end{center}


\Q Show the intermediate result of each operator below.
In other words, show your work as you evaluate the code in the same order that Java would.

\begin{center}
\java{!(initiative < analytical) && workEthic > analytical}
\vspace{1em}

\begin{tabular}{|C{50pt}|C{70pt}|C{200pt}|C{70pt}|}
\hline
\tr & \tr Operator & \tr Expression & \tr Result \\
\hline
1st & \java{<}  & \java{initiative < analytical} & false \\
\hline
2nd & \ans[3em]{\java{!}}  & \ans{\java{!}false} & \ans[3em]{true} \\
\hline
3rd & \ans[3em]{\java{>}}  & \ans{\java{workEthic > analytical}} & \ans[3em]{false} \\
\hline
4th & \ans[3em]{\java{&&}} & \ans{true \java{&&} false} & \ans[3em]{false} \\
\hline
\end{tabular}
\end{center}


\Q Change the parentheses in the original expression (from the previous question) so that the \java{&&} is evaluated before the \java{!}.
Then remove any unnecessary parentheses.

\begin{enumerate}
\item Expression: \ans[30em]{\java{!(initiative < analytical && workEthic > analytical)}}
\item New result: \ans[5em]{true}
\end{enumerate}


\Q Review the table from \ref{\currfilename} for evaluating \java{&&} and \java{||}.
Looking only at the \java{p} and \java{&&} columns, when is it necessary to examine \java{q} to determine how \java{p && q} should be evaluated?

\begin{answer}
You only need to look at \java{q} when \java{p} is true.
If \java{p} is false, you know the expression will be false.
\end{answer}


\Q Review the table from \ref{\currfilename} for evaluating \java{&&} and \java{||}.
Looking only at the \java{p} and \java{||} columns, when is it necessary to examine \java{q} to determine how \java{p || q} should be evaluated?

\begin{answer}
You only need to look at \java{q} when \java{p} is false.
If \java{p} is true, you know the expression will be true.
\end{answer}


\Q In Java, \java{&&} and \java{||} are \emph{short circuit} operators, meaning they evaluate only what is necessary.
If the expression \java{p} is more likely to be true than the expression \java{q}, which one should you place on the left of each operator to avoid doing extra work?

\begin{enumerate}
\item left of the \java{&&} expression:
\ans[25em]{\java{q} --- if it's false, then \java{p} won't be evaluated}
\item left of the \java{||} expression:
\ans[25em]{\java{p} --- if it's true, then \java{q} won't be evaluated}
\end{enumerate}


\Q What is the result of the following expressions?
\begin{enumerate}
\item \java{1 + 0 > 0 && 1 / 0 > 0}
\ans[24em]{java.lang.ArithmeticException: / by zero}
\item \java{1 + 0 > 0 || 1 / 0 > 0}
\ans[24em]{true}
\end{enumerate}


\end{document}
