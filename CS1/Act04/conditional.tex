% based on Model 3 of Activity 4 Boolean by Helen Hu

\model{Conditional Operators}

Boolean expressions may also use conditional operators to implement basic logic.
Relational operators are always executed first, so there is generally no need for parentheses.

\begin{center}
\begin{tabular}{|C{100pt}|C{100pt}|}
\hline
\tr Operator & \tr Meaning \\
\hline
\java{!}  & not \\
\hline
\java{&&} & and \\
\hline
\java{||} & or \\
\hline
\end{tabular}
\end{center}

If all three operators appear in the same expression, Java will evaluate the \java{!} first, then \java{&&}, and finally \java{||}.
If there are multiples of the same operator, they are evaluated from left to right.

\smallskip
\begin{multicols}{2}
\centering

\textbf{Example Variables:} \\[1ex]
\java{int a = 3;} \\
\java{int b = 4;} \\
\java{int c = 5;} \\
\java{boolean funny = true;} \\
\java{boolean weird = false;} \\

\columnbreak

\textbf{Example Expressions:} \\[1ex]
\java{a < b && funny} \\
\java{a < b && b < c} \\
\java{c < a || b < a} \\
\java{funny && a < c} \\
\java{!funny || weird} \\

\end{multicols}


\quest{15 min}


\Q What are the values (true or false) of the example expressions?
\ans{true, true, false, true, false}

\vspace{1em}


\Q Give different examples of boolean expressions that:

\begin{enumerate}
\item uses a, b, and !, and evaluates to false \ans{!(a < b)}
\item uses b, c, and !, and evaluates to true \ans{!(b > c)}
\item uses any variables, but evaluates to false \ans{weird}
\item uses any variables, but evaluates to true \ans{funny}
\end{enumerate}


\Q Using your answers from the previous question, write the boolean expression \java{p && q} where \java{p} is your answer to step a) and \java{q} is your answer to step b).

\begin{enumerate}
\item Your expression: \ans{\java{!(a < b) && !(b > c)}}
\item Result of ~\java{p && q}: \ans{false (no matter what)}
\end{enumerate}


\Q \label{truthtable} Complete the following table:

\begin{center}
\begin{tabular}{|C{50pt}|C{50pt}|C{50pt}|C{50pt}|C{50pt}|}
\hline
\tr \java{p} & \tr \java{q} & \tr \java{p && q} & \tr \java{p || q} & \tr \java{!p} \\
\hline
false & false & \ans{false} & \ans{false} & \ans{true}  \\
\hline
false & true  & \ans{false} & \ans{true}  & \ans{true}  \\
\hline
true  & false & \ans{false} & \ans{true}  & \ans{false} \\
\hline
true  & true  & \ans{true}  & \ans{true}  & \ans{false} \\
\hline
\end{tabular}
\end{center}


\Q Using the values in \ref{\currfilename}, give the result of each operator in the following expression.
In other words, show your work as you evaluate the code in the same order that Java would.

\begin{center}
\java{!(a > c) && b > c}
\vspace{1em}

\begin{tabular}{|C{50pt}|C{70pt}|C{200pt}|C{70pt}|}
\hline
\tr & \tr Operator & \tr Expression & \tr Result \\
\hline
1st & \java{>}  & \java{a > c} & false \\
\hline
2nd & \ans{\java{!}}  & \ans{\java{!}false} & \ans{true} \\
\hline
3rd & \ans{\java{>}}  & \ans{\java{b > c}}  & \ans{false} \\
\hline
4th & \ans{\java{&&}} & \ans{true \java{&&} false} & \ans{false} \\
\hline
\end{tabular}
\end{center}


\Q Add parentheses to the boolean expression from the previous question so that the \java{&&} is evaluated before the \java{!}. Then remove any unnecessary parentheses.

\begin{enumerate}
\item Expression: \ans{\java{!(a > c && b > c)}}
\item New result: \ans{true}
\end{enumerate}


\Q Review the table from \ref{truthtable} for evaluating \java{&&} and \java{||}.
Looking only at the \java{p} and \java{&&} columns, when is it necessary to examine \java{q} to determine how \java{p && q} should be evaluated?

\begin{answer}
You only need to look at \java{q} when \java{p} is true.
If \java{p} is false, you know the expression will be false.
\end{answer}


\Q Review the table from \ref{truthtable} for evaluating \java{&&} and \java{||}.
Looking only at the \java{p} and \java{||} columns, when is it necessary to examine \java{q} to determine how \java{p || q} should be evaluated?

\begin{answer}
You only need to look at \java{q} when \java{p} is false.
If \java{p} is true, you know the expression will be true.
\end{answer}


\Q In Java, \java{&&} and \java{||} are \emph{short circuit} operators, meaning they evaluate only what is necessary.
If the expression \java{p} is more likely to be true than the expression \java{q}, which one should you place on the left of each operator to avoid doing extra work?

\begin{enumerate}
\item left of the \java{&&} expression: \ans{\java{q} --- if it's false, then \java{p} won't be evaluated}
\item left of the \java{||} expression: \ans{\java{p} --- if it's true, then \java{q} won't be evaluated}
\end{enumerate}
