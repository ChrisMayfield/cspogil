\model{Invoke and Return}

Each statement in this program \emph{invokes} (or calls) a method.
At the end of a method, Java \emph{returns} to where it was invoked.
The list of events on the right illustrates how the program runs.

\vspace{3ex}

\begin{minipage}{0.68\textwidth}
\begin{javabox}
public class Model {
    
    public static void main(String[] args) {
        System.out.println("First line.");
        threeLine();
        System.out.println("Second line.");
    }
    
    public static void newLine() {
        System.out.println();
    }
    
    public static void threeLine() {
        newLine();
        newLine();
        newLine();
    }
    
}
\end{javabox}
\end{minipage}
\hfill
\begin{minipage}{0.34\textwidth}
\begin{verbatim}

INVOKE println
RETURN to line 5
INVOKE threeLine
    INVOKE newLine
        INVOKE println
        RETURN to line 11
    RETURN to line 15
    INVOKE newLine
        INVOKE println
        RETURN to line 11
    RETURN to line 16
    INVOKE newLine
        INVOKE println
        RETURN to line 11
    RETURN to line 17
RETURN to line 6
INVOKE println
RETURN to line 7

\end{verbatim}
\end{minipage}


\quest{15 min}


\Q \label{lines}
How many lines of source code invoke the \java{println} method? \ans{Three (lines 4, 6, 10)}
\vspace{1em}


\Q \label{times}
How many times is \java{println} invoked when the program runs? \ans{Five times}
\vspace{1em}


\Q For each \texttt{INVOKE} on the right, draw an arrow to the corresponding line of code.
(Plan ahead so that crossing lines will still be legible.) \ans{There should be 9 lines total.}
\vspace{1em}


\Q What is the output of the program? Please write \java{\\n} to show each newline character.

\begin{answer}[8em]
\vspace{-1ex}
\begin{javaans}
First line.\n
\n
\n
\n
Second line.\n
\end{javaans}
\end{answer}


\Q When Java sees a name like \java{x}, \java{count}, or \java{newLine}, how can it tell whether it's a variable or a method? (Hint: syntax)

\begin{answer}[5em]
Methods have parentheses, and variables do not.
They are similar to functions in math  : when you see $g(x)$, you know that $g$ is a function and $x$ is a variable.
\end{answer}


\Q What is the difference between a method and a variable? What do they have in common?

\begin{answer}[5em]
All computer programs, regardless of language, consist of \emph{code} and \emph{data}.
Methods contain code (statements or instructions), whereas variables contain data (references or values).
\end{answer}


\Q In your own words, describe what methods are for. Why not just write everything in \java{main}?

\begin{answer}[5em]
Methods help organize the code into separate parts.
They also make it possible to write code once and use it multiple times.
\end{answer}
