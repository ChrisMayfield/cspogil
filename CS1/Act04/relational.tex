% based on Model 2 of Activity 4 Boolean by Helen Hu

\model{Relational Operators}

When you omit the semicolon in the Interactions pane, DrJava will display the \emph{value} of the expression you have entered.
In the table below, predict what values DrJava will display and identify the relational operator (the first four rows are completed for you).

%TODO try to rework w/o DrJava; Interactions w/o println confuses students

\begin{center}
\begin{tabular}{|L{200pt}|C{110pt}|C{110pt}|}
\hline
\tr Interactions & \tr Value displayed & \tr Relational operator \\
\hline
\java{int three = 3}                   & none        & none            \\
\hline
\java{int four = 4}                    & none        & none            \\
\hline
\java{System.out.println(four)}        & 4           & none            \\
\hline
\java{three > four}                    & false       & \java{>}        \\
\hline
\java{boolean isLarger = three > four} & \ans{none}  & \ans{\java{>}}  \\
\hline
\java{System.out.println(isLarger)}    & \ans{false} & \ans{none}      \\
\hline
\java{three == four}                   & \ans{false} & \ans{\java{==}} \\
\hline
\java{three < four}                    & \ans{true}  & \ans{\java{<}}  \\
\hline
\java{three <= four}                   & \ans{true}  & \ans{\java{<=}} \\
\hline
\java{three = four}                    & \ans{4}     & \ans{none}      \\
\hline
\java{three == four}                   & \ans{true}  & \ans{\java{==}} \\
\hline
\end{tabular}
\end{center}


\quest{10 min}

\Q List the four unique \emph{boolean expressions} used in \ref{\currfilename}.

\begin{answer}[3em]
\begin{javaans}
    three > four    three == four    three < four    three <= four
\end{javaans}
\end{answer}


\Q Examine the fifth line of Java code in the above model.

\begin{enumerate}

\item What three actions are performed in this single line of code?
\ans{It declares the variable \java{isLarger}, compares the values of \java{three} and \java{four}, and assigns the result to \java{isLarger}.}

\item Write two lines of code, ending with semicolons, that would perform these same actions (but in two lines instead of a single line).

\ans{\tt boolean isLarger;} \\
\ans{\tt isLarger = three > four;}

\end{enumerate}


\Q The \java{!=} operator means ``not equals''.
Give an example of a boolean expression that uses \java{!=} and evaluates to false.

\begin{answer}[3em]
\java{5 != 5} is false (because they \emph{are} equal)
\end{answer}


\Q Explain why the same boolean expression \java{three == four} resulted with two different boolean values in the table.

\begin{answer}
The line \java{three = four} assigned the value of four to three, making the two variables equal.
They started out not being equal, but they ended up with the same value.
\end{answer}


\Q What is the difference between \java{=} and \java{==} in Java?

\begin{answer}
The \java{=} operator assigns a value to a variable, and the \java{==} operator compares two values.
\end{answer}


\Q List the six relational operators that can be used in a boolean expression.
(Five have been used so far, but you should be able to guess the sixth.)
Explain briefly what each one means.

\begin{answer}
\begin{tabular}{lll}
\java{<} is less than              & \java{>} is greater than              & \java{==} is equal to     \\
\java{<=} is less than or equal to & \java{>=} is greater than or equal to & \java{!=} is not equal to \\
\end{tabular}
\end{answer}
