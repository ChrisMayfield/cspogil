% comment out for student version
\ifdefined\Student\relax\else\def\Teacher{}\fi

\documentclass[12pt]{article}

\title{Activity 5: Boolean Logic}
\author{Chris Mayfield and Helen Hu}
\date{Spring 2018}

%\ProvidesPackage{cspogil}

% fonts
\usepackage[utf8]{inputenc}
\usepackage[T1]{fontenc}
\usepackage{mathpazo}

% spacing
\usepackage[margin=2cm]{geometry}
\renewcommand{\arraystretch}{1.4}
\setlength{\parindent}{0pt}

% orphans and widows
\clubpenalty=10000
\widowpenalty=10000
\pagestyle{empty}

% figures and tables
\usepackage{graphicx}
\usepackage{multicol}
\usepackage{tabularx}

% fixed-width columns
\usepackage{array}
\newcolumntype{L}[1]{>{\raggedright\let\newline\\\arraybackslash\hspace{0pt}}m{#1}}
\newcolumntype{C}[1]{>{\centering\let\newline\\\arraybackslash\hspace{0pt}}m{#1}}
\newcolumntype{R}[1]{>{\raggedleft\let\newline\\\arraybackslash\hspace{0pt}}m{#1}}

% include paths
\makeatletter
\def\input@path{{Models/}{../../Models/}}
\graphicspath{{Models/}{../../Models/}}
\makeatother

% colors
\usepackage[svgnames,table]{xcolor}
\definecolor{bgcolor}{HTML}{FAFAFA}
\definecolor{comment}{HTML}{007C00}
\definecolor{keyword}{HTML}{0000FF}
\definecolor{strings}{HTML}{B20000}

% table headers
\newcommand{\tr}{\bf\cellcolor{Yellow!10}}

% syntax highlighting
\usepackage{textcomp}
\usepackage{listings}
\lstset{
    basicstyle=\ttfamily,
    backgroundcolor=\color{bgcolor},
    numberstyle=\scriptsize\color{comment},
    commentstyle=\color{comment},
    keywordstyle=\color{keyword},
    stringstyle=\color{strings},
    columns=fullflexible,
    keepspaces=true,
    showstringspaces=false,
    upquote=true
}

% code environments
\newcommand{\java}[1]{\lstinline[language=java]{#1}}%[
\lstnewenvironment{javalst}{\lstset{language=java,backgroundcolor=}}{}
\lstnewenvironment{javabox}{\lstset{language=java,frame=single,numbers=left}\quote}{\endquote}

% PDF properties
\usepackage[pdftex]{hyperref}
\urlstyle{same}
\makeatletter
\hypersetup{
  pdftitle={\@title},
  pdfauthor={\@author},
  pdfsubject={\@date},
  pdfkeywords={},
  bookmarksopen=false,
  colorlinks=true,
  citecolor=black,
  filecolor=black,
  linkcolor=black,
  urlcolor=blue
}
\makeatother

% titles
\makeatletter
\renewcommand{\maketitle}{\begin{center}\LARGE\@title\end{center}}
\makeatother

% boxes
\newcommand{\emptybox}[1][10em]{
\vspace{1em}
\begin{tabularx}{\linewidth}{|X|}
\hline\\[#1]\hline
\end{tabularx}}

% models
\newcommand{\model}[1]{\section{#1}\nopagebreak}
\renewcommand{\thesection}{Model~\arabic{section}}

% questions
\newcommand{\quest}[1]{\subsection*{Questions~ (#1)}}
\newcounter{question}
\newcommand{\Q}{\vspace{1em}\refstepcounter{question}\arabic{question}.~ }
\renewcommand{\thequestion}{\#\arabic{question}}

% sub-question lists
\usepackage{enumitem}
\setenumerate[1]{label=\alph*)}
\setlist{itemsep=1em,after=\vspace{1ex}}

% inline answers
\definecolor{answers}{HTML}{C0C0C0}
\newcommand{\ans}[1]{%
\ifdefined\Student
    \phantom{~~\textcolor{answers}{#1}}
\else
    ~~\textcolor{answers}{#1}
\fi}

% longer answers [optional height]
\newsavebox{\ansbox}
\newenvironment{answer}[1][4em]{
\nopagebreak
\begin{lrbox}{\ansbox}
\begin{minipage}[t][#1]{\linewidth}
\color{answers}
}{
\end{minipage}
\end{lrbox}
\ifdefined\Student
    \phantom{\usebox{\ansbox}}%
\else
    \usebox{\ansbox}%
\fi}


\begin{document}

\maketitle

The primitive data type \java{boolean} has two values: \java{true} and \java{false}.
Boolean expressions are built using \emph{relational operators} and \emph{conditional operators}.

\guide{
  \item Recognize the value of developing process skills.
  \item Evaluate boolean expressions with relational operators (<, >, <=, >=, ==, !=).
  \item Explain the difference between assignment (=) and equality (==) operators.
  \item Evaluate boolean expressions that involve comparisons with \&\&, ||, and !.
}{
  \item Evaluating complex logic expressions based on operator precedence. (Critical Thinking)
}{
\ref{employers.tex} is ultimately about process skills and should help with student buy-in.
If you are using the \github{Handouts/role-cards-mayfield.pdf}{Role Cards}, have students look at the definitions on the reverse side.
Each activity targets specific ``process skill goals'' from these categories.

\ref{relational.tex} mentions DrJava, but it can be replaced with another IDE or \href{http://www.javarepl.com/}{Java REPL}.
Give students about three minutes to fill in the table without using a computer.
Then show them the actual results interactively (or in an example program) on the projector.

When reporting out, ask students to explain what \emph{expressions} are and how they differ from \emph{statements}.
Reinforce what it means to \emph{evaluate} an expression (i.e., compute a single value) versus \emph{execute} a statement (i.e., run an entire line of code).

During \ref{conditional.tex}, explain that the variables $p$ and $q$ are often used to represent logic values in discrete math.
Make sure students understand that ! is a \emph{unary} operator, and that \java{&&} and \java{||} are \emph{binary} operators.
}

\section*{Meta Activity: What Employers Want}

The following data is from the \textit{Job Outlook 2019} survey by the National Association of Colleges and Employers (NACE). A total of 172 organizations responded to the survey.

\begin{table}[h!]
\centering

{\bf Attributes Employers Seek on a Candidate's Resume}
\vspace{1ex}

\renewcommand{\arraystretch}{1.2}
\begin{tabular}{|l|c|}
\hline
\tr Attribute                  & \tr \% of respondents \\
\hline
Ability to work in a team      & 78.7\% \\
\hline
Analytical/quantitative skills & 71.9\% \\
\hline
Communication skills (verbal)  & 67.4\% \\
\hline
Communication skills (written) & 82.0\% \\
\hline
Detail-oriented                & 59.6\% \\
\hline
Initiative                     & 74.2\% \\
\hline
Leadership                     & 67.4\% \\
\hline
Problem-solving skills         & 80.9\% \\
\hline
Strong work ethic              & 70.8\% \\
\hline
Technical skills               & 59.6\% \\
\hline
\end{tabular}

\vspace{1ex}
{\footnotesize Source:~~\url{https://www.naceweb.org/talent-acquisition/candidate-selection/}}
\end{table}
\vspace{-1em}


\quest{10 min}


\Q What are the top three attributes that employers look for on a resume?

\begin{itemize}[itemsep=1ex]
\item \#1: \ans{Communication skills (written)}
\item \#2: \ans{Problem-solving skills}
\item \#3: \ans{Ability to work in a team}
\end{itemize}

\vspace{-1ex}


\Q Describe the process your team used to answer to the previous question.

\begin{answer}[3em]
We searched the table for the highest three percentages and then made sure that we all had the same answers.
(Note that searching/sorting involves comparison, which is the next model.)
\end{answer}


\Q How is communication (written and verbal) related to problem solving and teamwork?

\begin{answer}[3em]
Solving problems in teams involves talking to other people and trying different approaches.
Writing solutions down is necessary to solidify the details and share them with others.
\end{answer}


\Q How does the team-based learning approach in this class help you develop these skills?

\begin{answer}[3em]
POGIL activities provide an opportunity to develop these skills during class.
Ideally, students will learn these skills both in the classroom and in other activities like clubs and internships.
\end{answer}

\model{Relational Operators}
% based on Model 2 of Activity 4 Boolean by Helen Hu

In the meta activity, you determined the top three attributes by comparing percentages.
We can declare variables to represent these percentages in Java:

\begin{javalst}
    double written = 82.0;   // Communication skills (written)
    double problem = 80.9;   // Problem-solving skills
    double teamwork = 78.7;  // Ability to work in a team
\end{javalst}

In the table below, determine the result of each expression and identify the operator.
The first five rows are completed for you.
(Optional: Use {\it JShell} to check your work.)

\begin{center}
\setlength{\defaultwidth}{8.5em}
\begin{tabular}{|L{12em}|C{10em}|C{10em}|}
\hline
\tr Expression              & \tr Result    & \tr Operator \\ \hline
\java{written}              &       82.0    &       none   \\ \hline
\java{written > problem}    &       true    &       >      \\ \hline
\java{problem < teamwork}   &       false   &       <      \\ \hline
\java{teamwork = 79.5}      &       79.5    &       =      \\ \hline
\java{teamwork == 78.7}     &       false   &       ==     \\ \hline
\java{82.0 < written}       &  \ans{false}  &  \ans{<}     \\ \hline
\java{82.0 > written}       &  \ans{false}  &  \ans{>}     \\ \hline
\java{82.0 == written}      &  \ans{true}   &  \ans{==}    \\ \hline
\java{problem == written}   &  \ans{false}  &  \ans{==}    \\ \hline
\java{teamwork == problem}  &  \ans{false}  &  \ans{==}    \\ \hline
\java{teamwork = problem}   &  \ans{80.9}   &  \ans{=}     \\ \hline
\java{teamwork == problem}  &  \ans{true}   &  \ans{==}    \\ \hline
%\java{problem}              &  \ans{80.9}   &  \ans{none}  \\ \hline
\java{teamwork}             &  \ans{80.9}   &  \ans{none}  \\ \hline
\end{tabular}
\setlength{\defaultwidth}{15em}
\end{center}


\quest{15 min}

\Q A \emph{relational operator} compares two values; the result is either \java{true} or \java{false}.
Identify the three relational operators used in the table above.

\begin{answer}[2em]
\begin{javaans}
    >    <    ==
\end{javaans}
\end{answer}


\Q Explain why the same expression \java{teamwork == problem} resulted with two different values in the table.

\begin{answer}
The line \java{teamwork = problem} assigned the value of \java{problem} to \java{teamwork}, making the two variables equal.
They started out not being equal, but they ended up with the same value.
\end{answer}


\Q \label{key1}
What is the difference between \java{=} and \java{==} in Java?

\begin{answer}
The \java{=} operator assigns a value to a variable, and the \java{==} operator compares two values.
\end{answer}


\Q The \java{!=} relational operator means ``not equals''.
Give an example of a boolean expression that uses \java{!=} and evaluates to false.

\begin{answer}[2em]
\java{5 != 5} is false (because they {\it are} equal)
\end{answer}


\Q The \java{>=} relational operator means ``greater than or equal to''.
Give an example of a boolean expression that uses \java{>=} and evaluates to true.

\begin{answer}[2em]
\java{5 >= 5} is true (because they {\it are} equal)
\end{answer}


\Q Java has six relational operators.
Only five have been shown, but you should be able to guess the sixth.
List all six below, and explain briefly what each one means.

\begin{answer}[3em]
\vspace{-3pt}
\begin{tabular}{lll}
\java{<} is less than              & \java{>} is greater than              & \java{==} is equal to     \\
\java{<=} is less than or equal to & \java{>=} is greater than or equal to & \java{!=} is not equal to \\
\end{tabular}
\end{answer}

% based on Model 3 of Activity 4 Boolean by Helen Hu

\model{Conditional Operators}

Boolean expressions may also use conditional operators to implement basic logic.
Relational operators are always executed first, so there is generally no need for parentheses.

\begin{center}
\begin{tabular}{|C{100pt}|C{100pt}|}
\hline
\tr Operator & \tr Meaning \\
\hline
\java{!}  & not \\
\hline
\java{&&} & and \\
\hline
\java{||} & or \\
\hline
\end{tabular}
\end{center}

If all three operators appear in the same expression, Java will evaluate the \java{!} first, then \java{&&}, and finally \java{||}.
If there are multiples of the same operator, they are evaluated from left to right.

\smallskip
\begin{multicols}{2}
\centering

\textbf{Example Variables:} \\[1ex]
\java{int a = 3;} \\
\java{int b = 4;} \\
\java{int c = 5;} \\
\java{boolean funny = true;} \\
\java{boolean weird = false;} \\

\columnbreak

\textbf{Example Expressions:} \\[1ex]
\java{a < b && funny} \\
\java{a < b && b < c} \\
\java{c < a || b < a} \\
\java{funny && a < c} \\
\java{!funny || weird} \\

\end{multicols}


\quest{15 min}


\Q What are the values (true or false) of the example expressions?
\ans{true, true, false, true, false}

\vspace{1em}


\Q Give different examples of boolean expressions that:

\begin{enumerate}
\item uses a, b, and !, and evaluates to false \ans{!(a < b)}
\item uses b, c, and !, and evaluates to true \ans{!(b > c)}
\item uses any variables, but evaluates to false \ans{weird}
\item uses any variables, but evaluates to true \ans{funny}
\end{enumerate}


\Q Using your answers from the previous question, write the boolean expression \java{p && q} where \java{p} is your answer to step a) and \java{q} is your answer to step b).

\begin{enumerate}
\item Your expression: \ans{\java{!(a < b) && !(b > c)}}
\item Result of ~\java{p && q}: \ans{false (no matter what)}
\end{enumerate}


\Q \label{truthtable} Complete the following table:

\begin{center}
\begin{tabular}{|C{50pt}|C{50pt}|C{50pt}|C{50pt}|C{50pt}|}
\hline
\tr \java{p} & \tr \java{q} & \tr \java{p && q} & \tr \java{p || q} & \tr \java{!p} \\
\hline
false & false & \ans{false} & \ans{false} & \ans{true}  \\
\hline
false & true  & \ans{false} & \ans{true}  & \ans{true}  \\
\hline
true  & false & \ans{false} & \ans{true}  & \ans{false} \\
\hline
true  & true  & \ans{true}  & \ans{true}  & \ans{false} \\
\hline
\end{tabular}
\end{center}


\Q Using the values in \ref{\currfilename}, give the result of each operator in the following expression.
In other words, show your work as you evaluate the code in the same order that Java would.

\begin{center}
\java{!(a > c) && b > c}
\vspace{1em}

\begin{tabular}{|C{50pt}|C{70pt}|C{200pt}|C{70pt}|}
\hline
\tr & \tr Operator & \tr Expression & \tr Result \\
\hline
1st & \java{>}  & \java{a > c} & false \\
\hline
2nd & \ans{\java{!}}  & \ans{\java{!}false} & \ans{true} \\
\hline
3rd & \ans{\java{>}}  & \ans{\java{b > c}}  & \ans{false} \\
\hline
4th & \ans{\java{&&}} & \ans{true \java{&&} false} & \ans{false} \\
\hline
\end{tabular}
\end{center}


\Q Add parentheses to the boolean expression from the previous question so that the \java{&&} is evaluated before the \java{!}. Then remove any unnecessary parentheses.

\begin{enumerate}
\item Expression: \ans{\java{!(a > c && b > c)}}
\item New result: \ans{true}
\end{enumerate}


\Q Review the table from \ref{truthtable} for evaluating \java{&&} and \java{||}.
Looking only at the \java{p} and \java{&&} columns, when is it necessary to examine \java{q} to determine how \java{p && q} should be evaluated?

\begin{answer}
You only need to look at \java{q} when \java{p} is true.
If \java{p} is false, you know the expression will be false.
\end{answer}


\Q Review the table from \ref{truthtable} for evaluating \java{&&} and \java{||}.
Looking only at the \java{p} and \java{||} columns, when is it necessary to examine \java{q} to determine how \java{p || q} should be evaluated?

\begin{answer}
You only need to look at \java{q} when \java{p} is false.
If \java{p} is true, you know the expression will be true.
\end{answer}


\Q In Java, \java{&&} and \java{||} are \emph{short circuit} operators, meaning they evaluate only what is necessary.
If the expression \java{p} is more likely to be true than the expression \java{q}, which one should you place on the left of each operator to avoid doing extra work?

\begin{enumerate}
\item left of the \java{&&} expression: \ans{\java{q} --- if it's false, then \java{p} won't be evaluated}
\item left of the \java{||} expression: \ans{\java{p} --- if it's true, then \java{q} won't be evaluated}
\end{enumerate}


\end{document}
