\model{Conditional Operators}
% based on Model 3 of Activity 4 Boolean by Helen Hu

Boolean expressions, like \java{written > problem} and \java{teamwork < 75.0}, can be combined using the \emph{conditional operators}:

\begin{center}
\begin{tabular}{|C{100pt}|C{100pt}|}
\hline
\tr Operator & \tr Meaning \\
\hline
\java{!}  & not \\
\hline
\java{&&} & and \\
\hline
\java{||} & or \\
\hline
\end{tabular}
\end{center}

For example, \java{written > problem && teamwork < 75.0} is false, because \java{teamwork} is not less than \java{75.0}.
(Both conditions need to be true in order for \java{&&} to be true.)

\bigskip

The following table summarizes the result of \java{&&}, \java{||}, and \java{!} for all possible inputs.
The variables \java{p} and \java{q} represent conditions like \java{written > problem} and \java{teamwork < 75.0}.

\begin{center}
\begin{tabular}{|C{50pt}|C{50pt}||C{50pt}|C{50pt}|C{50pt}|}
\hline
\tr \java{p} & \tr \java{q} & \tr \java{p && q} & \tr \java{p || q} & \tr \java{!p} \\
\hline
false & false & false & false & true  \\ \hline  % \multirow{2}{*}{true}  \\ \cline{1-4}
false & true  & false & true  & true  \\ \hline  %                        \\ \hline
true  & false & false & true  & false \\ \hline  % \multirow{2}{*}{false} \\ \cline{1-4}
true  & true  & true  & true  & false \\ \hline  %                        \\ \hline
\end{tabular}
\end{center}


\quest{20 min}


\Q Consider the following variables:

\begin{quote}
\begin{javalst}
double initiative = 74.2;
double analytical = 71.9;
double workEthic  = 70.8;
boolean hired = true;
boolean fired = false;
\end{javalst}
\end{quote}

What are the results (true or false) of the following expressions?

\begin{quote}
\begin{tabular}{|l|l|}
\hline
\tr Expression & \tr Result \\
\hline
\java{!fired} & \ans[3em]{true} \\
\hline
\java{!(workEthic < initiative)} & \ans[3em]{false} \\
\hline
\java{workEthic < 71.0 && 71.0 < initiative} & \ans[3em]{true} \\
\hline
\java{initiative < 70.0 || workEthic > 70.0} & \ans[3em]{true} \\
\hline
\java{fired || workEthic < 50.0} & \ans[3em]{false} \\
\hline
\java{analytical < initiative && fired} & \ans[3em]{false} \\
\hline
\java{hired && !fired} & \ans[3em]{true} \\
\hline
\end{tabular}
\end{quote}


\Q Write a boolean expression that \ldots

\begin{enumerate}
\item uses \java{initiative}, \java{analytical}, and \java{!}, and evaluates to false. \ans[12em]{!(analytical < initiative)}
\item uses \java{analytical}, \java{workEthic}, and \java{!}, and evaluates to true. \ans[12em]{!(workEthic > analytical)}
\item uses any variable(s), and evaluates to false. \ans[12em]{fired}
\item uses any variable(s), and evaluates to true. \ans[12em]{hired}
\end{enumerate}


\Q Using your answers to the previous question, write a boolean expression ``\java{p && q}'' where \java{p} is your answer to part a) and \java{q} is your answer to part b).

\begin{enumerate}
\item Your expression: \ans[30em]{\java{!(analytical < initiative) && !(workEthic > analytical)}}
\item Result of ~\java{p && q}: \ans{false (no matter what)}
\end{enumerate}


\comment{
Relational operators (\java{<}, \java{>}, and \java{==}) are evaluated before conditional operators (\java{!}, \java{&&}, and \java{||}).
When multiple conditional operators are used, Java evaluates \java{!} first, then \java{&&}, and finally \java{||}.
}


%\Q \label{truthtable} Complete the following table, which explores all possible values for \java{p} and \java{q}:
%
%\begin{center}
%\begin{tabular}{|C{50pt}|C{50pt}|C{50pt}|C{50pt}|C{50pt}|}
%\hline
%\tr \java{p} & \tr \java{q} & \tr \java{p && q} & \tr \java{p || q} & \tr \java{!p} \\
%\hline
%false & false & \ans[3em]{false} & \ans[3em]{false} & \ans[3em]{true}  \\
%\hline
%false & true  & \ans[3em]{false} & \ans[3em]{true}  & \ans[3em]{true}  \\
%\hline
%true  & false & \ans[3em]{false} & \ans[3em]{true}  & \ans[3em]{false} \\
%\hline
%true  & true  & \ans[3em]{true}  & \ans[3em]{true}  & \ans[3em]{false} \\
%\hline
%\end{tabular}
%\end{center}


\Q Show the intermediate result of each operator below.
In other words, show your work as you evaluate the code in the same order that Java would.

\begin{center}
\java{!(initiative < analytical) && workEthic > analytical}
\vspace{1em}

\begin{tabular}{|C{50pt}|C{70pt}|C{200pt}|C{70pt}|}
\hline
\tr & \tr Operator & \tr Expression & \tr Result \\
\hline
1st & \java{<}  & \java{initiative < analytical} & false \\
\hline
2nd & \ans[3em]{\java{!}}  & \ans{\java{!}false} & \ans[3em]{true} \\
\hline
3rd & \ans[3em]{\java{>}}  & \ans{\java{workEthic > analytical}} & \ans[3em]{false} \\
\hline
4th & \ans[3em]{\java{&&}} & \ans{true \java{&&} false} & \ans[3em]{false} \\
\hline
\end{tabular}
\end{center}


\Q Change the parentheses in the original expression (from the previous question) so that the \java{&&} is evaluated before the \java{!}.
Then remove any unnecessary parentheses.

\begin{enumerate}
\item Expression: \ans[30em]{\java{!(initiative < analytical && workEthic > analytical)}}
\item New result: \ans[5em]{true}
\end{enumerate}


\Q Review the table from \ref{\currfilename} for evaluating \java{&&} and \java{||}.
Looking only at the \java{p} and \java{&&} columns, when is it necessary to examine \java{q} to determine how \java{p && q} should be evaluated?

\begin{answer}
You only need to look at \java{q} when \java{p} is true.
If \java{p} is false, you know the expression will be false.
\end{answer}


\Q Review the table from \ref{\currfilename} for evaluating \java{&&} and \java{||}.
Looking only at the \java{p} and \java{||} columns, when is it necessary to examine \java{q} to determine how \java{p || q} should be evaluated?

\begin{answer}
You only need to look at \java{q} when \java{p} is false.
If \java{p} is true, you know the expression will be true.
\end{answer}


\Q In Java, \java{&&} and \java{||} are \emph{short circuit} operators, meaning they evaluate only what is necessary.
If the expression \java{p} is more likely to be true than the expression \java{q}, which one should you place on the left of each operator to avoid doing extra work?

\begin{enumerate}
\item left of the \java{&&} expression:
\ans[25em]{\java{q} --- if it's false, then \java{p} won't be evaluated}
\item left of the \java{||} expression:
\ans[25em]{\java{p} --- if it's true, then \java{q} won't be evaluated}
\end{enumerate}


\Q What is the result of the following expressions?
\begin{enumerate}
\item \java{1 + 0 > 0 && 1 / 0 > 0}
\ans[24em]{java.lang.ArithmeticException: / by zero}
\item \java{1 + 0 > 0 || 1 / 0 > 0}
\ans[24em]{true}
\end{enumerate}
