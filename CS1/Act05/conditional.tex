\model{Conditional Operators}
% based on Model 3 of Activity 4 Boolean by Helen Hu

Boolean expressions, like \java{written > problem} and \java{teamwork < 75.0}, can be combined using the \emph{conditional operators}:

\begin{center}
\begin{tabular}{|C{100pt}|C{100pt}|}
\hline
\tr Operator & \tr Meaning \\
\hline
\java{!}  & not \\
\hline
\java{&&} & and \\
\hline
\java{||} & or \\
\hline
\end{tabular}
\end{center}

If all three operators appear in the same expression, Java will evaluate \java{!} first, then \java{&&}, and finally \java{||}.
If there are multiples of the same operator, they are evaluated from left to right.
Relational operators are evaluated before ~\java{&&} and \java{||}, so there is generally no need for parentheses.


\smallskip
\begin{multicols}{2}
\centering

\textbf{Example Variables:} \\[1ex]
\java{double initiative = 74.2;} \\
\java{double analytical = 71.9;} \\
\java{double workEthic  = 70.8;} \\
\java{boolean hired = true;} \\
\java{boolean fired = false;} \\

\columnbreak

\textbf{Example Expressions:} \\[1ex]
\java{analytical < initiative && fired} \\
\java{workEthic < 71.0 && 71.0 < initiative} \\
\java{initiative < 70.0 || workEthic > 70.0} \\
\java{fired || workEthic < 50.0} \\
\java{hired && !fired} \\

\end{multicols}


\quest{20 min}


\Q What are the values (true or false) of the example expressions?

\vspace{1ex}
\hspace{15pt}
\ans[5em]{false} ~~
\ans[5em]{true} ~~
\ans[5em]{true} ~~
\ans[5em]{false} ~~
\ans[5em]{true}


\Q Give different examples of boolean expressions that:

\begin{enumerate}
\item uses \java{initiative}, \java{analytical}, and \java{!}, and evaluates to false \ans[12em]{!(initiative < analytical)}
\item uses \java{analytical}, \java{workEthic}, and \java{!}, and evaluates to true \ans[12em]{!(analytical > workEthic)}
\item uses any variables, and evaluates to false \ans[12em]{fired}
\item uses any variables, and evaluates to true \ans[12em]{hired}
\end{enumerate}


\Q Using your answers from the previous question, write a boolean expression ``\java{p && q}'' where \java{p} is your answer to step a) and \java{q} is your answer to step b).

\begin{enumerate}
\item Your expression: \ans[30em]{\java{!(initiative < analytical) && !(analytical > workEthic)}}
\item Result of ~\java{p && q}: \ans{false (no matter what)}
\end{enumerate}


\Q \label{truthtable} Complete the following table, which explores all possible values for \java{p} and \java{q}:

\begin{center}
\begin{tabular}{|C{50pt}|C{50pt}|C{50pt}|C{50pt}|C{50pt}|}
\hline
\tr \java{p} & \tr \java{q} & \tr \java{p && q} & \tr \java{p || q} & \tr \java{!p} \\
\hline
false & false & \ans[3em]{false} & \ans[3em]{false} & \ans[3em]{true}  \\
\hline
false & true  & \ans[3em]{false} & \ans[3em]{true}  & \ans[3em]{true}  \\
\hline
true  & false & \ans[3em]{false} & \ans[3em]{true}  & \ans[3em]{false} \\
\hline
true  & true  & \ans[3em]{true}  & \ans[3em]{true}  & \ans[3em]{false} \\
\hline
\end{tabular}
\end{center}


\Q Using the values in \ref{relational.tex}, give the result of each operator below.
In other words, show your work as you evaluate the code in the same order that Java would.

\begin{center}
\java{!(written < teamwork) && problem > teamwork}
\vspace{1em}

\begin{tabular}{|C{50pt}|C{70pt}|C{200pt}|C{70pt}|}
\hline
\tr & \tr Operator & \tr Expression & \tr Result \\
\hline
1st & \java{<}  & \java{written < teamwork} & false \\
\hline
2nd & \ans[3em]{\java{!}}  & \ans{\java{!}false} & \ans[3em]{true} \\
\hline
3rd & \ans[3em]{\java{>}}  & \ans{\java{problem > teamwork}}  & \ans[3em]{true} \\
\hline
4th & \ans[3em]{\java{&&}} & \ans{true \java{&&} true} & \ans[3em]{true} \\
\hline
\end{tabular}
\end{center}


\Q Add parentheses to the boolean expression from the previous question so that the \java{&&} is evaluated before the \java{!}. Then remove any unnecessary parentheses.

\begin{enumerate}
\item Expression: \ans[25em]{\java{!(written < teamwork && problem > teamwork)}}
\item New result: \ans[5em]{true}
\end{enumerate}


\Q Review the table from \ref{truthtable} for evaluating \java{&&} and \java{||}.
Looking only at the \java{p} and \java{&&} columns, when is it necessary to examine \java{q} to determine how \java{p && q} should be evaluated?

\begin{answer}
You only need to look at \java{q} when \java{p} is true.
If \java{p} is false, you know the expression will be false.
\end{answer}


\Q Review the table from \ref{truthtable} for evaluating \java{&&} and \java{||}.
Looking only at the \java{p} and \java{||} columns, when is it necessary to examine \java{q} to determine how \java{p || q} should be evaluated?

\begin{answer}
You only need to look at \java{q} when \java{p} is false.
If \java{p} is true, you know the expression will be true.
\end{answer}


\Q In Java, \java{&&} and \java{||} are \emph{short circuit} operators, meaning they evaluate only what is necessary.
If the expression \java{p} is more likely to be true than the expression \java{q}, which one should you place on the left of each operator to avoid doing extra work?

\begin{enumerate}
\item left of the \java{&&} expression:
\ans[25em]{\java{q} --- if it's false, then \java{p} won't be evaluated}
\item left of the \java{||} expression:
\ans[25em]{\java{p} --- if it's true, then \java{q} won't be evaluated}
\end{enumerate}


\Q What is the result of the following expressions?
\begin{enumerate}
\item \java{1 + 0 > 0 && 1 / 0 > 0}
\ans[24em]{java.lang.ArithmeticException: / by zero}
\item \java{1 + 0 > 0 || 1 / 0 > 0}
\ans[24em]{true}
\end{enumerate}
