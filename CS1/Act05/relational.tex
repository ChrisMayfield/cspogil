\model{Relational Operators}
% based on Model 2 of Activity 4 Boolean by Helen Hu

In \ref{employers.tex}, you determined the top three attributes by comparing percentages.
We can declare variables to represent these percentages in Java:

\begin{javalst}
    double written = 82.0;   // Communication skills (written)
    double problem = 80.9;   // Problem-solving skills
    double teamwork = 78.7;  // Ability to work in a team
\end{javalst}

In the table below, predict the result of each expression and identify the operator (if any).
The first three rows are completed for you.

\begin{center}
\setlength{\defaultwidth}{8.5em}
\begin{tabular}{|L{12em}|C{10em}|C{10em}|}
\hline
\tr Expression              & \tr Result    & \tr Operator \\ \hline
\java{written}              &       82.0    &       none   \\ \hline
\java{written > problem}    &       true    &       >      \\ \hline
\java{problem < teamwork}   &       false   &       <      \\ \hline
\java{82.0 < written}       &  \ans{false}  &  \ans{<}     \\ \hline
\java{82.0 > written}       &  \ans{false}  &  \ans{>}     \\ \hline
\java{82.0 == written}      &  \ans{true}   &  \ans{==}    \\ \hline
\java{problem == written}   &  \ans{false}  &  \ans{==}    \\ \hline
\java{teamwork == problem}  &  \ans{false}  &  \ans{==}    \\ \hline
\java{teamwork = problem}   &  \ans{80.9}   &  \ans{=}     \\ \hline
\java{teamwork == problem}  &  \ans{true}   &  \ans{==}    \\ \hline
\java{problem}              &  \ans{80.9}   &  \ans{none}  \\ \hline
\java{teamwork}             &  \ans{80.9}   &  \ans{none}  \\ \hline
\end{tabular}
\setlength{\defaultwidth}{15em}
\end{center}


\quest{15 min}

\Q A \emph{relational operator} compares two values; the result is either \java{true} or \java{false}.
Identify the three relational operators used in the table above.

\begin{answer}[2em]
\begin{javaans}
    >    <    ==
\end{javaans}
\end{answer}


\Q Explain why the same expression \java{teamwork == problem} resulted with two different values in the table.

\begin{answer}
The line \java{teamwork = problem} assigned the value of \java{problem} to \java{teamwork}, making the two variables equal.
They started out not being equal, but they ended up with the same value.
\end{answer}


\Q What is the difference between \java{=} and \java{==} in Java?

\begin{answer}
The \java{=} operator assigns a value to a variable, and the \java{==} operator compares two values.
\end{answer}


\Q The \java{!=} relational operator means ``not equals''.
Give an example of a boolean expression that uses \java{!=} and evaluates to false.

\begin{answer}[2em]
\java{5 != 5} is false (because they {\it are} equal)
\end{answer}


\Q The \java{>=} relational operator means ``greater than or equal to''.
Give an example of a boolean expression that uses \java{>=} and evaluates to true.

\begin{answer}[2em]
\java{5 >= 5} is true (because they {\it are} equal)
\end{answer}


\Q Java has six relational operators.
Only five have been mentioned, but you should be able to guess the sixth.
List all six below, and explain briefly what each one means.

\begin{answer}
\begin{tabular}{lll}
\java{<} is less than              & \java{>} is greater than              & \java{==} is equal to     \\
\java{<=} is less than or equal to & \java{>=} is greater than or equal to & \java{!=} is not equal to \\
\end{tabular}
\end{answer}
