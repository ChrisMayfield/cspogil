\model{Assignment}
% based on Model 2 of Activity 08 - Introduction to Loops by Helen Hu

Consider the following Java statements. What is the resulting value of each variable?

\begin{center}
\vspace{-6pt}
\begin{tabular}{cp{120pt}cp{120pt}cp{120pt}}

\textsf{A:}
&
\vspace{-1em}
\begin{javalst}
int x, y;
x = 1;
y = 2;
y = x;
x = y;

\end{javalst}

&
\textsf{B:}
&
\vspace{-1em}
\begin{javalst}
int x, y, z;
x = 1;
y = 2;
z = y;
y = x;
x = z;
\end{javalst}

&
\textsf{C:}
&
\vspace{-1em}
\begin{javalst}
int z, y;
z = 2;
z = z + 1;
z = z + 1;
y = y + 1;

\end{javalst}

\\[-1em]
&
Value of x: \blank \ans{1}

\vspace{1em}
Value of y: \blank \ans{1}

&
&
Value of x: \blank \ans{2}

\vspace{1em}
Value of y: \blank \ans{1}

\vspace{1em}
Value of z: \blank \ans{2}

&
&
Value of z: \blank \ans{4}

\vspace{1em}
Value of y: \blank \ans{?}

\end{tabular}
\vspace{-14pt}
\end{center}


\quest{10 min}


\Q In program~\textsf{A}, why is the value of \java{x} not 2?

\begin{answer}
Each statement is executed one after the other, so the third assignment changes the value of \java{y} to 1.
The last assignment then assigns 1 to the value \java{x}.
\end{answer}


\Q In program~\textsf{B}, what happens to the values of \java{x} and \java{y}?

\begin{answer}
They get swapped; \java{x} was 1 and \java{y} was 2, but in the end \java{x} was 2 and \java{y} was 1.
\end{answer}


\Q In program~\textsf{B}, what is the purpose of the variable \java{z}?

\begin{answer}
It is a temporary variable that makes it possible to swap the values of \java{x} and \java{y}.
\end{answer}


\Q If program~\textsf{C} runs, what happens to the value of \java{z}?

\begin{answer}
It gets incremented twice; the value starts at 2, then it becomes 3, and then it becomes 4.
\end{answer}


\Q In program~\textsf{C}, why is it possible to increment \java{z} but not \java{y}?

\begin{answer}
The variable \java{z} was initialized, but \java{y} was not.
Java doesn't know what value to increment.
\end{answer}


\Q Because \emph{increment} and \emph{decrement} are so common in algorithms, Java provides the operators \java{++} and \java{--}.
For example, ~\java{x++} is the same as ~\java{x = x + 1}, and \java{y--} is the same as ~\java{y = y - 1}.
Write the value of \java{x} and \java{y} next to each statement below.

\vspace{-1ex}
\hspace{2em}
\begin{minipage}[t]{100pt}
\begin{javalst}
int x = 5;
x--;
x--;
\end{javalst}
\end{minipage}
\begin{minipage}[t]{100pt}
\begin{answer}
\begin{javaans}
x is 5
x is 4
x is 3
\end{javaans}
\end{answer}
\end{minipage}
\begin{minipage}[t]{100pt}
\begin{javalst}
int y = -10;
y++;
y++;
\end{javalst}
\end{minipage}
\begin{minipage}[t]{100pt}
\begin{answer}
\begin{javaans}
y is -10
y is -9
y is -8
\end{javaans}
\end{answer}
\end{minipage}
\vspace{2pt}


\Q \label{compound}
Like the assignment operator, the \java{++} and \java{--} operators replace the value of a variable.
Java also has \emph{compound assignment} operators for convenience.
For example, the statement ~\java{x = x + 2} can be rewritten as ~\java{x += 2}.
Simplify the following assignment statements.

\vspace{-1ex}
\hspace{2em}
\begin{minipage}[t]{150pt}
\begin{javalst}
step = step + 5;
size = size - 3;
total = total * 2;
change = change / 10;
hours = hours % 24;
\end{javalst}
\end{minipage}
\begin{minipage}[t]{150pt}
\begin{answer}[6em]
\begin{javaans}
step += 5;
size -= 3;
total *= 2;
change /= 10;
hours %= 24;
\end{javaans}
\end{answer}
\end{minipage}


\Q Which of the following assignment statements can also be rewritten like the ones in \ref{compound}?

\vspace{-1ex}
\hspace{2em}
\begin{minipage}[t]{150pt}
\begin{javalst}
step = 5 + step;
size = 3 - size;
total = 2 * total;
change = 10 / change;
hours = 24 % hours;
\end{javalst}
\end{minipage}
\begin{minipage}[t]{150pt}
\begin{answer}[6em]
\begin{javaans}
step += 5;
NO
total *= 2;
NO
NO
\end{javaans}
\end{answer}
\end{minipage}
