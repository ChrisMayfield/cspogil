% comment out for student version
\ifdefined\Student\relax\else\def\Teacher{}\fi

\documentclass[12pt]{article}

\title{Arrays of Numbers}
\author{Chris Mayfield and Helen Hu}
\date{Summer 2021}

%\ProvidesPackage{cspogil}

% fonts
\usepackage[utf8]{inputenc}
\usepackage[T1]{fontenc}
\usepackage{mathpazo}

% spacing
\usepackage[margin=2cm]{geometry}
\renewcommand{\arraystretch}{1.4}
\setlength{\parindent}{0pt}

% orphans and widows
\clubpenalty=10000
\widowpenalty=10000
\pagestyle{empty}

% figures and tables
\usepackage{graphicx}
\usepackage{multicol}
\usepackage{tabularx}

% fixed-width columns
\usepackage{array}
\newcolumntype{L}[1]{>{\raggedright\let\newline\\\arraybackslash\hspace{0pt}}m{#1}}
\newcolumntype{C}[1]{>{\centering\let\newline\\\arraybackslash\hspace{0pt}}m{#1}}
\newcolumntype{R}[1]{>{\raggedleft\let\newline\\\arraybackslash\hspace{0pt}}m{#1}}

% include paths
\makeatletter
\def\input@path{{Models/}{../../Models/}}
\graphicspath{{Models/}{../../Models/}}
\makeatother

% colors
\usepackage[svgnames,table]{xcolor}
\definecolor{bgcolor}{HTML}{FAFAFA}
\definecolor{comment}{HTML}{007C00}
\definecolor{keyword}{HTML}{0000FF}
\definecolor{strings}{HTML}{B20000}

% table headers
\newcommand{\tr}{\bf\cellcolor{Yellow!10}}

% syntax highlighting
\usepackage{textcomp}
\usepackage{listings}
\lstset{
    basicstyle=\ttfamily,
    backgroundcolor=\color{bgcolor},
    numberstyle=\scriptsize\color{comment},
    commentstyle=\color{comment},
    keywordstyle=\color{keyword},
    stringstyle=\color{strings},
    columns=fullflexible,
    keepspaces=true,
    showstringspaces=false,
    upquote=true
}

% code environments
\newcommand{\java}[1]{\lstinline[language=java]{#1}}%[
\lstnewenvironment{javalst}{\lstset{language=java,backgroundcolor=}}{}
\lstnewenvironment{javabox}{\lstset{language=java,frame=single,numbers=left}\quote}{\endquote}

% PDF properties
\usepackage[pdftex]{hyperref}
\urlstyle{same}
\makeatletter
\hypersetup{
  pdftitle={\@title},
  pdfauthor={\@author},
  pdfsubject={\@date},
  pdfkeywords={},
  bookmarksopen=false,
  colorlinks=true,
  citecolor=black,
  filecolor=black,
  linkcolor=black,
  urlcolor=blue
}
\makeatother

% titles
\makeatletter
\renewcommand{\maketitle}{\begin{center}\LARGE\@title\end{center}}
\makeatother

% boxes
\newcommand{\emptybox}[1][10em]{
\vspace{1em}
\begin{tabularx}{\linewidth}{|X|}
\hline\\[#1]\hline
\end{tabularx}}

% models
\newcommand{\model}[1]{\section{#1}\nopagebreak}
\renewcommand{\thesection}{Model~\arabic{section}}

% questions
\newcommand{\quest}[1]{\subsection*{Questions~ (#1)}}
\newcounter{question}
\newcommand{\Q}{\vspace{1em}\refstepcounter{question}\arabic{question}.~ }
\renewcommand{\thequestion}{\#\arabic{question}}

% sub-question lists
\usepackage{enumitem}
\setenumerate[1]{label=\alph*)}
\setlist{itemsep=1em,after=\vspace{1ex}}

% inline answers
\definecolor{answers}{HTML}{C0C0C0}
\newcommand{\ans}[1]{%
\ifdefined\Student
    \phantom{~~\textcolor{answers}{#1}}
\else
    ~~\textcolor{answers}{#1}
\fi}

% longer answers [optional height]
\newsavebox{\ansbox}
\newenvironment{answer}[1][4em]{
\nopagebreak
\begin{lrbox}{\ansbox}
\begin{minipage}[t][#1]{\linewidth}
\color{answers}
}{
\end{minipage}
\end{lrbox}
\ifdefined\Student
    \phantom{\usebox{\ansbox}}%
\else
    \usebox{\ansbox}%
\fi}


\begin{document}

\maketitle

Programs often need to store multiple values of the same type, such as a list of phone numbers, or the names of your top 20 favorite songs.
Rather than create a separate variable for each one, we can store them together using an array.

\rolenames

\guide{
  \item Explain course/school policies about academic honesty.
  \item Declare and initialize array variables of primitive types.
  \item Draw a memory diagram of an array of reference types.
}{
  \item Justifying answers based on evidence provided in the model. (Problem Solving)
}{
The case studies can be helpful to clarify policies and discuss what is appropriate collaboration vs cheating.
Each team will need a copy of your school's \github{Handouts/jmu-honor-code.pdf}{Honor Code} (or similar document).
Only 15 minutes are allocated for this discussion, but it could last longer if desired.

\ref{array-syntax.tex} introduces arrays for the first time, with a focus on array syntax.
Students learn about declaring, initializing, and using array variables.
Be sure to discuss \ref{arraysta} and \ref{arrayexp} when reporting out.
Explain how Java generally requires the \texttt{new} operator when creating arrays, except in the case of \ref{arraysta}.

In \ref{array-diagrams.tex}, explain that the \texttt{new} operator automatically zeros-out memory for the array.
Therefore, the default value will be 0 for integers, 0.0 for doubles, \texttt{null} for strings, etc.
These values should be present in the team's diagrams.
You may need to remind students that references (i.e., to string objects) are drawn with an arrow.

Key questions: \ref{arrayexp}, \ref{nocopy}, \ref{draw3str}
}

\model{Case Study: Panic Attack}

Frank was behind in his programming assignment.
He approached Martin to see if he could get some help.
But he was so far behind and so confused that Martin just gave him his code with the intent that he would ``just look at it to get some ideas.''

\vspace{1em}

In the paraphrased words of Frank: ``I started the assignment three days after you put it up.
But then other assignments came in and I started on them too.
I felt like I was chasing rabbits and began to panic.
It was already past the due date and I got really scared.
That's when I went to Martin to see if he could help.''
Frank copied much of the code and turned it in as his own.


\quest{8 min}


\Q Which, if any, of the students were at fault? Why?

\begin{answer}[5em]
\end{answer}


\Q Which specific Honor Code violations occurred?

\begin{answer}[5em]
\end{answer}


\Q What should Martin have done in this situation?

\begin{answer}[5em]
\end{answer}


\Q What options did Frank have besides cheating?

\begin{answer}[5em]
\end{answer}

\model{Case Study: Oops}

Emily was working in the lab on her programming assignment.
She finally finished the program and submitted it and went on to do some other work.
Shortly thereafter, she left the lab.

\vspace{1em}

Another student, Kyle, was working nearby.
He knew that she had successfully submitted the assignment, and he had not been able to get his to work properly.
When Emily left, he noticed that she had not logged out of her computer.
He moved to her workstation, found the work under her Documents directory, and copied it onto his memory stick.
He then logged out, logged in as himself, and copied the code onto his Desktop where he modified the program a bit, then successfully submitted it.


\quest{10 min}


\Q Which, if any, of the students were at fault? Why?

\begin{answer}
\end{answer}


\Q Which specific Honor Code violations occurred?

\begin{answer}
\end{answer}


\Q What should Emily have done in this situation?

\begin{answer}
\end{answer}


\Q What options did Kyle have besides cheating?

\begin{answer}
\end{answer}

\model{Array Syntax}
% based on Model 1 of Activity 13 - TwoD Arrays by Helen Hu

An \emph{array} variable allows you to store multiple variables (of the same type).
Each value in an array is known as an \emph{element}.
The programmer must specify the \emph{length} of the array (the number of array elements).
Once the array is created, its length cannot be changed.

\begin{quote}
\begin{javalst}
char[] letterArray = {'H', 'i'};
System.out.println(letterArray[0]);          // outputs H
System.out.println(letterArray.length);      // outputs 2

double[] numberArray = new double[365];
System.out.println(numberArray[0]);          // outputs 0.0
System.out.println(numberArray.length);      // outputs 365
\end{javalst}
\end{quote}

Array elements are accessed by \emph{index} number, starting at zero:

\begin{quote}
\begin{tabular}{C{2em}C{2em}}
\hline
\multicolumn{1}{|c|}{\java{'H'}} &
\multicolumn{1}{ c|}{\java{'i'}} \\
\hline
\fs 0 & \fs 1 \\
\end{tabular}
\hspace{3em}
\begin{tabular}{C{2em}C{2em}C{4em}C{2em}}
\hline
\multicolumn{1}{|c|}{\java{0.0}} &
\multicolumn{1}{ c|}{\java{0.0}} &
\multicolumn{1}{ c|}{$\cdots$} &
\multicolumn{1}{ c|}{\java{0.0}} \\
\hline
\fs 0 & \fs 1 &   & \fs 364 \\
\end{tabular}
\end{quote}


\quest{15 min}


\Q Examine the results of the code.

\begin{enumerate}
\item What is the length of \java{letterArray}? \ans[5em]{2}
\item What is the length of \java{numberArray}? \ans[5em]{365}
\item What is the index of the element \java{'i'} in \java{letterArray}? \ans[5em]{1}
\item What is the index of the last element of \java{numberArray}? \ans[5em]{\java{364}}
\end{enumerate}


\Q Now examine the syntax of the code.

\begin{enumerate}
\item What are three ways that square brackets [] are used?

\vspace{-1ex}
\begin{answer}[4em]
1) To declare the type: {\tt double[]}

2) To specify the length: {\tt double[365]}

3) To access an element: {\tt numberArray[0]}
\end{answer}

\item In contrast, how are curly braces \{\} used for an array?

\vspace{-1ex}
\begin{answer}[2em]
To create an array with an initial set of values.
\end{answer}
\end{enumerate}


\Q \label{typeval}
What are the resulting type and value of the following expressions?
Show your work by writing the value of each array element in the space provided.

\begin{javalst}
int[] a = {3, 6, 15, 22, 100, 0};
double[] b = {3.5, 4.5, 2.0, 2.0, 2.0};
String[] c = {"alpha", "beta", "gamma"};
\end{javalst}

\begin{enumerate}
\item \java{a[3] + a[2]}
\hfill
~ Type:  \ans[4em]{\tt int}
~ Value: \ans[4em]{\tt 37}
\hspace{8em} \\
\ans[2em]{22} ~~~ \ans[2em]{15}

\item \java{b[2] - b[0] + a[4]}
\hfill
~ Type:  \ans[4em]{\tt double}
~ Value: \ans[4em]{\tt 98.5}
\hspace{8em} \\
\ans[2em]{2.0} ~~~ \ans[2em]{3.5} ~~~ \ans[2em]{100}

\item \java{c[1].charAt(a[0])}
\hfill
~ Type:  \ans[4em]{\tt char}
~ Value: \ans[4em]{\tt \qs{a}\qs}
\hspace{8em} \\
\ans[2em]{beta} ~ ~ ~ ~ ~ ~ ~ \ans[2em]{3}

\item \java{a[4] * b[1] <= a[5] * a[0]}
\hfill
~ Type:  \ans[4em]{\tt boolean}
~ Value: \ans[4em]{\tt false}
\hspace{8em} \\
\ans[2em]{100} ~~~ \ans[2em]{4.5} ~~ ~~~ \ans[2em]{0} ~~~ \ans[2em]{3}

\end{enumerate}


\vspace{0pt}

\comment{
As shown in \ref{typeval}, an array variable can be declared and initialized without using \java{new}.
However, to assign an array variable that was previously declared, \java{new} is required: \\
~~~~~~~~\java{a = new int[] \{3, 6, 15, 22, 100, 0\};} \\[-1ex]
~~~~~~~~\java{c = new String[] \{"alpha", "beta", "gamma"\};}
}


\Q \label{arraysta}
Write statements that declare and initialize variables for the following arrays.

\begin{enumerate}

\item
\begin{tabular}{|C{3em}|C{3em}|C{3em}|C{3em}|C{3em}|C{3em}|}
\hline
0 & 14 & 1024 & 127 & 3 & 5521 \\
\hline
\end{tabular}

\vspace{1ex}
\ans[35em]{\tt int[] a = \{0, 14, 1024, 127, 3, 5521\};}

\item
\begin{tabular}{|C{3em}|C{3em}|C{3em}|C{3em}|C{3em}|C{3em}|C{3em}|}
\hline
3.23 & 1.52 & 4.23 & 32.5 & 2.45 & 5.23 & 3.33 \\
\hline
\end{tabular}

\vspace{1ex}
\ans[35em]{\tt double[] b = \{3.23, 1.52, 4.23, 32.5, 2.45, 5.23, 3.33\};}

\end{enumerate}


\Q \label{arrayexp}
Write statements that assign the following arrays to variables you declared in \ref{arraysta}.

\begin{enumerate}

\item
\begin{tabular}{|C{3em}|C{3em}|C{3em}|C{3em}|C{3em}|C{3em}|}
\hline
0 & 14 & 1024 & 127 & 3 & 5521 \\
\hline
\end{tabular}

\vspace{1ex}
\ans[35em]{\tt a = new int[] \{0, 14, 1024, 127, 3, 5521\}}

\item
\begin{tabular}{|C{3em}|C{3em}|C{3em}|C{3em}|C{3em}|C{3em}|C{3em}|}
\hline
3.23 & 1.52 & 4.23 & 32.5 & 2.45 & 5.23 & 3.33 \\
\hline
\end{tabular}

\vspace{1ex}
\ans[35em]{\tt b = new double[] \{3.23, 1.52, 4.23, 32.5, 2.45, 5.23, 3.33\}}

\end{enumerate}

\newpage
\model{Array Diagrams}

Array elements are stored together in one contiguous block of memory. To show arrays in memory diagrams, we simply draw adjacent boxes.

\begin{center}
\java{int[] nums = \{10, 3, 7, -5\};}

\vspace{1ex}
\includegraphics[width=225pt]{array-diagram1.png}
\end{center}


\quest{10 min}


\Q Draw a memory diagram for the following array declarations.

%TODO solution for array memory diagrams
\begin{enumerate}

\item
\begin{javalst}
int[] sizes = new int[5];
sizes[2] = 7;
\end{javalst}

\item
\begin{javalst}
char[] codes = new char[3];
codes[2] = 'X';
\end{javalst}

\item
\begin{javalst}
double[] costs = new double[4];
costs[0] = 0.99;
\end{javalst}

\end{enumerate}


\Q What is the \emph{default} value for uninitialized array elements? (Hint: You should have no empty boxes in your memory diagrams above.)

\begin{answer}
Zero or equivalent value, depending on the data type.
For numeric types like {\tt int} and {\tt double}, the default is 0; for {\tt boolean}, it's {\tt false}; for {\tt char}, it's {\tt \qs{\bs u0000}\qs}; for reference types, it's {\tt null}.
\end{answer}


\Q Like strings, arrays are reference types. What is the \emph{value} of an array variable?

\begin{answer}
The memory location of the array. If you assign one array variable to another, you're only copying the reference, not the array itself.
\end{answer}


\Q Draw a memory diagram of the following array.
(Hint: You should have four arrows.)

\begin{javalst}
String[] greek = {"alpha", "beta", "gamma"};
\end{javalst}

\begin{answer}
%TODO string array memory diagram
\end{answer}


\end{document}
