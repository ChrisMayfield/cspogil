\model{Arrays and Loops}

The real power of arrays is the ability to process them using loops, i.e., performing the same task for multiple elements.

\begin{javalst}
    for (int i = 0; i < array.length; i++) {
       // ... process array[i] ...
    }
\end{javalst}

Here are two specific examples:

\begin{javalst}
    // set all of the elements of x to -1.0
    double[] x = new double[100];
    for (int i = 0; i < x.length; i++) {
        x[i] = -1.0;
    }
    // sum the elements of scores
    int sum = 0;
    for (int i = 0; i < scores.length; i++) {
        sum += scores[i];
    }
\end{javalst}


\quest{20 min}


\Q What is the value of \java{array} and \java{accumulator} at the end of the following code?
Trace the loop by hand in the space below.

\begin{javalst}
int[] array = {5, 26, 13, 12, 37, 15, 16, 4, 1, 3};
int accumulator = 0;
for (int i = 0; i < array.length; i++) {
    if (array[i] % 2 == 1 && i + 1 < array.length) {
        array[i] *= -1;
        accumulator += array[i+1];
    }
}
\end{javalst}

\begin{answer}[12em]
\begin{tabular}{|C{20pt}|C{40pt}|C{40pt}|}
\hline
i & array[i] & accum \\
\hline
\hline
0 & 5 & 0 \\
\hline
1 & 26 & 26 \\
\hline
2 & 13 & 26 \\
\hline
3 & 12 & 38 \\
\hline
4 & 37 & 38 \\
\hline
\end{tabular}
\hspace{20pt}
\begin{tabular}{|C{20pt}|C{40pt}|C{40pt}|}
\hline
i & array[i] & accum \\
\hline
\hline
5 & 15 & 53 \\
\hline
6 & 16 & 69 \\
\hline
7 & 4 & 69 \\
\hline
8 & 1 & 69 \\
\hline
9 & 3 & 72 \\
\hline
\end{tabular}
\hspace{20pt}
\begin{minipage}{175pt}
\begin{javalst}
array:
  { -5, 26, -13, 12, -37,
   -15, 16,   4, -1,   3}

accumulator:
  72
\end{javalst}
\end{minipage}
\end{answer}


\newpage
\Q Implement the following method that creates and returns a new array.

\begin{javalst}
/**
 * Return a new array containing the pairwise maximum value of
 * the arguments. For each subscript i, the return value at [i]
 * will be the larger of x[i] and y[i].
 *
 * @param x an array of double values
 * @param y an array of double values
 * @return pairwise max of x and y
 */
public static double[] pairwiseMax(double[] x, double[] y) {
\end{javalst}

\vspace{-3ex}
\begin{answer}[11.3em]
\begin{javaans}
    double[] z = new double[x.length];
    for (int i = 0; i < x.length; i++) {
        if (x[i] > y[i]) {
            z[i] = x[i];
        } else {
            z[i] = y[i];
        }
    }
    return z;
\end{javaans}
\end{answer}

\begin{javalst}
}
\end{javalst}


\Q Implement the following method that reads through two integer arrays.

\begin{javalst}
/**
 * Computes the final average grade for a student. The labs are
 * worth 40% and the exams are worth 60%. All scores range from
 * 0 to 100, inclusive.
 *
 * @param labs the student's lab scores
 * @param exams the student's exam scores
 * @return weighted average of all scores
 */
public static double finalGrade(int[] labs, int[] exams) {
\end{javalst}

\vspace{-3ex}
\begin{answer}[13.7em]
\begin{javaans}
    int labSum = 0;
    for (int score : labs) {
        labSum += score;
    }
    int examSum = 0;
    for (int score : exams) {
        examSum += score;
    }
    double labGrade = 1.0 * labSum / labs.length;
    double examGrade = 1.0 * examSum / exams.length;
    return 0.40 * labGrade + 0.60 * examGrade;
\end{javaans}
\end{answer}

\begin{javalst}
}
\end{javalst}
