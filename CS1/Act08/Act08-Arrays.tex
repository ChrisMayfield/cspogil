\documentclass[12pt]{article}

\title{Activity 7: Arrays}
\author{Chris Mayfield and Helen Hu}
\date{July 2017}

%\ProvidesPackage{cspogil}

% fonts
\usepackage[utf8]{inputenc}
\usepackage[T1]{fontenc}
\usepackage{mathpazo}

% spacing
\usepackage[margin=2cm]{geometry}
\renewcommand{\arraystretch}{1.4}
\setlength{\parindent}{0pt}

% orphans and widows
\clubpenalty=10000
\widowpenalty=10000
\pagestyle{empty}

% figures and tables
\usepackage{graphicx}
\usepackage{multicol}
\usepackage{tabularx}

% fixed-width columns
\usepackage{array}
\newcolumntype{L}[1]{>{\raggedright\let\newline\\\arraybackslash\hspace{0pt}}m{#1}}
\newcolumntype{C}[1]{>{\centering\let\newline\\\arraybackslash\hspace{0pt}}m{#1}}
\newcolumntype{R}[1]{>{\raggedleft\let\newline\\\arraybackslash\hspace{0pt}}m{#1}}

% include paths
\makeatletter
\def\input@path{{Models/}{../../Models/}}
\graphicspath{{Models/}{../../Models/}}
\makeatother

% colors
\usepackage[svgnames,table]{xcolor}
\definecolor{bgcolor}{HTML}{FAFAFA}
\definecolor{comment}{HTML}{007C00}
\definecolor{keyword}{HTML}{0000FF}
\definecolor{strings}{HTML}{B20000}

% table headers
\newcommand{\tr}{\bf\cellcolor{Yellow!10}}

% syntax highlighting
\usepackage{textcomp}
\usepackage{listings}
\lstset{
    basicstyle=\ttfamily,
    backgroundcolor=\color{bgcolor},
    numberstyle=\scriptsize\color{comment},
    commentstyle=\color{comment},
    keywordstyle=\color{keyword},
    stringstyle=\color{strings},
    columns=fullflexible,
    keepspaces=true,
    showstringspaces=false,
    upquote=true
}

% code environments
\newcommand{\java}[1]{\lstinline[language=java]{#1}}%[
\lstnewenvironment{javalst}{\lstset{language=java,backgroundcolor=}}{}
\lstnewenvironment{javabox}{\lstset{language=java,frame=single,numbers=left}\quote}{\endquote}

% PDF properties
\usepackage[pdftex]{hyperref}
\urlstyle{same}
\makeatletter
\hypersetup{
  pdftitle={\@title},
  pdfauthor={\@author},
  pdfsubject={\@date},
  pdfkeywords={},
  bookmarksopen=false,
  colorlinks=true,
  citecolor=black,
  filecolor=black,
  linkcolor=black,
  urlcolor=blue
}
\makeatother

% titles
\makeatletter
\renewcommand{\maketitle}{\begin{center}\LARGE\@title\end{center}}
\makeatother

% boxes
\newcommand{\emptybox}[1][10em]{
\vspace{1em}
\begin{tabularx}{\linewidth}{|X|}
\hline\\[#1]\hline
\end{tabularx}}

% models
\newcommand{\model}[1]{\section{#1}\nopagebreak}
\renewcommand{\thesection}{Model~\arabic{section}}

% questions
\newcommand{\quest}[1]{\subsection*{Questions~ (#1)}}
\newcounter{question}
\newcommand{\Q}{\vspace{1em}\refstepcounter{question}\arabic{question}.~ }
\renewcommand{\thequestion}{\#\arabic{question}}

% sub-question lists
\usepackage{enumitem}
\setenumerate[1]{label=\alph*)}
\setlist{itemsep=1em,after=\vspace{1ex}}

% inline answers
\definecolor{answers}{HTML}{C0C0C0}
\newcommand{\ans}[1]{%
\ifdefined\Student
    \phantom{~~\textcolor{answers}{#1}}
\else
    ~~\textcolor{answers}{#1}
\fi}

% longer answers [optional height]
\newsavebox{\ansbox}
\newenvironment{answer}[1][4em]{
\nopagebreak
\begin{lrbox}{\ansbox}
\begin{minipage}[t][#1]{\linewidth}
\color{answers}
}{
\end{minipage}
\end{lrbox}
\ifdefined\Student
    \phantom{\usebox{\ansbox}}%
\else
    \usebox{\ansbox}%
\fi}


\begin{document}

\maketitle

Programs often need to store multiple values of the same type, such as a list of 1000 phone numbers or the names of your top 20 favorite songs.
Rather than create a separate variable for each one, we can store them together using an array.

\guide{
  \item Explain course policies about academic honesty.
  \item Declare and initialize array variables of primitive types.
  \item Draw a memory digram of an array of reference types.
  \item Write a for loop that iterates the contents of an array.
}{
  \item Developing algorithms that loop through two arrays to compute a result. (Problem Solving)
}{
This activity introduces arrays for the first time, with a focus on array syntax and memory diagrams.
Students learn about zero-based indexing array processing using loops.

Be sure to discuss \#4 and \#5 while reporting out. Students often confuse \textbf{expressions} with \textbf{statements}. Explain how Java generally requires the \texttt{new} operator when creating arrays.

In Model 2, explain how the \texttt{new} operator automatically zeros-out memory for the array. Therefore, the default value will be 0 for integers, 0.0 for doubles, \texttt{null} for strings, etc.

Students may have the most trouble with Model 3, particularly if writing code on paper. One way to jump start the process is to report out before teams write their solution. Give each team 2--3 minutes to develop an algorithm and have the presenter describe it to the class. If short on time, have half of the teams work on \#11 and the other half work on \#12.
}

\model{Array Syntax}
% based on Model 1 of Activity 13 - TwoD Arrays by Helen Hu

An \emph{array} variable allows you to store multiple variables (of the same type).
Each value in an array is known as an \emph{element}.
The programmer must specify the \emph{length} of the array (the number of array elements).
Once the array is created, its length cannot be changed.

\begin{quote}
\begin{javalst}
char[] letterArray = {'H', 'i'};
System.out.println(letterArray[0]);          // outputs H
System.out.println(letterArray.length);      // outputs 2

double[] numberArray = new double[365];
System.out.println(numberArray[0]);          // outputs 0.0
System.out.println(numberArray.length);      // outputs 365
\end{javalst}
\end{quote}

Array elements are accessed by \emph{index} number, starting at zero:

\begin{quote}
\begin{tabular}{C{2em}C{2em}}
\hline
\multicolumn{1}{|c|}{\java{'H'}} &
\multicolumn{1}{ c|}{\java{'i'}} \\
\hline
\fs 0 & \fs 1 \\
\end{tabular}
\hspace{3em}
\begin{tabular}{C{2em}C{2em}C{4em}C{2em}}
\hline
\multicolumn{1}{|c|}{\java{0.0}} &
\multicolumn{1}{ c|}{\java{0.0}} &
\multicolumn{1}{ c|}{$\cdots$} &
\multicolumn{1}{ c|}{\java{0.0}} \\
\hline
\fs 0 & \fs 1 &   & \fs 364 \\
\end{tabular}
\end{quote}


\quest{15 min}


\Q Examine the results of the code.

\begin{enumerate}
\item What is the length of \java{letterArray}? \ans[5em]{2}
\item What is the length of \java{numberArray}? \ans[5em]{365}
\item What is the index of the element \java{'i'} in \java{letterArray}? \ans[5em]{1}
\item What is the index of the last element of \java{numberArray}? \ans[5em]{\java{364}}
\end{enumerate}


\Q Now examine the syntax of the code.

\begin{enumerate}
\item What are three ways that square brackets [] are used?

\vspace{-1ex}
\begin{answer}[4em]
1) To declare the type: {\tt double[]}

2) To specify the length: {\tt double[365]}

3) To access an element: {\tt numberArray[0]}
\end{answer}

\item In contrast, how are curly braces \{\} used for an array?

\vspace{-1ex}
\begin{answer}[2em]
To create an array with an initial set of values.
\end{answer}
\end{enumerate}


\Q \label{typeval}
What are the resulting type and value of the following expressions?
Show your work by writing the value of each array element in the space provided.

\begin{javalst}
int[] a = {3, 6, 15, 22, 100, 0};
double[] b = {3.5, 4.5, 2.0, 2.0, 2.0};
String[] c = {"alpha", "beta", "gamma"};
\end{javalst}

\begin{enumerate}
\item \java{a[3] + a[2]}
\hfill
~ Type:  \ans[4em]{\tt int}
~ Value: \ans[4em]{\tt 37}
\hspace{8em} \\
\ans[2em]{22} ~~~ \ans[2em]{15}

\item \java{b[2] - b[0] + a[4]}
\hfill
~ Type:  \ans[4em]{\tt double}
~ Value: \ans[4em]{\tt 98.5}
\hspace{8em} \\
\ans[2em]{2.0} ~~~ \ans[2em]{3.5} ~~~ \ans[2em]{100}

\item \java{c[1].charAt(a[0])}
\hfill
~ Type:  \ans[4em]{\tt char}
~ Value: \ans[4em]{\tt \qs{a}\qs}
\hspace{8em} \\
\ans[2em]{beta} ~ ~ ~ ~ ~ ~ ~ \ans[2em]{3}

\item \java{a[4] * b[1] <= a[5] * a[0]}
\hfill
~ Type:  \ans[4em]{\tt boolean}
~ Value: \ans[4em]{\tt false}
\hspace{8em} \\
\ans[2em]{100} ~~~ \ans[2em]{4.5} ~~ ~~~ \ans[2em]{0} ~~~ \ans[2em]{3}

\end{enumerate}


\vspace{0pt}

\comment{
As shown in \ref{typeval}, an array variable can be declared and initialized without using \java{new}.
However, to assign an array variable that was previously declared, \java{new} is required: \\
~~~~~~~~\java{a = new int[] \{3, 6, 15, 22, 100, 0\};} \\[-1ex]
~~~~~~~~\java{c = new String[] \{"alpha", "beta", "gamma"\};}
}


\Q \label{arraysta}
Write statements that declare and initialize variables for the following arrays.

\begin{enumerate}

\item
\begin{tabular}{|C{3em}|C{3em}|C{3em}|C{3em}|C{3em}|C{3em}|}
\hline
0 & 14 & 1024 & 127 & 3 & 5521 \\
\hline
\end{tabular}

\vspace{1ex}
\ans[35em]{\tt int[] a = \{0, 14, 1024, 127, 3, 5521\};}

\item
\begin{tabular}{|C{3em}|C{3em}|C{3em}|C{3em}|C{3em}|C{3em}|C{3em}|}
\hline
3.23 & 1.52 & 4.23 & 32.5 & 2.45 & 5.23 & 3.33 \\
\hline
\end{tabular}

\vspace{1ex}
\ans[35em]{\tt double[] b = \{3.23, 1.52, 4.23, 32.5, 2.45, 5.23, 3.33\};}

\end{enumerate}


\Q \label{arrayexp}
Write statements that assign the following arrays to variables you declared in \ref{arraysta}.

\begin{enumerate}

\item
\begin{tabular}{|C{3em}|C{3em}|C{3em}|C{3em}|C{3em}|C{3em}|}
\hline
0 & 14 & 1024 & 127 & 3 & 5521 \\
\hline
\end{tabular}

\vspace{1ex}
\ans[35em]{\tt a = new int[] \{0, 14, 1024, 127, 3, 5521\}}

\item
\begin{tabular}{|C{3em}|C{3em}|C{3em}|C{3em}|C{3em}|C{3em}|C{3em}|}
\hline
3.23 & 1.52 & 4.23 & 32.5 & 2.45 & 5.23 & 3.33 \\
\hline
\end{tabular}

\vspace{1ex}
\ans[35em]{\tt b = new double[] \{3.23, 1.52, 4.23, 32.5, 2.45, 5.23, 3.33\}}

\end{enumerate}

\model{Array Diagrams}

Array elements are stored together in one contiguous block of memory. To show arrays in memory diagrams, we simply draw adjacent boxes.

\begin{center}
\java{int[] nums = \{10, 3, 7, -5\};}

\vspace{1ex}
\includegraphics[width=225pt]{array-diagram1.png}
\end{center}


\quest{10 min}


\Q Draw a memory diagram for the following array declarations.

%TODO solution for array memory diagrams
\begin{enumerate}

\item
\begin{javalst}
int[] sizes = new int[5];
sizes[2] = 7;
\end{javalst}

\item
\begin{javalst}
char[] codes = new char[3];
codes[2] = 'X';
\end{javalst}

\item
\begin{javalst}
double[] costs = new double[4];
costs[0] = 0.99;
\end{javalst}

\end{enumerate}


\Q What is the \emph{default} value for uninitialized array elements? (Hint: You should have no empty boxes in your memory diagrams above.)

\begin{answer}
Zero or equivalent value, depending on the data type.
For numeric types like {\tt int} and {\tt double}, the default is 0; for {\tt boolean}, it's {\tt false}; for {\tt char}, it's {\tt \qs{\bs u0000}\qs}; for reference types, it's {\tt null}.
\end{answer}


\Q Like strings, arrays are reference types. What is the \emph{value} of an array variable?

\begin{answer}
The memory location of the array. If you assign one array variable to another, you're only copying the reference, not the array itself.
\end{answer}


\Q Draw a memory diagram of the following array.
(Hint: You should have four arrows.)

\begin{javalst}
String[] greek = {"alpha", "beta", "gamma"};
\end{javalst}

\begin{answer}
%TODO string array memory diagram
\end{answer}

%  \item Write a for loop that iterates the contents of an array.

%Students will have the most trouble with \ref{array-loops.tex}, particularly if writing code on paper.
%One way to jump start the process is to report out before teams write their solution.
%Give each team 2--3 minutes to develop an algorithm and have the presenter describe it to the class.
%Then have students complete \ref{pairwiseMax} and \ref{finalGrade} as an exercise outside of class.


\model{Arrays and Loops ~ (optional)}

The real power of arrays is the ability to process them using loops, i.e., performing the same task for multiple elements.

\begin{javalst}
    for (int i = 0; i < array.length; i++) {
       // ... process array[i] ...
    }
\end{javalst}

Here are two specific examples:

\begin{javalst}
    // set all of the elements of x to -1.0
    double[] x = new double[100];
    for (int i = 0; i < x.length; i++) {
        x[i] = -1.0;
    }
    // sum the elements of scores
    int sum = 0;
    for (int i = 0; i < scores.length; i++) {
        sum += scores[i];
    }
\end{javalst}


\quest{15 min}


\Q What is the value of \java{array} and \java{accumulator} at the end of the following code?
Trace the loop by hand in the space below.

\begin{javalst}
int[] array = {5, 26, 13, 12, 37, 15, 16, 4, 1, 3};
int accumulator = 0;
for (int i = 0; i < array.length; i++) {
    if (array[i] % 2 == 1 && i + 1 < array.length) {
        array[i] *= -1;
        accumulator += array[i+1];
    }
}
\end{javalst}

\begin{answer}[12em]
\begin{tabular}{|C{20pt}|C{40pt}|C{40pt}|}
\hline
i & array[i] & accum \\
\hline
\hline
0 & 5 & 0 \\
\hline
1 & 26 & 26 \\
\hline
2 & 13 & 26 \\
\hline
3 & 12 & 38 \\
\hline
4 & 37 & 38 \\
\hline
\end{tabular}
\hspace{20pt}
\begin{tabular}{|C{20pt}|C{40pt}|C{40pt}|}
\hline
i & array[i] & accum \\
\hline
\hline
5 & 15 & 53 \\
\hline
6 & 16 & 69 \\
\hline
7 & 4 & 69 \\
\hline
8 & 1 & 69 \\
\hline
9 & 3 & 72 \\
\hline
\end{tabular}
\hspace{20pt}
\begin{minipage}{175pt}
\begin{javalst}
array:
  { -5, 26, -13, 12, -37,
   -15, 16,   4, -1,   3}

accumulator:
  72
\end{javalst}
\end{minipage}
\end{answer}


\newpage

\Q \label{pairwiseMax}
Implement the following method that creates and returns a new array.

\begin{javalst}
/**
 * Return a new array containing the pairwise maximum value of
 * the arguments. For each subscript i, the return value at [i]
 * will be the larger of x[i] and y[i].
 *
 * @param x an array of double values
 * @param y an array of double values
 * @return pairwise max of x and y
 */
public static double[] pairwiseMax(double[] x, double[] y) {
\end{javalst}

\vspace{-2ex}
\begin{answer}[11em]
\begin{javaans}
    double[] z = new double[x.length];
    for (int i = 0; i < x.length; i++) {
        if (x[i] > y[i]) {
            z[i] = x[i];
        } else {
            z[i] = y[i];
        }
    }
    return z;
\end{javaans}
\end{answer}
\ifdefined\Teacher
\vspace{-2ex}
\else
\vspace{-2em}
\fi

\begin{javalst}
}
\end{javalst}


\Q \label{finalGrade}
Implement the following method that reads through two integer arrays.

\begin{javalst}
/**
 * Computes the final average grade for a student. The labs are
 * worth 40% and the exams are worth 60%. All scores range from
 * 0 to 100, inclusive.
 *
 * @param labs the student's lab scores
 * @param exams the student's exam scores
 * @return weighted average of all scores
 */
public static double finalGrade(int[] labs, int[] exams) {
\end{javalst}

\vspace{-2ex}
\begin{answer}[13em]
\begin{javaans}
    int labSum = 0;
    for (int score : labs) {
        labSum += score;
    }
    int examSum = 0;
    for (int score : exams) {
        examSum += score;
    }
    double labGrade = 1.0 * labSum / labs.length;
    double examGrade = 1.0 * examSum / exams.length;
    return 0.40 * labGrade + 0.60 * examGrade;
\end{javaans}
\end{answer}
\ifdefined\Teacher
\vspace{-2ex}
\else
\vspace{-2em}
\fi

\begin{javalst}
}
\end{javalst}


\end{document}
