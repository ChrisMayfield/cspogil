% based on Model 1 of Activity 13 - TwoD Arrays by Helen Hu

\model{Array Syntax}

%An \emph{array} allows you to declare a collection of related variables (of the same type) at once.
Each value in an array is known as an \emph{element}.
The programmer must specify the \emph{length} of the array (the number of array elements).
Once the array is created, its length cannot be changed.

\begin{quote}
\begin{javalst}
String[] wordArray = {"hello", "world"};
System.out.println(wordArray[0]);            // outputs hello
System.out.println(wordArray.length);        // outputs 2

double[] numberArray = new double[365];
System.out.println(numberArray[0]);          // outputs 0.0
System.out.println(numberArray.length);      // outputs 365
\end{javalst}
\end{quote}

Array elements are accessed by \emph{index} number, starting at zero:

\begin{quote}
\begin{tabular}{cc}
\hline
\multicolumn{1}{|c|}{\java{"hello"}} &
\multicolumn{1}{ c|}{\java{"world"}} \\
\hline
\fs 0 & \fs 1 \\
\end{tabular}
\hspace{3em}
\begin{tabular}{ccC{4em}c}
\hline
\multicolumn{1}{|c|}{\java{0.0}} &
\multicolumn{1}{ c|}{\java{0.0}} &
\multicolumn{1}{ c|}{$\cdots$} &
\multicolumn{1}{ c|}{\java{0.0}} \\
\hline
\fs 0 & \fs 1 &   & \fs 364 \\
\end{tabular}
\end{quote}


\quest{15 min}


\Q Examine the results of the above code.

\begin{enumerate}
\item What is the index for the element \java{"world"}? \ans{1}
\item What is the length of the \java{wordArray}? \ans{2}
\item What is the length of the \java{numberArray}? \ans{365}
\item How would you access the last element of \java{numberArray}? \ans{\java{numberArray[364]}}
\end{enumerate}


\Q Now examine the syntax of the code.

\begin{enumerate}
\item What are three ways that square brackets [] are used? \ans{
1) To declare the type: {\tt String[]}.
2) To specify the length: {\tt double[365]}.
3) To access an element: {\tt wordArray[0]}.}

\item In contrast, how are curly braces {} used for an array?
\ans{To create an array with an initial set of values.}
\end{enumerate}


\begin{tabularx}{\linewidth}{|X|}
\hline
Array variables can be initialized without the \java{new} keyword: \\
~~~~~~~~\java{int[] picks = \{3, 5, 7, 2, 1\};} \\[-1ex]
~~~~~~~~\java{String[] names = \{"Grace", "Alan", "Tim"\};} \\[1ex]

However, if the variable is already declared, \java{new} is required: \\
~~~~~~~~\java{picks = new int[] \{3, 5, 7, 2, 1\};} \\[-1ex]
~~~~~~~~\java{names = new String[] \{"Grace", "Alan", "Tim"\};} \\
\hline
\end{tabularx}
\vspace{1ex}


\Q Write \emph{expressions} that create the following \java{new} arrays. (Do not declare variables.)

\begin{enumerate}

\item
\begin{tabular}{|C{3em}|C{3em}|C{3em}|C{3em}|C{3em}|C{3em}|}
\hline
0 & 14 & 1024 & 127 & 3 & 5521 \\
\hline
\end{tabular}

\vspace{1ex}
\ans{\tt new int[] \{0, 14, 1024, 127, 3, 5521\}}

\item 
\begin{tabular}{|C{3em}|C{3em}|C{3em}|C{3em}|C{3em}|C{3em}|C{3em}|}
\hline
3.23 & 1.52 & 4.23 & 32.5 & 2.45 & 5.23 & 3.33 \\
\hline
\end{tabular}

\vspace{1ex}
\ans{\tt new double[] \{3.23, 1.52, 4.23, 32.5, 2.45, 5.23, 3.33\}}

\end{enumerate}


\Q Write \emph{statements} that both declare and initialize variables for these \java{new} arrays.

\begin{enumerate}

\item
\begin{tabular}{|C{3em}|C{3em}|C{3em}|C{3em}|C{3em}|C{3em}|}
\hline
0 & 14 & 1024 & 127 & 3 & 5521 \\
\hline
\end{tabular}

\vspace{1ex}
\ans{\tt int[] a = \{0, 14, 1024, 127, 3, 5521\};}

\item 
\begin{tabular}{|C{3em}|C{3em}|C{3em}|C{3em}|C{3em}|C{3em}|C{3em}|}
\hline
3.23 & 1.52 & 4.23 & 32.5 & 2.45 & 5.23 & 3.33 \\
\hline
\end{tabular}

\vspace{1ex}
\ans{\tt double[] b = \{3.23, 1.52, 4.23, 32.5, 2.45, 5.23, 3.33\};}

\end{enumerate}


\Q What are the type and value for each of the four \emph{expressions} below?

\begin{javalst}
int[] a = {3, 6, 15, 22, 100, 0};
double[] b = {3.5, 4.5, 2.0, 2.0, 2.0};
String[] c = {"alpha", "beta", "gamma"};
\end{javalst}

\begin{enumerate}
\item \java{a[3] + a[2]}
\ans{Type: {\tt int}, Value: {\tt 37}}

\item \java{b[2] - b[0] + a[4]}
\ans{Type: {\tt double}, Value: {\tt 98.5}}

\item \java{c[1].charAt(a[0])}
\ans{Type: {\tt char}, Value: {\tt \qs{a}\qs}}

\item \java{a[4] * b[1] <= a[5] * a[0]}
\ans{Type: {\tt boolean}, Value: {\tt false}}
\end{enumerate}
