% based on Model 1 of Activity 12 - Recursion by Helen Hu, with modifications by Dee Weikle

\model{Factorial}

''In mathematics, the \emph{factorial} of a non-negative integer $n$, denoted by $n!$, is the product of all positive integers less than or equal to $n$. For example, $5! = 5 \times 4 \times 3 \times 2 \times 1 = 120$.''

\smallskip\hfill
Source: \url{https://en.wikipedia.org/wiki/Factorial}

\begin{center}
\begin{tabular}{|C{1cm}|C{1cm}|}
\hline
\tr $n$ & \tr $n!$ \\
\hline
0 & 1 \\
\hline
1 & 1 \\
\hline
2 & 2 \\
\hline
3 & 6 \\
\hline
4 & 24 \\
\hline
5 & 120 \\
%\hline
%6 & 720 \\
\hline
\end{tabular}
\end{center}


\quest{20 min}


\Q \label{fact4} Consider how to calculate $4! = 24$.

\begin{enumerate}
\item Write out all the numbers that need to be multiplied:

$4! =$ \ans{4 * 3 * 2 * 1}

\item Rewrite the expression using 3! instead of $3 \times 2 \times 1$:

$4! =$ \ans{4 * 3!}
\end{enumerate}


\Q \label{factn} Write an expression similar to \ref{fact4}b showing how each factorial can be calculated in terms of a simpler factorial.

\begin{enumerate}
\item $3! =$ \ans{3 * 2!}
\item $2! =$ \ans{2 * 1!}
\item $100! =$ \ans{100 * 99!}
\item $n! =$ \ans{$n * (n - 1)!$}
\end{enumerate}


\Q What is the value of $0!$ based on \ref{\currfilename}?
Does it make sense to define $0!$ in terms of a simpler factorial?
Why or why not?

\begin{answer}
$0!$ is 1 by convention for an empty product.
We can't say $0 \times -1!$, because factorial is only defined for non-negative integers.
At some point we need to define the solution in concrete terms, without referencing itself.
\end{answer}


\comment{
If we repeatedly break down a problem into smaller versions of itself, we eventually reach a basic problem that can't be broken down any further.
Such a problem, like $0!$, is referred to as the \textbf{base case}.
}


\Q Assume you already have a working method named ~\java{factorial(int n)} that returns $n!$ for any positive integer.

\begin{enumerate}
\item Review your answer to \ref{factn}c that shows how to compute $100!$ using a simpler factorial.
Convert this expression to Java by using the \java{factorial} method instead of the \java{!} operator.

\ans{\java{100 * factorial(99)}}

\item Now rewrite your answer to \ref{factn}d in Java using the variable \java{n}.

\ans{\java{n * factorial(n - 1)}}
\end{enumerate}


\Q \label{factjava} Here is a \java{factorial} method that includes output for debugging:

\begin{javabox}
public static int factorial(int n) {
    System.out.println("n is " + n);
    if (n == 0) {
        return 1;  // base case
    } else {
        System.out.printf("need factorial of %d\n", n - 1);
        int answer = factorial(n - 1);
        System.out.printf("factorial of %d is %d\n", n - 1, answer);
        return n * answer;
    }
}

public static void main(String[] args) {
    System.out.println(factorial(3));
}
\end{javabox}

\begin{enumerate}
\item What specific method is invoked on line 7?

\ans{The \java{factorial} method invokes itself with a smaller argument.}

\item Why is the \java{if} statement required on line 3?

\ans{Without the base case, it would invoke itself forever (until running out of memory).}
\end{enumerate}


\Q A method that invokes itself is called \textbf{recursive}.
What two steps were necessary to define the \java{factorial} method?
How were these steps implemented in Java?

\begin{answer}
1. The base case, which was implemented using an if statement. \\
2. The recursive case, which as implemented using a method call.
\end{answer}


\Q How many distinct method calls would be made to \java{factorial} to compute the factorial of 3?
Identify the value of the parameter $n$ for each of these separate calls.

\begin{answer}
Four method calls: \java{factorial(3)} $\to$ \java{factorial(2)} $\to$ \java{factorial(1)} $\to$ \java{factorial(0)}.
\end{answer}


\Q Here is the complete output from the program in \ref{factjava}.
Identify which distinct method call printed each line.
In other words, which lines were printed by \java{factorial(3)}, which lines were printed by \java{factorial(2)}, and so on.

\vspace{-1ex}
\begin{center}
\renewcommand{\arraystretch}{1.2}
\begin{tabular}{ll}
\tt n is 3              & \tt \ans{factorial(3)} \\
\tt need factorial of 2 & \tt \ans{factorial(3)} \\
\tt n is 2              & \tt \ans{factorial(2)} \\
\tt need factorial of 1 & \tt \ans{factorial(2)} \\
\tt n is 1              & \tt \ans{factorial(1)} \\
\tt need factorial of 0 & \tt \ans{factorial(1)} \\
\tt n is 0              & \tt \ans{factorial(0)} \\
\tt factorial of 0 is 1 & \tt \ans{factorial(1)} \\
\tt factorial of 1 is 1 & \tt \ans{factorial(2)} \\
\tt factorial of 2 is 2 & \tt \ans{factorial(3)} \\
\tt 6                   & \tt \ans{main} \\
\end{tabular}
\end{center}


\Q What happens if you try to calculate the factorial of a negative number?
How could you prevent this behavior in the \java{factorial} method?

\begin{answer}
The recursion would repeat until the program runs out of memory (\java{StackOverflowError}).
To fix this bug, you could add an if statement that checks for \java{n < 0} and returns -1.
\end{answer}


\Q Trivia question: What is the largest factorial you can compute in Java when using \java{int} as the data type? If you don't know, how could you find out?

\begin{answer}
$12! = 479,001,600$. Anything larger exceeds the 32-bit range. $20!$ is the largest for \texttt{long} integers.
\end{answer}
