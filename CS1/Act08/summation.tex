\model{Summation}
% based on Model 2 of Activity 12 - Recursion by Helen Hu, with modifications by Dee Weikle

''In mathematics, \emph{summation} (capital Greek sigma symbol: $\Sigma$) is the addition of a sequence of numbers; the result is their sum or total.''

$$ \sum_{i=1}^{100} i = 1 + 2 + 3 + \ldots + 100 = 5050 $$

\smallskip\hfill
Source: \url{https://en.wikipedia.org/wiki/Summation}


\quest{20 min}


\Q \label{sum4}
Consider how to calculate $\sum\limits_{i=1}^{4} i = 10$.

\begin{enumerate}
\item Write out all the numbers that need to be added:

$\sum\limits_{i=1}^{4} i =$ \ans{4 + 3 + 2 + 1}

\item Show how this sum can be calculated in terms of a smaller summation.

$\sum\limits_{i=1}^{4} i =$ \ans{4 + $\sum\limits_{i=1}^{3} i$}
\end{enumerate}


\Q Write an expression similar to \ref{sum4}b showing how any summation of $n$ integers can be calculated in terms of a smaller summation.

\begin{center}
$\sum\limits_{i=1}^{n} i =$ \ans{n + $\sum\limits_{i=1}^{n-1} i$}
\end{center}


\Q What is the base case of the summation? (Write the complete formula, not just the value.)

\begin{answer}[3em]
$$\sum\limits_{i=1}^{1} i = 1$$
\end{answer}


\Q Implement a recursive method that takes a single parameter \java{n} and returns the sum $1 + 2 + \ldots + n$.
It should only have an \java{if} statement and two \java{return} statements.

\begin{javalst}
public static int summation(int n) {
\end{javalst}
\vspace{-1em}
\begin{quote}
\begin{answer}[6em]
\begin{javaans}
if (n == 1) {
    return 1;
} else {
    return n + summation(n - 1);
}
\end{javaans}
\end{answer}
\end{quote}
\vspace{-1em}
\begin{javalst}
}
\end{javalst}


\newpage


\Q Discuss how the \java{factorial} method below uses temporary variables.
What lines would you have to change to implement the \java{summation} method instead?

\begin{multicols}{2}

\vspace{1ex}
\begin{javalst}
public static int factorial(int n) {
    if (n == 0) {
        return 1;  // base case
    }
    int recurse = factorial(n - 1);
    int result = n * recurse;
    return result;
}
\end{javalst}

\columnbreak

\vspace*{0pt}
\begin{answer}[6em]
1) Rename the method to \java{summation}. \\[1ex]
2) Change the base case to be \texttt{if (n == 1)}. \\[1ex]
3) The recursive step must invoke \java{summation}. \\[1ex]
4) The result must add instead of multiply.
\end{answer}

\end{multicols}


\Q \label{diagram}
Here is a stack diagram of ~\java{factorial(3)} when invoked from \java{main}.
Draw a similar diagram for \java{summation(3)} as described in the previous question.

\vspace{1em}
\begin{minipage}{0.48\linewidth}

\includegraphics[width=\linewidth]{stack3.pdf}

\end{minipage}
\hfill
\begin{minipage}{0.48\linewidth}

\note{
\includegraphics[width=\linewidth]{stack3-ans.pdf}
}

\end{minipage}
\vspace{1em}


\Q Why are there no values for \java{recurse} and \java{result} in the stack diagram for the last call to \java{factorial} (when \java{n == 0})?

\begin{answer}
The method returns without declaring and using those variables.
\end{answer}


\Q Looking at the stack diagram, how is it possible that the parameter \java{n} can have multiple values in memory at the same time?

\begin{answer}
Each distinct method call has its own memory for parameters and local variables.
The value of \java{n - 1} in the first method call becomes the value of \java{n} in the next.
\end{answer}
