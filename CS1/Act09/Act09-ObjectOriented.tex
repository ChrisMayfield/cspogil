\documentclass[12pt]{article}

\title{Activity 9: Object-Oriented}
\author{Chris Mayfield and Helen Hu}
\date{July 2017}

%\ProvidesPackage{cspogil}

% fonts
\usepackage[utf8]{inputenc}
\usepackage[T1]{fontenc}
\usepackage{mathpazo}

% spacing
\usepackage[margin=2cm]{geometry}
\renewcommand{\arraystretch}{1.4}
\setlength{\parindent}{0pt}

% orphans and widows
\clubpenalty=10000
\widowpenalty=10000
\pagestyle{empty}

% figures and tables
\usepackage{graphicx}
\usepackage{multicol}
\usepackage{tabularx}
\usepackage{wrapfig}

% fixed-width columns
\usepackage{array}
\newcolumntype{L}[1]{>{\raggedright\let\newline\\\arraybackslash\hspace{0pt}}m{#1}}
\newcolumntype{C}[1]{>{\centering\let\newline\\\arraybackslash\hspace{0pt}}m{#1}}
\newcolumntype{R}[1]{>{\raggedleft\let\newline\\\arraybackslash\hspace{0pt}}m{#1}}

% include paths
\makeatletter
\def\input@path{{Models/}{../../Models/}}
\graphicspath{{Models/}{../../Models/}}
\makeatother

% colors
\usepackage[svgnames,table]{xcolor}
\definecolor{bgcolor}{HTML}{FAFAFA}
\definecolor{comment}{HTML}{007C00}
\definecolor{keyword}{HTML}{0000FF}
\definecolor{strings}{HTML}{B20000}

% table headers
\newcommand{\tr}{\bf\cellcolor{Yellow!10}}

% syntax highlighting
\usepackage{textcomp}
\usepackage{listings}
\lstset{
    basicstyle=\ttfamily\color{black},
    backgroundcolor=\color{bgcolor},
    numberstyle=\scriptsize\color{comment},
    commentstyle=\color{comment},
    keywordstyle=\color{keyword},
    stringstyle=\color{strings},
    columns=fullflexible,
    keepspaces=true,
    showlines=true,
    showstringspaces=false,
    upquote=true
}

% code environments
\newcommand{\java}[1]{\lstinline[language=java]{#1}}%[
\lstnewenvironment{javalst}{\lstset{language=java,backgroundcolor=}}{}
\lstnewenvironment{javabox}{\lstset{language=java,frame=single,numbers=left}\quote}{\endquote}

% PDF properties
\usepackage[pdftex]{hyperref}
\urlstyle{same}
\makeatletter
\hypersetup{
  pdftitle={\@title},
  pdfauthor={\@author},
  pdfsubject={\@date},
  pdfkeywords={},
  bookmarksopen=false,
  colorlinks=true,
  citecolor=black,
  filecolor=black,
  linkcolor=black,
  urlcolor=blue
}
\makeatother

% titles
\makeatletter
\renewcommand{\maketitle}{\begin{center}\LARGE\@title\end{center}}
\makeatother

% boxes [optional height]
\newcommand{\emptybox}[1][10em]{
\vspace{1em}
\begin{tabularx}{\linewidth}{|X|}
\hline\\[#1]\hline
\end{tabularx}}

% models
\newcommand{\model}[1]{\section{#1}\nopagebreak}
\renewcommand{\thesection}{Model~\arabic{section}}

% questions
\newcommand{\quest}[1]{\subsection*{Questions~ (#1)}}
\newcounter{question}
\newcommand{\Q}{\vspace{1em}\refstepcounter{question}\arabic{question}.~ }
\renewcommand{\thequestion}{\#\arabic{question}}

% sub-question lists
\usepackage{enumitem}
\setenumerate[1]{label=\alph*)}
\setlist{itemsep=1em,after=\vspace{1ex}}

% inline answers
\definecolor{answers}{HTML}{C0C0C0}
\newcommand{\ans}[1]{%
\ifdefined\Student
    \leavevmode\phantom{~~\textcolor{answers}{#1}}
\else
    ~~\textcolor{answers}{#1}
\fi}

% longer answers [optional height]
\newsavebox{\ansbox}
\newenvironment{answer}[1][4em]{
\nopagebreak
\begin{lrbox}{\ansbox}
\begin{minipage}[t][#1]{\linewidth}
\color{answers}
}{
\end{minipage}
\end{lrbox}
\ifdefined\Student
    \phantom{\usebox{\ansbox}}%
\else
    \usebox{\ansbox}%
\fi}


\begin{document}

\maketitle

Internally, the library class \java{java.lang.String} stores an array of characters.
It also provides a variety of useful methods for comparing, manipulating, and searching text in general.

\guide{
  \item Explain how characters and strings are represented in memory.
  \item Predict the output of string methods, including boundary cases.
  \item Recognize common mistakes when working with the String API.
}{
  \item Working with all team members to reach consensus on hard questions. (Teamwork)
}{
TODO

Report out on Question 3 and make connections to previous activities about strings and arrays. Note that when drawing strings in the future, students will not need to use this level of detail. Point out that methods from the String class make it much easier to manipulate character arrays.

Question 11 introduces the concept of \texttt{this} as an implicit argument. It may help to show students an incorrect static way of designing these methods, for example: \java{String.charAt(str, 8)}. Both parameters (\texttt{str} and \texttt{8}) are necessary, but the object-oriented way is simpler: \texttt{str.charAt(8)} where \texttt{str} becomes \texttt{this}.
}

\model{Character Arrays}

The primitive type \java{char} is used to store a single \emph{character}, which can be a letter, a number, or a symbol.
In contrast, the reference type \java{java.lang.String} represents a sequence of characters.

\begin{quote}
\begin{javalst}
char letter;           String word;
letter = 'a';          word = "food";
\end{javalst}
\end{quote}

Internally, a string is stored as an \emph{array} of characters.
Each character is indexed by a position starting at zero:

\begin{quote}
\begin{tabular}{cccc}
\hline
\multicolumn{1}{|c|}{\java{'f'}} &
\multicolumn{1}{ c|}{\java{'o'}} &
\multicolumn{1}{ c|}{\java{'o'}} &
\multicolumn{1}{ c|}{\java{'d'}} \\
\hline
\fs 0 & \fs 1 & \fs 2 & \fs 3 \\
\end{tabular}
\end{quote}


\quest{15 min}


\Q How is the syntax of character literals and string literals different?

\begin{answer}[3em]
\end{answer}


\Q Why can you use the \java{String} class in Java programs without having to import it first?

\begin{answer}[3em]
\end{answer}


\Q What is the index of \java{'d'} in the string above?
In general, what is the index of the last character of a string?

\begin{answer}[3em]
\end{answer}


\Q Sketch the underlying array for the string \java{"hello world"} (with indexes as shown above).

\begin{answer}
\end{answer}


\Q What is the \emph{value} of a \java{char} variable? What is the \emph{value} of a \java{String} variable?

\begin{answer}
\end{answer}


\Q Draw a memory diagram for the given code. Each variable should be a name next to a box containing its \emph{value}.

\begin{javalst}
String str;
str = "Hi!";

char let;
let = 'X';

short num;
num = -17;

int foo;
foo = num;

String hmm;
hmm = str;
\end{javalst}


\Q Recall that the == operator compares the value of two variables. What does it mean for two \java{char} variables to be ==? What does it mean for two \java{String} variables to be ==?

\begin{answer}
\end{answer}


\Q To compare strings (and other objects), you should use either the equals or compareTo method. Predict the output of the following code.

\begin{javalst}
String name1 = "Mark";    String name2 = "Ma" + "rk";    String name3 = "Mary";

// compare name1 and name2
if (name1 == name2) {
    System.out.println("name1 and name2 are identical");
} else {
    System.out.println("name1 and name2 are NOT identical");
}
// compare "Mark" and "Mark"
if (name1.equals(name2)) {
    System.out.println("name1 and name2 are equal");
} else {
    System.out.println("name1 and name2 are NOT equal");
}
// compare "Mark" and "Mary"
if (name1.equals(name3)) {
    System.out.println("name1 and name3 are equal");
} else {
    System.out.println("name1 and name3 are NOT equal");
}
\end{javalst}

\model{String Methods}
% Based on Model 2 of Activity 5 - Strings by Helen Hu

\begin{tabularx}{\linewidth}{|p{128pt}|p{72pt}|X|}
\hline
\tr Method & \tr Returns & \tr Description \\
\hline
\java{charAt(int)}          & \java{char}
  & Returns the char value at the specified index of \java{this} string. \\
%\hline
%\java{equals(Object)}       & \java{boolean}
%  & Compares \java{this} string to the specified object. \\
\hline
\java{indexOf(String)}      & \java{int}
  & Returns the index within \java{this} string of the first occurrence of the specified substring. \\
\hline
\java{length()}             & \java{int}
  & Returns the length of \java{this} string. \\
\hline
\java{substring(int, int)}  & \java{String}
  & Returns a new string that is a substring of \java{this} string (from \java{beginIndex} to \java{endIndex - 1}). \\
\hline
\java{toUpperCase()}        & \java{String}
  & Returns a copy of \java{this} string with all the characters converted to upper case. \\
\hline
\end{tabularx}

\vspace{1em}

Each method listed above is non-\java{static}.
That is, they have an \emph{implicit} parameter named \java{this} that is passed automatically.
(Note: There are many other \java{String} methods not listed above.)


\quest{20 min}


\Q \label{strAPI}
If \java{str} contains the string \java{"hello world"}, then what is the return value of the following method calls?

\begin{multicols}{2}
\begin{enumerate}
\item \java{str.charAt(8)} \ans{\chr{r}}
%\item \java{str.equals("hello")} \ans{\tt false}
\item \java{str.indexOf("wo")} \ans{6}
\item \java{str.length()} \ans{11}
\item \java{str.substring(4, 7)} \ans{\str{o w}}
\item \java{str.toUpperCase()} \ans{\str{HELLO WORLD}}
\end{enumerate}
\end{multicols}


\Q Explain what precedes the \java{.} (dot) operator in the expressions above.
What does it have to do with the keyword \java{this} in \ref{\currfilename}?

\begin{answer}
The variable containing the string precedes the dot operator. It is passed as the implicit {\tt this} parameter to the non-{\tt static} method.
\end{answer}


\comment{\normalfont

To call a \java{static} method, you write \emph{ClassName.methodName},
for example: ~\java{Math.abs(-5)}

\vspace{1ex}
To call a non-\java{static} method, you write \emph{objectName.methodName},
for example: ~\java{str.charAt(8)}

\vspace{1ex}
Methods can be designed either way. Most \java{String} methods are non-\java{static}, because that makes the code easier to read.
%, because they associate the method with the object itself.
From the \java{charAt} method's point of view, ~\java{str} is ~\java{this} object.

\begin{center}
\begin{tabular}{c|c}
\tr Static (\java{str} passed explicitly) &
\tr Non-Static (\java{str} passed implicitly) \\
\hline
\java{String.charAt(str, 8)  // wrong}
& \java{str.charAt(8)} \\
\end{tabular}
\end{center}

} % end comment


\Q \label{howmany}
How many arguments does each method call in \ref{strAPI} have?
(Hint: None of them have zero.)

\begin{multicols}{2}
\begin{enumerate}
\item \ans{2}
%\item \ans{2}
\item \ans{2}
\item \ans{1}
\item \ans{3}
\item \ans{1}
\end{enumerate}
\end{multicols}


\Q To compare strings, you must use either the \java{String.equals} or \java{String.compareTo} method.
Predict the output of the following code.

\begin{javalst}
String name1 = new String("Mark");
String name2 = new String("Mark");
\end{javalst}

\begin{javalst}
// compare name1 and name2
if (name1 == name2) {
    System.out.println("name1 and name2 are identical");
} else {
    System.out.println("name1 and name2 are NOT identical");
}
\end{javalst}

\vspace{-2ex}
\begin{answer}[1em]
\tt \hspace{1em} name1 and name2 are NOT identical
\end{answer}

\begin{javalst}
// compare "Mark" and "Mark"
if (name1.equals(name2)) {
    System.out.println("name1 and name2 are equal");
} else {
    System.out.println("name1 and name2 are NOT equal");
}
\end{javalst}

\vspace{-2ex}
\begin{answer}[1em]
\tt \hspace{1em} name1 and name2 are equal
\end{answer}


\Q What is the difference between \emph{identical} and \emph{equal} in the previous question?

\begin{answer}
``Identical'' means the variables refer to the same \java{String} object.
``Equal'' means the strings have the same characters, in the \java{Arrays.equals} sense.
\end{answer}


\Q \label{stringMatch}
Discuss the \java{stringMatch} problem on the next page.
What three \java{String} methods will you need to solve it?
(If you have time during the activity, complete the method.)

\begin{answer}
\java{String.length} for the loop, and \java{String.substring} and \java{String.equals} for the comparison.
\end{answer}


\Q \label{stringYak}
Discuss the \java{stringYak} problem on the next page.
What two \java{String} methods will you need to solve it?
(If you have time during the activity, complete the method.)

\begin{answer}
\java{String.length} for the loop, and \java{String.charAt} to identify \chr{y} and \chr{k}.
\end{answer}


\newpage

[CodingBat] Given two strings, return the number of positions where they contain the same substring of length two. So \java{"xxcaazz"} and \java{"xxbaaz"} yields 3, since the \java{"xx"}, \java{"aa"}, and \java{"az"} substrings appear in the same place in both strings.

\medskip
\begin{javalst}
public static int stringMatch(String a, String b) {
\end{javalst}

\vspace{-1em}
\begin{answer}[18em]
\begin{javaans}
    // figure which string is shorter
    int len = Math.min(a.length(), b.length());
    int count = 0;

    // look at both substrings starting at i
    for (int i = 0; i < len - 1; i++) {
        String aSub = a.substring(i, i + 2);
        String bSub = b.substring(i, i + 2);
        if (aSub.equals(bSub)) {
            count++;
        }
    }

    return count;
\end{javaans}
\end{answer}

\begin{javalst}
}
\end{javalst}

\vfill

[CodingBat] Suppose the string \java{"yak"} is unlucky. Given a string, return a version where all the \java{"yak"} are removed, but the \java{'a'} can be any character. The \java{"yak"} strings will not overlap.

\bigskip

\java{stringYak("yakpak")} $\rightarrow$ \java{"pak"}

\java{stringYak("pikyik")} $\rightarrow$ \java{"pik"}

\java{stringYak("yak123ya")} $\rightarrow$ \java{"123ya"}

\medskip
\begin{javalst}
public static String stringYak(String str) {
\end{javalst}

\vspace{-1em}
\begin{answer}[16em]
\begin{javaans}
    String result = "";

    for (int i = 0; i < str.length(); i++) {
        if (i + 2 < str.length() && str.charAt(i) == 'y'
                                 && str.charAt(i + 2) == 'k') {
            i = i + 2;
        } else {
            result += str.charAt(i);
        }
    }

    return result;
\end{javaans}
\end{answer}

\begin{javalst}
}
\end{javalst}

\newpage

\model{Common Mistakes}
% Based on Model 3 of Activity 5 - Strings by Helen Hu

Use \textit{JShell} to run each of the code snippets.
For Program B, do not provide any user input (just press Enter).
Record the output of the last statement in the space below the table.

\newsavebox{\programA}
\begin{lrbox}{\programA}
\begin{javalst}
String greeting = "hello world";
greeting.toUpperCase();
System.out.println(greeting);
\end{javalst}
\end{lrbox}

\newsavebox{\programB}
\begin{lrbox}{\programB}
\begin{javalst}
Scanner in = new Scanner(System.in);
String line = in.nextLine();
char letter = line.charAt(1);
\end{javalst}
\end{lrbox}

\vspace{-1ex}
\begin{table}[h!]
\begin{tabularx}{\linewidth}{|X|X|}
\hline
\tr Program A & \tr Program B \\
\hline
\usebox{\programA} & \usebox{\programB} \\
\hline
\multicolumn{1}{c}{\ans[17em]{\tt hello world}} &
\multicolumn{1}{c}{\ans[17em]{\tt StringIndexOutOfBoundsException}} \\[-3ex]
\end{tabularx}
\end{table}


\quest{15 min}


\Q What is the logic error you see when you run Program A?

\begin{answer}
The \str{hello world} greeting is still lowercase.
\end{answer}


\Q \label{errorA}
In Program A, what is returned by the string method?
What happens to the return value?

\begin{answer}
A new string is returned, but its value is not saved anywhere.
\end{answer}


\Q Describe two different ways you can fix the logic error in \ref{errorA}.

\begin{answer}
\vspace{-1ex}
\begin{javaans}
greeting = greeting.toUpperCase();   // or just combine the two lines
System.out.println(greeting);        System.out.println(greeting.toUpperCase());
\end{javaans}
\end{answer}


%\Q In general, what happens to \java{this} string when calling the methods in \ref{string-methods.tex}?
%
%\begin{answer}
%It is passed as the implicit first argument, but nothing happens to {\tt this} because strings cannot be modified.
%\end{answer}


\Q \label{errorB}
In what cases will Program B throw an exception? What is the error message displayed?

\begin{answer}
If the user enters a blank line, then there will be no character at position 1. It will show the out of bounds exception and display the stack trace.
\end{answer}


\Q Describe two different ways you can fix the run-time error in \ref{errorB}.

\begin{answer}
1. Check if the string is long enough: ~{\tt if (line.length() >= 2)} \\
2. Keep asking for input until valid: ~{\tt do \{~...~\} while (line.length() < 2);}
\end{answer}


%\Q Explain why these errors occur based on what you learned in \ref{string-chararray.tex} and \ref{string-methods.tex}.
%
%\begin{answer}
%The \java{toUpperCase} method returns a copy of {\tt this} string; it doesn't actually change the original string.
%The \java{charAt} method reads the internal character array; if the index is out of bounds, you'll get an exception.
%\end{answer}


\newpage

\Q To compare strings, you must use either the \java{String.equals} or \java{String.compareTo} method.
Predict the output of the following code.

\begin{javalst}
String name1 = new String("Mark");
String name2 = new String("Mark");
\end{javalst}

\begin{javalst}
// compare name1 and name2
if (name1 == name2) {
    System.out.println("name1 and name2 are identical");
} else {
    System.out.println("name1 and name2 are NOT identical");
}
\end{javalst}

\vspace{-2ex}
\begin{answer}[1em]
\tt \hspace{1em} name1 and name2 are NOT identical
\end{answer}

\begin{javalst}
// compare "Mark" and "Mark"
if (name1.equals(name2)) {
    System.out.println("name1 and name2 are equal");
} else {
    System.out.println("name1 and name2 are NOT equal");
}
\end{javalst}

\vspace{-2ex}
\begin{answer}[1em]
\tt \hspace{1em} name1 and name2 are equal
\end{answer}


\Q \label{key3}
What is the difference between \emph{identical} and \emph{equal} in the previous question?

\begin{answer}[4em]
``Identical'' means the variables refer to the same \java{String} object.
``Equal'' means the strings have the same characters, in the \java{Arrays.equals} sense.
\end{answer}


\end{document}
