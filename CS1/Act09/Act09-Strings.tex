\documentclass[12pt]{article}

\title{Activity 9: Object-Oriented}
\author{Chris Mayfield and Helen Hu}
\date{July 2017}

%\ProvidesPackage{cspogil}

% fonts
\usepackage[utf8]{inputenc}
\usepackage[T1]{fontenc}
\usepackage{mathpazo}

% spacing
\usepackage[margin=2cm]{geometry}
\renewcommand{\arraystretch}{1.4}
\setlength{\parindent}{0pt}

% orphans and widows
\clubpenalty=10000
\widowpenalty=10000
\pagestyle{empty}

% figures and tables
\usepackage{graphicx}
\usepackage{multicol}
\usepackage{tabularx}

% fixed-width columns
\usepackage{array}
\newcolumntype{L}[1]{>{\raggedright\let\newline\\\arraybackslash\hspace{0pt}}m{#1}}
\newcolumntype{C}[1]{>{\centering\let\newline\\\arraybackslash\hspace{0pt}}m{#1}}
\newcolumntype{R}[1]{>{\raggedleft\let\newline\\\arraybackslash\hspace{0pt}}m{#1}}

% include paths
\makeatletter
\def\input@path{{Models/}{../../Models/}}
\graphicspath{{Models/}{../../Models/}}
\makeatother

% colors
\usepackage[svgnames,table]{xcolor}
\definecolor{bgcolor}{HTML}{FAFAFA}
\definecolor{comment}{HTML}{007C00}
\definecolor{keyword}{HTML}{0000FF}
\definecolor{strings}{HTML}{B20000}

% table headers
\newcommand{\tr}{\bf\cellcolor{Yellow!10}}

% syntax highlighting
\usepackage{textcomp}
\usepackage{listings}
\lstset{
    basicstyle=\ttfamily,
    backgroundcolor=\color{bgcolor},
    numberstyle=\scriptsize\color{comment},
    commentstyle=\color{comment},
    keywordstyle=\color{keyword},
    stringstyle=\color{strings},
    columns=fullflexible,
    keepspaces=true,
    showstringspaces=false,
    upquote=true
}

% code environments
\newcommand{\java}[1]{\lstinline[language=java]{#1}}%[
\lstnewenvironment{javalst}{\lstset{language=java,backgroundcolor=}}{}
\lstnewenvironment{javabox}{\lstset{language=java,frame=single,numbers=left}\quote}{\endquote}

% PDF properties
\usepackage[pdftex]{hyperref}
\urlstyle{same}
\makeatletter
\hypersetup{
  pdftitle={\@title},
  pdfauthor={\@author},
  pdfsubject={\@date},
  pdfkeywords={},
  bookmarksopen=false,
  colorlinks=true,
  citecolor=black,
  filecolor=black,
  linkcolor=black,
  urlcolor=blue
}
\makeatother

% titles
\makeatletter
\renewcommand{\maketitle}{\begin{center}\LARGE\@title\end{center}}
\makeatother

% boxes
\newcommand{\emptybox}[1][10em]{
\vspace{1em}
\begin{tabularx}{\linewidth}{|X|}
\hline\\[#1]\hline
\end{tabularx}}

% models
\newcommand{\model}[1]{\section{#1}\nopagebreak}
\renewcommand{\thesection}{Model~\arabic{section}}

% questions
\newcommand{\quest}[1]{\subsection*{Questions~ (#1)}}
\newcounter{question}
\newcommand{\Q}{\vspace{1em}\refstepcounter{question}\arabic{question}.~ }
\renewcommand{\thequestion}{\#\arabic{question}}

% sub-question lists
\usepackage{enumitem}
\setenumerate[1]{label=\alph*)}
\setlist{itemsep=1em,after=\vspace{1ex}}

% inline answers
\definecolor{answers}{HTML}{C0C0C0}
\newcommand{\ans}[1]{%
\ifdefined\Student
    \phantom{~~\textcolor{answers}{#1}}
\else
    ~~\textcolor{answers}{#1}
\fi}

% longer answers [optional height]
\newsavebox{\ansbox}
\newenvironment{answer}[1][4em]{
\nopagebreak
\begin{lrbox}{\ansbox}
\begin{minipage}[t][#1]{\linewidth}
\color{answers}
}{
\end{minipage}
\end{lrbox}
\ifdefined\Student
    \phantom{\usebox{\ansbox}}%
\else
    \usebox{\ansbox}%
\fi}


\begin{document}

\maketitle

Internally, the library class \java{java.lang.String} stores an array of characters.
It also provides a variety of useful methods for comparing, manipulating, and searching text in general.

\guide{
  \item Explain how characters and strings are represented in memory.
  \item Predict the output of string methods, including boundary cases.
  \item Recognize common mistakes when working with the String API.
}{
  \item Working with all team members to reach consensus on hard questions. (Teamwork)
}{
TODO

Report out on Question 3 and make connections to previous activities about strings and arrays. Note that when drawing strings in the future, students will not need to use this level of detail. Point out that methods from the String class make it much easier to manipulate character arrays.

Question 11 introduces the concept of \texttt{this} as an implicit argument. It may help to show students an incorrect static way of designing these methods, for example: \java{String.charAt(str, 8)}. Both parameters (\texttt{str} and \texttt{8}) are necessary, but the object-oriented way is simpler: \texttt{str.charAt(8)} where \texttt{str} becomes \texttt{this}.
}

\model{Character Arrays}
% Based on Model 1 of Activity 5 - Strings by Helen Hu

The primitive type \java{char} is used to store a single character, which can be a letter, a number, or a symbol.
In contrast, the reference type \java{String} \emph{encapsulates} an array of characters.

\begin{minipage}[t]{150pt}
\begin{javalst}
char letter;
letter = 'A';

char[] array;
array = new char[]
        {'c', 'a', 't'};

String word;
word = "dog";
\end{javalst}
\end{minipage}
\hfill
\begin{minipage}[t]{345pt}
\null
\includegraphics[width=\linewidth]{string1.pdf}
\null
\end{minipage}


\quest{15 min}


\Q How is the syntax of character literals (like \chr{A}) and string literals (like \str{dog}) different?

\begin{answer}
Character literals use single quotes, and strings use double quotes.
\end{answer}


\Q What is the index of \java{'d'} in the string above?
What is the index of \java{'g'}?
%In general, what is the index of the last character of a string?

\begin{answer}
The index of \chr{d} is 0, the index of \chr{g} is 2.

\medskip
In general, the last character is at {\tt length - 1}.
\end{answer}


\Q What is the {\it value} of a \java{char} variable (i.e., stored in the variable's memory)?
What is the {\it value} of an array variable?
What is the {\it value} of a \java{String} variable?

\begin{answer}
Since {\tt char}s are primitive, the value of a {\tt char} variable is the character itself.
Arrays and strings are reference types, so their variables contain memory locations.
\end{answer}


\Q \label{encap}
Based on the diagram, what does it mean for an object to encapsulate data?
How do you access the \java{char[]} inside of a \java{String} object?

\begin{answer}
The data is stored inside the class and accessed via methods.
\java{String} (the class) makes it more convenient to deal with character arrays.
\end{answer}


%\Q Why can you use the \java{String} class in Java programs without having to import it first?

%\begin{answer}
%It's in the \java{java.lang} package, which gets imported automatically.
%\end{answer}


\Q \label{diagram}
Draw a memory diagram for the given code.
(List the name of each variable next to a box containing its value.)

\hspace*{1em}
\begin{minipage}{0.25\linewidth}

\begin{javalst}
String str;
str = "Hi!";

char let;
let = 'X';

int num;
num = -1;

double foo;
foo = num;

String hmm;
hmm = str;
\end{javalst}

\end{minipage}
\begin{minipage}{0.65\linewidth}

\note{\includegraphics[height=3in]{string2.pdf}}

\end{minipage}
\vspace{1ex}


\Q \label{key1}
Recall that the \java{==} operator compares the {\it value} of two variables.
What does it mean for two \java{char} variables to be \java{==}?
What does it mean for two \java{String} variables to be \java{==}?

\begin{answer}
Two {\tt char} variables are \java{==} if they have the same character.
In contrast, two {\tt String} variables are \java{==} if they refer to the same \java{String} object.
\end{answer}


\Q How could you determine whether two character arrays have the same contents?
In other words, how does the \java{String.equals} method work internally?

\begin{answer}
Using a loop, you compare each character in the first array to the corresponding character in the second array. If any two characters don't match, or the arrays have different lengths, then they are not equal.
\end{answer}

\model{String Methods}
% Based on Model 2 of Activity 5 - Strings by Helen Hu

\begin{tabularx}{\linewidth}{|p{128pt}|p{72pt}|X|}
\hline
\tr Method & \tr Returns & \tr Description \\
\hline
\java{charAt(int)}          & \java{char}
  & Returns the char value at the specified index of \java{this} string. \\
%\hline
%\java{equals(Object)}       & \java{boolean}
%  & Compares \java{this} string to the specified object. \\
\hline
\java{indexOf(String)}      & \java{int}
  & Returns the index within \java{this} string of the first occurrence of the specified substring. \\
\hline
\java{length()}             & \java{int}
  & Returns the length of \java{this} string. \\
\hline
\java{substring(int, int)}  & \java{String}
  & Returns a new string that is a substring of \java{this} string (from \java{beginIndex} to \java{endIndex - 1}). \\
\hline
\java{toUpperCase()}        & \java{String}
  & Returns a copy of \java{this} string with all the characters converted to upper case. \\
\hline
\end{tabularx}

\vspace{1em}

Each method listed above is non-\java{static}.
That is, they have an \emph{implicit} parameter named \java{this} that is passed automatically.
(Note: There are many other \java{String} methods not listed above.)


\quest{20 min}


\Q \label{strAPI}
If \java{str} contains the string \java{"hello world"}, then what is the return value of the following method calls?

\begin{multicols}{2}
\begin{enumerate}
\item \java{str.charAt(8)} \ans{\chr{r}}
%\item \java{str.equals("hello")} \ans{\tt false}
\item \java{str.indexOf("wo")} \ans{6}
\item \java{str.length()} \ans{11}
\item \java{str.substring(4, 7)} \ans{\str{o w}}
\item \java{str.toUpperCase()} \ans{\str{HELLO WORLD}}
\end{enumerate}
\end{multicols}


\Q Explain what precedes the \java{.} (dot) operator in the expressions above.
What does it have to do with the keyword \java{this} in \ref{\currfilename}?

\begin{answer}
The variable containing the string precedes the dot operator. It is passed as the implicit {\tt this} parameter to the non-{\tt static} method.
\end{answer}


\comment{\normalfont

To call a \java{static} method, you write \emph{ClassName.methodName},
for example: ~\java{Math.abs(-5)}

\vspace{1ex}
To call a non-\java{static} method, you write \emph{objectName.methodName},
for example: ~\java{str.charAt(8)}

\vspace{1ex}
Methods can be designed either way. Most \java{String} methods are non-\java{static}, because that makes the code easier to read.
%, because they associate the method with the object itself.
From the \java{charAt} method's point of view, ~\java{str} is ~\java{this} object.

\begin{center}
\begin{tabular}{c|c}
\tr Static (\java{str} passed explicitly) &
\tr Non-Static (\java{str} passed implicitly) \\
\hline
\java{String.charAt(str, 8)  // wrong}
& \java{str.charAt(8)} \\
\end{tabular}
\end{center}

} % end comment


\Q \label{howmany}
How many arguments does each method call in \ref{strAPI} have?
(Hint: None of them have zero.)

\begin{multicols}{2}
\begin{enumerate}
\item \ans{2}
%\item \ans{2}
\item \ans{2}
\item \ans{1}
\item \ans{3}
\item \ans{1}
\end{enumerate}
\end{multicols}


\Q To compare strings, you must use either the \java{String.equals} or \java{String.compareTo} method.
Predict the output of the following code.

\begin{javalst}
String name1 = new String("Mark");
String name2 = new String("Mark");
\end{javalst}

\begin{javalst}
// compare name1 and name2
if (name1 == name2) {
    System.out.println("name1 and name2 are identical");
} else {
    System.out.println("name1 and name2 are NOT identical");
}
\end{javalst}

\vspace{-2ex}
\begin{answer}[1em]
\tt \hspace{1em} name1 and name2 are NOT identical
\end{answer}

\begin{javalst}
// compare "Mark" and "Mark"
if (name1.equals(name2)) {
    System.out.println("name1 and name2 are equal");
} else {
    System.out.println("name1 and name2 are NOT equal");
}
\end{javalst}

\vspace{-2ex}
\begin{answer}[1em]
\tt \hspace{1em} name1 and name2 are equal
\end{answer}


\Q What is the difference between \emph{identical} and \emph{equal} in the previous question?

\begin{answer}
``Identical'' means the variables refer to the same \java{String} object.
``Equal'' means the strings have the same characters, in the \java{Arrays.equals} sense.
\end{answer}


\Q \label{stringMatch}
Discuss the \java{stringMatch} problem on the next page.
What three \java{String} methods will you need to solve it?
(If you have time during the activity, complete the method.)

\begin{answer}
\java{String.length} for the loop, and \java{String.substring} and \java{String.equals} for the comparison.
\end{answer}


\Q \label{stringYak}
Discuss the \java{stringYak} problem on the next page.
What two \java{String} methods will you need to solve it?
(If you have time during the activity, complete the method.)

\begin{answer}
\java{String.length} for the loop, and \java{String.charAt} to identify \chr{y} and \chr{k}.
\end{answer}


\newpage

[CodingBat] Given two strings, return the number of positions where they contain the same substring of length two. So \java{"xxcaazz"} and \java{"xxbaaz"} yields 3, since the \java{"xx"}, \java{"aa"}, and \java{"az"} substrings appear in the same place in both strings.

\medskip
\begin{javalst}
public static int stringMatch(String a, String b) {
\end{javalst}

\vspace{-1em}
\begin{answer}[18em]
\begin{javaans}
    // figure which string is shorter
    int len = Math.min(a.length(), b.length());
    int count = 0;

    // look at both substrings starting at i
    for (int i = 0; i < len - 1; i++) {
        String aSub = a.substring(i, i + 2);
        String bSub = b.substring(i, i + 2);
        if (aSub.equals(bSub)) {
            count++;
        }
    }

    return count;
\end{javaans}
\end{answer}

\begin{javalst}
}
\end{javalst}

\vfill

[CodingBat] Suppose the string \java{"yak"} is unlucky. Given a string, return a version where all the \java{"yak"} are removed, but the \java{'a'} can be any character. The \java{"yak"} strings will not overlap.

\bigskip

\java{stringYak("yakpak")} $\rightarrow$ \java{"pak"}

\java{stringYak("pikyik")} $\rightarrow$ \java{"pik"}

\java{stringYak("yak123ya")} $\rightarrow$ \java{"123ya"}

\medskip
\begin{javalst}
public static String stringYak(String str) {
\end{javalst}

\vspace{-1em}
\begin{answer}[16em]
\begin{javaans}
    String result = "";

    for (int i = 0; i < str.length(); i++) {
        if (i + 2 < str.length() && str.charAt(i) == 'y'
                                 && str.charAt(i + 2) == 'k') {
            i = i + 2;
        } else {
            result += str.charAt(i);
        }
    }

    return result;
\end{javaans}
\end{answer}

\begin{javalst}
}
\end{javalst}

\newpage

%% Based on Model 3 of Activity 5 - Strings by Helen Hu

\model{Common Mistakes}

\newsavebox{\programA}
\begin{lrbox}{\programA}
\begin{javalst}
String greeting = "hello world";
greeting.toUpperCase();
System.out.println(greeting);

\end{javalst}
\end{lrbox}

\newsavebox{\programB}
\begin{lrbox}{\programB}
\begin{javalst}
Scanner in = new Scanner(System.in);
String line = in.nextLine();
char letter = line.charAt(1);
System.out.println(letter);
\end{javalst}
\end{lrbox}

\vspace{-1ex}
\begin{table}[h!]
\begin{tabularx}{\linewidth}{|X|X|}
\hline
\tr Program A & \tr Program B \\
\hline
\usebox{\programA} & \usebox{\programB} \\
\hline
\multicolumn{1}{c}{\ans{\tt hello world}} &
\multicolumn{1}{c}{\ans{\tt StringIndexOutOfBoundsException}} \\[-3ex]
\end{tabularx}
\end{table}


\quest{15 min}


\Q Write the output of each program in the space under the table above.
What is the logic error you see when you run Program A?

\begin{answer}
Hello world is still lowercase.
\end{answer}


\Q \label{errorA} In Program A, what is returned by the string method?
What happens to the return value?

\begin{answer}
A new string is returned, but its value is not saved anywhere.
\end{answer}


\Q List two different ways you can fix the logic error in \ref{errorA}.

\begin{answer}
\end{answer}


\Q In general, what happens to \java{this} string when calling the methods in \ref{CS1/string-methods}?

\begin{answer}
\end{answer}


\Q \label{errorB} In what cases will Program B crash? What is the error message displayed?

\begin{answer}
\end{answer}


\Q List two different ways you can fix the run-time error in \ref{errorB}.

\begin{answer}[3em]
\end{answer}


%\Q Solve the \java{stringYak} problem (see attached page) using \java{charAt}.


\newpage

[CodingBat] Given two strings, return the number of positions where they contain the same substring of length two. So \java{"xxcaazz"} and \java{"xxbaaz"} yields 3, since the \java{"xx"}, \java{"aa"}, and \java{"az"} substrings appear in the same place in both strings.

\medskip
\begin{javalst}
public int stringMatch(String a, String b) {
\end{javalst}

\vspace{-1em}
\begin{answer}[18em]
\begin{javaans}
    // figure which string is shorter
    int len = Math.min(a.length(), b.length());
    int count = 0;

    // Look at both substrings starting at i
    for (int i = 0; i < len - 1; i++) {
        String aSub = a.substring(i, i + 2);
        String bSub = b.substring(i, i + 2);
        if (aSub.equals(bSub)) {
            count++;
        }
    }

    return count;
\end{javaans}
\end{answer}

\begin{javalst}
}
\end{javalst}

\vfill

[CodingBat] Suppose the string \java{"yak"} is unlucky. Given a string, return a version where all the \java{"yak"} are removed, but the \java{'a'} can be any character. The \java{"yak"} strings will not overlap.

\bigskip

\java{stringYak("yakpak")} $\rightarrow$ \java{"pak"}

\java{stringYak("pakyak")} $\rightarrow$ \java{"pak"}

\java{stringYak("yak123ya")} $\rightarrow$ \java{"123ya"}

\medskip
\begin{javalst}
public String stringYak(String str) {
\end{javalst}

\vspace{-1em}
\begin{answer}[16em]
\begin{javaans}
    String result = "";

    for (int i = 0; i < str.length(); i++) {
        if (i + 2 < str.length() && str.charAt(i) == 'y'
                                 && str.charAt(i + 2) == 'k') {
            i = i + 2;
        } else {
            result += str.charAt(i);
        }
    }

    return result;
\end{javaans}
\end{answer}

\begin{javalst}
}
\end{javalst}


\end{document}
