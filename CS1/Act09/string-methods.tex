\model{String Methods}
% Based on Model 2 of Activity 5 - Strings by Helen Hu

\begin{tabularx}{\linewidth}{|p{128pt}|p{72pt}|X|}
\hline
\tr Method & \tr Returns & \tr Description \\
\hline
\java{charAt(int)}          & \java{char}
  & Returns the char value at the specified index of \java{this} string. \\
%\hline
%\java{equals(Object)}       & \java{boolean}
%  & Compares \java{this} string to the specified object. \\
\hline
\java{indexOf(String)}      & \java{int}
  & Returns the index within \java{this} string of the first occurrence of the specified substring. \\
\hline
\java{length()}             & \java{int}
  & Returns the length of \java{this} string. \\
\hline
\java{substring(int, int)}  & \java{String}
  & Returns a new string that is a substring of \java{this} string (from \java{beginIndex} to \java{endIndex - 1}). \\
\hline
\java{toUpperCase()}        & \java{String}
  & Returns a copy of \java{this} string with all the characters converted to upper case. \\
\hline
\end{tabularx}

\vspace{1em}

Each method listed above is non-\java{static}.
That is, they have an \emph{implicit} parameter named \java{this} that is passed automatically.
(Note: There are many other \java{String} methods not listed above.)


\quest{20 min}


\Q \label{strAPI}
If \java{str} contains the string \java{"hello world"}, then what is the return value of the following method calls?

\begin{multicols}{2}
\begin{enumerate}
\item \java{str.charAt(8)} \ans{\chr{r}}
%\item \java{str.equals("hello")} \ans{\tt false}
\item \java{str.indexOf("wo")} \ans{6}
\item \java{str.length()} \ans{11}
\item \java{str.substring(4, 7)} \ans{\str{o w}}
\item \java{str.toUpperCase()} \ans{\str{HELLO WORLD}}
\end{enumerate}
\end{multicols}


\Q Explain what precedes the \java{.} (dot) operator in the expressions above.
What does it have to do with the keyword \java{this} in \ref{\currfilename}?

\begin{answer}
The variable containing the string precedes the dot operator. It is passed as the implicit {\tt this} parameter to the non-{\tt static} method.
\end{answer}


\comment{\normalfont

To call a \java{static} method, you write \emph{ClassName.methodName},
for example: ~\java{Math.abs(-5);}

\vspace{1ex}
To call a non-\java{static} method, you write \emph{objectName.methodName},
for example: ~\java{str.length();}

\vspace{1ex}
Non-\java{static} methods make the code easier to read, because they associate the method with the object itself.
From the \java{charAt} method's point of view, ~\java{str} becomes \java{this} object.

\begin{center}
\begin{tabular}{c|c}
\tr Static (\java{str} passed explicitly) &
\tr Non-Static (\java{str} passed implicitly) \\
\hline
\java{String.charAt(str, 8)  // wrong}
& \java{str.charAt(8)} \\
\end{tabular}
\end{center}

} % end comment


\Q \label{howmany}
How many arguments does each method call in \ref{strAPI} have?
(Hint: None of them have zero.)

\begin{multicols}{2}
\begin{enumerate}
\item \ans{2}
%\item \ans{2}
\item \ans{2}
\item \ans{1}
\item \ans{3}
\item \ans{1}
\end{enumerate}
\end{multicols}


\Q To compare strings, you must use either the \java{String.equals} or \java{String.compareTo} method.
Predict the output of the following code.

\begin{javalst}
String name1 = "Mark";
String name2 = "Ma" + "rk";
String name3 = "Mary";

// compare name1 and name2
if (name1 == name2) {
    System.out.println("name1 and name2 are identical");
} else {
    System.out.println("name1 and name2 are NOT identical");
}
\end{javalst}

\vspace{-3ex}
\begin{answer}[1em]
\tt \hspace{1em} name1 and name2 are identical ~ \normalfont Note: compiler optimization
\end{answer}

\begin{javalst}
// compare "Mark" and "Mark"
if (name1.equals(name2)) {
    System.out.println("name1 and name2 are equal");
} else {
    System.out.println("name1 and name2 are NOT equal");
}
\end{javalst}

\vspace{-3ex}
\begin{answer}[1em]
\tt \hspace{1em} name1 and name2 are equal
\end{answer}

\begin{javalst}
// compare "Mark" and "Mary"
if (name1.equals(name3)) {
    System.out.println("name1 and name3 are equal");
} else {
    System.out.println("name1 and name3 are NOT equal");
}
\end{javalst}

\vspace{-3ex}
\begin{answer}[1em]
\tt \hspace{1em} name1 and name3 are NOT equal
\end{answer}


\Q What is the difference between \emph{identical} and \emph{equal} in the previous question?

\begin{answer}
``Identical'' means the variables refer to the same \java{String} object.
``Equal'' means the strings have the same characters, in the \java{Arrays.equals} sense.
\end{answer}


\Q \label{stringMatch}
Discuss the \java{stringMatch} problem on the next page.
What three \java{String} methods will you need to solve it?
(If you have time during the activity, complete the method.)

\begin{answer}
\java{String.length} for the loop, and \java{String.substring} and \java{String.equals} for the comparison.
\end{answer}


\Q \label{stringYak}
Discuss the \java{stringYak} problem on the next page.
What two \java{String} methods will you need to solve it?
(If you have time during the activity, complete the method.)

\begin{answer}
\java{String.length} for the loop, and \java{String.charAt} to identify \chr{y} and \chr{k}.
\end{answer}


\newpage

[CodingBat] Given two strings, return the number of positions where they contain the same substring of length two. So \java{"xxcaazz"} and \java{"xxbaaz"} yields 3, since the \java{"xx"}, \java{"aa"}, and \java{"az"} substrings appear in the same place in both strings.

\medskip
\begin{javalst}
public int stringMatch(String a, String b) {
\end{javalst}

\vspace{-1em}
\begin{answer}[18em]
\begin{javaans}
    // figure which string is shorter
    int len = Math.min(a.length(), b.length());
    int count = 0;

    // look at both substrings starting at i
    for (int i = 0; i < len - 1; i++) {
        String aSub = a.substring(i, i + 2);
        String bSub = b.substring(i, i + 2);
        if (aSub.equals(bSub)) {
            count++;
        }
    }

    return count;
\end{javaans}
\end{answer}

\begin{javalst}
}
\end{javalst}

\vfill

[CodingBat] Suppose the string \java{"yak"} is unlucky. Given a string, return a version where all the \java{"yak"} are removed, but the \java{'a'} can be any character. The \java{"yak"} strings will not overlap.

\bigskip

\java{stringYak("yakpak")} $\rightarrow$ \java{"pak"}

\java{stringYak("pakyak")} $\rightarrow$ \java{"pak"}

\java{stringYak("yak123ya")} $\rightarrow$ \java{"123ya"}

\medskip
\begin{javalst}
public String stringYak(String str) {
\end{javalst}

\vspace{-1em}
\begin{answer}[16em]
\begin{javaans}
    String result = "";

    for (int i = 0; i < str.length(); i++) {
        if (i + 2 < str.length() && str.charAt(i) == 'y'
                                 && str.charAt(i + 2) == 'k') {
            i = i + 2;
        } else {
            result += str.charAt(i);
        }
    }

    return result;
\end{javaans}
\end{answer}

\begin{javalst}
}
\end{javalst}

\newpage
