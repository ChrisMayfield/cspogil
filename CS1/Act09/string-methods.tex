\model{Non-Static Methods}
% Based on Model 2 of Activity 5 - Strings by Helen Hu

\begin{tabularx}{\linewidth}{|p{128pt}|p{72pt}|X|}
\hline
\tr Method & \tr Returns & \tr Description \\
\hline
\java{charAt(int)}          & \java{char}
  & Returns the char value at the specified index of \java{this} string. \\
%\hline
%\java{equals(Object)}       & \java{boolean}
%  & Compares \java{this} string to the specified object. \\
\hline
\java{indexOf(String)}      & \java{int}
  & Returns the index within \java{this} string of the first occurrence of the specified substring. \\
\hline
\java{length()}             & \java{int}
  & Returns the length of \java{this} string. \\
\hline
\java{substring(int, int)}  & \java{String}
  & Returns a new string that is a substring of \java{this} string (from \java{beginIndex} to \java{endIndex - 1}). \\
\hline
\java{toUpperCase()}        & \java{String}
  & Returns a copy of \java{this} string with all the characters converted to upper case. \\
\hline
\end{tabularx}

\vspace{1em}

Each method listed above is non-\java{static}.
That is, they have an \emph{implicit parameter} named \java{this} that is passed automatically.
(Note: There are many other \java{String} methods not listed above.)


\quest{15 min}


\Q \label{strAPI}
If the variable \java{str} references the string \java{"hello world"}, then what is the return value of the following method calls?

\begin{multicols}{2}
\begin{enumerate}
\item \java{str.charAt(8)} \ans[4em]{\chr{r}}
%\item \java{str.equals("hello")} \ans[4em]{\tt false}
\item \java{str.indexOf("wo")} \ans[4em]{6}
\item \java{str.length()} \ans[4em]{11}
\item \java{str.substring(4, 7)} \ans[4em]{\str{o w}}
\item \java{str.toUpperCase()} \ans[8em]{\str{HELLO WORLD}}
\end{enumerate}
\end{multicols}


\Q Explain what precedes the dot operator (\java{.}) in the expressions above.
What does it have to do with the keyword \java{this} in \ref{\currfilename}?

\begin{answer}
The variable referencing the string precedes the dot operator. It is passed as the implicit {\tt this} parameter to the non-{\tt static} method.
\end{answer}


\Q \label{howmany}
How many arguments does each method call in \ref{strAPI} have?
(Hint: None of them have zero; don't forget to count the implicit argument.)

\begin{multicols}{2}
\begin{enumerate}
\item \ans[3em]{2}
%\item \ans[3em]{2}
\item \ans[3em]{2}
\item \ans[3em]{1}
\item \ans[3em]{3}
\item \ans[3em]{1}
\end{enumerate}
\end{multicols}


\comment{\normalfont

To call a \java{static} method, you write {\it ClassName.methodName},
for example: ~\java{Math.abs(-5)}

\vspace{1ex}
To call a non-\java{static} method, you write {\it objectName.methodName},
for example: ~\java{str.charAt(8)}

\vspace{1ex}
Methods can be designed either way. Most \java{String} methods are non-\java{static}, because that makes the code easier to read.
%, because they associate the method with the object itself.
%From the \java{charAt} method's point of view, ~\java{str} is ~\java{this} object.

\begin{center}
\begin{tabular}{c|c}
\tr Static (\java{str} passed explicitly) &
\tr Non-Static (\java{str} passed implicitly) \\
\hline
\java{String.charAt(str, 8)  // wrong}
& \java{str.charAt(8)} \\
\end{tabular}
\end{center}

} % end comment


\Q \label{nonstatic}
Label each method below as \java{static} or non-\java{static} (based on how it is called).

\begin{multicols}{2}
\begin{enumerate}

\item \java{color.indexOf("RED")}
\ans[5em]{non-static}

\item \java{String.format("\%3d", x)}
\ans[5em]{static}

\item \java{title.substring(0, 10)}
\ans[5em]{non-static}

\item \java{String.valueOf(3.14)}
\ans[5em]{static}

\item \java{name.charAt(0)}
\ans[5em]{non-static}

\end{enumerate}
\end{multicols}


\Q \label{key2}
Consider the following statement and compiler error.
Why is it impossible to call \java{charAt} this way?
How would you correct the error?

\begin{quote}
\begin{javalst}
char c = String.charAt(0);
\end{javalst}
\textit{Error:~} non-static method charAt(int) cannot be referenced from a static context
\end{quote}

\begin{answer}
\jans{charAt} is a non-static method; you can't call it using the \jans{String} class.
You need to call it using an object like \jans{name}.
Otherwise, the compiler won't know which string you want.
\end{answer}


%\Q Consider the following method and compiler error.
%Explain how this situation is different from the previous question.
%How would you correct the error?
%
%\begin{quote}
%\begin{javalst}
%public static void printAddress() {
%    // show the memory location of this object
%    System.out.println(this);
%}
%\end{javalst}
%\textit{Error:~} non-static variable \java{this} cannot be referenced from a static context
%\end{quote}
%
%\begin{answer}[3em]
%Static methods do not have an implicit \jans{this} parameter.
%In order to work, the \jans{printAddress} method must not be static.
%\end{answer}


\Q For each method in \ref{nonstatic}, what object is referenced by \java{this}?
(Write N/A if \java{this} is not passed to the method.)

\begin{multicols}{2}
\begin{enumerate}

\item \ans[5em]{color}

\item \ans[5em]{N/A}

\item \ans[5em]{title}

\item \ans[5em]{N/A}

\item \ans[5em]{name}

\end{enumerate}
\end{multicols}
