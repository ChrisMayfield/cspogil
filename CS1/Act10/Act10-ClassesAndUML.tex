\documentclass[12pt]{article}

\title{Activity 10: Classes and UML}
\author{Chris Mayfield and Helen Hu}
\date{July 2017}

%\ProvidesPackage{cspogil}

% fonts
\usepackage[utf8]{inputenc}
\usepackage[T1]{fontenc}
\usepackage{mathpazo}

% spacing
\usepackage[margin=2cm]{geometry}
\renewcommand{\arraystretch}{1.4}
\setlength{\parindent}{0pt}

% orphans and widows
\clubpenalty=10000
\widowpenalty=10000
\pagestyle{empty}

% figures and tables
\usepackage{graphicx}
\usepackage{multicol}
\usepackage{tabularx}

% fixed-width columns
\usepackage{array}
\newcolumntype{L}[1]{>{\raggedright\let\newline\\\arraybackslash\hspace{0pt}}m{#1}}
\newcolumntype{C}[1]{>{\centering\let\newline\\\arraybackslash\hspace{0pt}}m{#1}}
\newcolumntype{R}[1]{>{\raggedleft\let\newline\\\arraybackslash\hspace{0pt}}m{#1}}

% include paths
\makeatletter
\def\input@path{{Models/}{../../Models/}}
\graphicspath{{Models/}{../../Models/}}
\makeatother

% colors
\usepackage[svgnames,table]{xcolor}
\definecolor{bgcolor}{HTML}{FAFAFA}
\definecolor{comment}{HTML}{007C00}
\definecolor{keyword}{HTML}{0000FF}
\definecolor{strings}{HTML}{B20000}

% table headers
\newcommand{\tr}{\bf\cellcolor{Yellow!10}}

% syntax highlighting
\usepackage{textcomp}
\usepackage{listings}
\lstset{
    basicstyle=\ttfamily,
    backgroundcolor=\color{bgcolor},
    numberstyle=\scriptsize\color{comment},
    commentstyle=\color{comment},
    keywordstyle=\color{keyword},
    stringstyle=\color{strings},
    columns=fullflexible,
    keepspaces=true,
    showstringspaces=false,
    upquote=true
}

% code environments
\newcommand{\java}[1]{\lstinline[language=java]{#1}}%[
\lstnewenvironment{javalst}{\lstset{language=java,backgroundcolor=}}{}
\lstnewenvironment{javabox}{\lstset{language=java,frame=single,numbers=left}\quote}{\endquote}

% PDF properties
\usepackage[pdftex]{hyperref}
\urlstyle{same}
\makeatletter
\hypersetup{
  pdftitle={\@title},
  pdfauthor={\@author},
  pdfsubject={\@date},
  pdfkeywords={},
  bookmarksopen=false,
  colorlinks=true,
  citecolor=black,
  filecolor=black,
  linkcolor=black,
  urlcolor=blue
}
\makeatother

% titles
\makeatletter
\renewcommand{\maketitle}{\begin{center}\LARGE\@title\end{center}}
\makeatother

% boxes
\newcommand{\emptybox}[1][10em]{
\vspace{1em}
\begin{tabularx}{\linewidth}{|X|}
\hline\\[#1]\hline
\end{tabularx}}

% models
\newcommand{\model}[1]{\section{#1}\nopagebreak}
\renewcommand{\thesection}{Model~\arabic{section}}

% questions
\newcommand{\quest}[1]{\subsection*{Questions~ (#1)}}
\newcounter{question}
\newcommand{\Q}{\vspace{1em}\refstepcounter{question}\arabic{question}.~ }
\renewcommand{\thequestion}{\#\arabic{question}}

% sub-question lists
\usepackage{enumitem}
\setenumerate[1]{label=\alph*)}
\setlist{itemsep=1em,after=\vspace{1ex}}

% inline answers
\definecolor{answers}{HTML}{C0C0C0}
\newcommand{\ans}[1]{%
\ifdefined\Student
    \phantom{~~\textcolor{answers}{#1}}
\else
    ~~\textcolor{answers}{#1}
\fi}

% longer answers [optional height]
\newsavebox{\ansbox}
\newenvironment{answer}[1][4em]{
\nopagebreak
\begin{lrbox}{\ansbox}
\begin{minipage}[t][#1]{\linewidth}
\color{answers}
}{
\end{minipage}
\end{lrbox}
\ifdefined\Student
    \phantom{\usebox{\ansbox}}%
\else
    \usebox{\ansbox}%
\fi}


\begin{document}

\maketitle

The \java{String} class provides methods for working with text.
The \java{Random} class provides methods for generating random numbers.
In this activity, you'll learn how to make your own classes that represent everyday objects.

\guide{
  \item Define the terms: attribute, method, constructor, instance.
  \item Implement non-static methods based on a UML diagram.
  \item Distinguish between static, instance, parameter, and local variables.
}{
  \item Writing method signatures exactly as specified in a UML diagram. (Information Processing)
}{
\ref{die-class.tex} is primarily about introducing object-oriented vocabulary.
As you walk around, keep an eye on \ref{dievar}.
Some students may incorrectly write \texttt{int lucky}, and if that happens, have them re-read the text in \ref{die-class.tex}.
For reporting out, have presenters of neighboring teams compare their answers and ask questions as needed.

\ref{circle-class.tex} asks students to implement a \texttt{Circle} class, one method at a time.
After they complete \ref{circmain}, show Circle.java on the projector and step through the code beginning from \texttt{main}.
It may be helpful to use \href{http://pythontutor.com/java.html}{Java Tutor} to illustrate how \texttt{Circle} objects are created on the heap.

\ref{variable-scope.tex} presents a slightly different circle class named \texttt{SwapCircle}.
Have students work through the first questions as quickly as possible so they can spend time on \ref{predict}.
You'll need about ten minutes at the end of class to walk students through the solution, again using Java Tutor or similar tool.
}

\model{The Die Class}

The following class represents an individual ``die'' in a game of dice.
The diagram on the right is a graphical summary of the \emph{attributes} (variables) and \emph{methods} of the class.

\vspace{1em}
\begin{quote}
\hfill\includegraphics{Die.pdf}
\vspace*{-84pt}

\begin{javalst}
/**
 * Simulates a die object.
 */
public class Die {

    private int face;

    /**
     * Constructs a die with face value 1.
     */
    public Die() {
        this.face = 1;
    }

    /**
     * @return current face value of the die
     */
    public int getFace() {
        return this.face;
    }

    /**
     * Simulates rolling the die.
     * 
     * @return new face value of the die
     */
    public int roll() {
        this.face = (int) (Math.random() * 6) + 1;
        return this.face;
    }

}
\end{javalst}

\vspace*{-72pt}
% https://commons.wikimedia.org/wiki/File:2-Dice-Icon.svg
\hfill\includegraphics[width=2in]{dice.png}
\end{quote}


\newpage
\quest{10 min}


\Q Consider the \java{Die} class:

\begin{enumerate}
\item What are the attributes?
\ans[10em]{face}

\item What are the methods?
\ans[10em]{Die, getFace, roll}
\end{enumerate}


\Q In the class diagram (on the upper right):

\begin{enumerate}
\item What do the \java{+} and \java{-} symbols represent?
\ans[18em]{\jans{+} means \jans{public} ~ \jans{-} means \jans{private}}

\item What does the \java{:} represent?
\ans[18em]{the data type of the attribute/method}
\end{enumerate}


\Q Open the provided \textit{Die.java} and run the program several times.
Then answer the following questions about the \java{main} method:
\begin{enumerate}
\item What is the data type of \java{d1} and \java{d2}?
\ans[8em]{Die}

\item What are the initial values of the dice?
\ans[8em]{d1 = 1, d2 = 1}

\item What method changed the dice values?
\ans[8em]{roll}
\end{enumerate}


\Q \label{dievar}
Write a statement that declares and initializes a \java{Die} variable named \java{lucky}.
%(Do not initialize the variable.)

\begin{answer}[3em]
\tt Die lucky = new Die();
\end{answer}


%\Q Write a statement that creates a new \java{Die} and assigns it to \java{lucky}.
%(Do not declare the variable.)
%
%\begin{answer}[2em]
%\tt lucky = new Die();
%\end{answer}


\Q When you create an object, Java invokes a \emph{constructor}.
This method has no return type and has the same name as the class itself.
What does the \java{Die()} constructor do?

\begin{answer}
It initializes the \java{face} attribute to 1.
(Without this constructor, the default value would be 0, which is invalid for dice.)
\end{answer}


\Q Notice how the \java{roll} method refers to \java{this.face}, yet that variable is not declared in the method. What does the \java{roll} method change, in terms of the \java{Die} object?

\begin{answer}[3em]
It updates the value of the \java{face} attribute.
\end{answer}


%\Q What is the purpose of the \java{getFace} method?
%Show how you would use it in a \java{main} method of another class.
%
%\begin{answer}[3em]
%In a {\tt main} method, you would do something like: {\tt System.out.println(lucky.getFace());}
%\end{answer}

\model{The Circle Class}

Unified Modeling Language (UML) provides a way of graphically illustrating a class's design, independent of the programming language.

\begin{center}
\includegraphics{Circle.pdf}
\end{center}


\quest{15 min}


\Q Consider the \java{Circle} class:

\begin{enumerate}
\item How many attributes does it have?
\ans[3em]{1}

\item How many methods does it have?
\ans[3em]{6}
\end{enumerate}


\Q Based on \ref{die-class.tex} and \ref{\currfilename}, what is typically \java{public} and what is typically \java{private}?

\begin{answer}[2em]
Methods are typically public,
and attributes are typically private.
\end{answer}


%\Q How would you declare a variable named \java{unit} that is a \java{Circle} object?
%
%\begin{answer}[1em]
%\tt Circle unit;
%\end{answer}


%\Q How would you initialize \java{unit} to be a new \java{Circle} with a radius of 1.0?
%
%\begin{answer}[1em]
%\tt unit = new Circle(1.0);
%\end{answer}


\comment{The following questions will have you implement the \java{Circle} class exactly as shown in the UML diagram above.
Do not worry about writing Javadoc comments for this activity.}


\Q Write the code that declares the \java{radius} attribute.
An outline of \textit{Circle.java} is provided below for context.

\begin{javalst}
public class Circle {
\end{javalst}

{\tt ~~~}\ans{\tt private double radius;}

\medskip

\begin{javalst}
    // constructor goes here
    
    // other methods go here
}
\end{javalst}


\Q Write the code for the \java{Circle} constructor.
Notice that, in contrast to \ref{die-class.tex}, the \java{Circle} constructor has a parameter.
Assign the parameter \java{radius} to the attribute \java{this.radius}.

\begin{answer}[5em]
\begin{javaans}
public Circle(double radius) {
    this.radius = radius;
}
\end{javaans}
\end{answer}


\Q Write the code for \java{getRadius}.
(Refer to \ref{die-class.tex} for an example.)

\begin{answer}[5em]
\begin{javaans}
public double getRadius() {
    return this.radius;
}
\end{javaans}
\end{answer}


\Q Write the code for \java{setRadius}.
Like the constructor, it should assign the parameter to the corresponding attribute.

\begin{answer}[5em]
\begin{javaans}
public void setRadius(double radius) {
    this.radius = radius;
}
\end{javaans}
\end{answer}


\Q Write the code for \java{area}.
The area of a circle is $\pi r^2$.

\begin{answer}[5em]
\begin{javaans}
public double area() {
    return Math.PI * radius * radius;
}
\end{javaans}
\end{answer}


\Q Write the code for \java{circumference}.
The circumference of a circle is $2 \pi r$.

\begin{answer}[5em]
\begin{javaans}
public double circumference() {
    return 2.0 * Math.PI * radius;
}
\end{javaans}
\end{answer}


\Q \label{circmain}
Write a \java{main} method that creates a \java{Circle} object with a radius of 2.0 and displays its area and circumference (using \java{println}).

\begin{answer}[8em]
\begin{javaans}
public static void main(String[] args) {
    Circle circle = new Circle(2);
    System.out.println(circle.area());
    System.out.println(circle.circumference());
}
\end{javaans}
\end{answer}

\newpage
\model{Variable Scope}
% based on Model 1 of "Improved Scope" activity by Helen Hu

As a team, review and discuss the provided \textit{SwapCircle.java} and \textit{SwapDriver.java} source files.
Then identify the \emph{scope} of each variable (i.e., where it can base used) based on the table below.

\begin{center}
\small
\begin{tabular}{|L{95pt}|L{125pt}|L{115pt}|L{105pt}|}
\hline
\tr &
\tr \textbf{Where declared?} &
\tr \textbf{Where used?} &
\tr \textbf{Example} \\
\hline
\textbf{static variables} \par (``class variables'') &
declared outside of all methods (typically at the start of the class) &
anywhere in the class (or in other classes if also \java{public}) &
\java{circleCount} in the \java{SwapCircle} class \\
\hline
\textbf{instance variables} \par (``attributes'') &
declared outside of all methods (typically after any static variables) &
any non-static method (or in static methods when another object has been created) &
\java{radius} in the \java{SwapCircle} class \\
\hline
\textbf{parameters} &
declared inside the ()'s of a method signature &
anywhere within the method where they are declared &
\java{radius} in the \java{SwapCircle} constructor \\
\hline
\textbf{local variables} &
declared inside a method (or inside another block of code, like a \java{for} loop) &
anywhere within the method or code block after they are declared &
\java{temp} in the \java{swapInts} method \\
\hline
\end{tabular}
\end{center}


\quest{20 min}

\setlength{\defaultwidth}{38em}


\Q Identify one static variable from the \java{SwapCircle} class.

\begin{enumerate}
\item What is the name and purpose of the variable?
\\ \ans{\java{circleCount} -- tracks the number of \java{SwapCircle} objects that have been created}

\item What is the scope of the variable?
\\ \ans{{\tt private static} -- it can be used anywhere within the \java{SwapCircle} class only}

\item What is one example of somewhere it cannot be used?
\\ \ans{\tt SwapDriver.main}
\end{enumerate}


\Q Identify one instance variable from the \java{SwapCircle} class.

\begin{enumerate}
\item What is the name and purpose of the variable?
\\ \ans{\java{radius} -- stores the radius of {\tt this} SwapCircle}

\item What is the scope of the variable?
\\ \ans{{\tt private} and non-{\tt static} -- it can only be used in \java{SwapCircle} in non-static contexts}

\item What is one example of somewhere it cannot be used?
\\ \ans{\tt SwapDriver.main}
\end{enumerate}


\Q Identify one parameter from the \java{SwapCircle} class.

\begin{enumerate}
\item What is the name and purpose of the variable? \\[2pt]
\ans{Possible answers include: \java{radius} (in the constructor), \java{x} and \java{y}, \java{c1} and \java{c2}}

\item What is the scope of the variable? \\[2pt]
\ans{The variable exists throughout the entire method (but not other methods).}

\item Where can the variable not be used? \\[2pt]
\ans{It can't be used in other methods, e.g., you can't refer to \java{x} in \java{swapCircles}.}
\end{enumerate}


\Q Identify one local variable from the \java{SwapCircle} class.

\begin{enumerate}
\item What is the name and purpose of the variable? \\[2pt]
\ans{Possible answers include: \java{temp} and \java{r} (both used for swapping values)}

\item What is the scope of the variable? \\[2pt]
\ans{The variable exists from when it's declared until the end of the method.}

\item Where can the variable not be used? \\[2pt]
\ans{It cannot be used in other methods and/or before it has been declared.}
\end{enumerate}


%\Q \label{predict}
%In the space below, predict the result of the \java{SwapDriver} program.
%How might \java{swapInts} and \java{swapCircles} behave differently?
%
%\begin{answer}
%Answers will vary.
%The \java{swapInts} method is using primitive values, and the \java{swapCircles} method is using references.
%\end{answer}


\Q \label{summary}
Run the \java{SwapDriver} program and summarize what you learn based on the output.

\begin{answer}
Answers will vary.
The primitive integers were not swapped, but the attributes of the \java{Circle} objects were.
The \jans{static circleCount} kept track of how many objects were created.
\end{answer}


\Q Notice that \java{getRadius} returns \java{this.radius} (from \java{this} object).
In contrast, \java{getCircleCount} does not use the keyword \java{this}.
Why not?

\begin{answer}
getCircleCount is a static method, so there is no object.
If you try to use \jans{this}, you will get a compiler error that says, ``non-static variable this cannot be referenced from a static context''.
\end{answer}


\Q Identify an example of where an instance variable is used within a static method.

\begin{enumerate}
\item In which method does this occur? \\[2pt]
\ans{\java{radius} is used in the \java{swapCircles} method}

\item Why is the method able to access these instance variables, even though they are private? \\[2pt]
\ans{\java{swapCircles} belongs to the \java{SwapCircle} class}

\item Explain how this method is not a violation of the rule that instance variables cannot be accessed inside a static method. \\[1ex]
\ans{You can't use {\tt this}.radius in a {\tt static} method, but it's okay to use \java{c1.radius}}
\end{enumerate}


\end{document}
