% comment out for student version
\ifdefined\Student\relax\else\def\Teacher{}\fi

\documentclass[12pt]{article}

\title{Designing Classes}
\author{Chris Mayfield and Helen Hu}
\date{Summer 2021}

%\ProvidesPackage{cspogil}

% fonts
\usepackage[utf8]{inputenc}
\usepackage[T1]{fontenc}
\usepackage{mathpazo}

% spacing
\usepackage[margin=2cm]{geometry}
\renewcommand{\arraystretch}{1.4}
\setlength{\parindent}{0pt}

% orphans and widows
\clubpenalty=10000
\widowpenalty=10000
\pagestyle{empty}

% figures and tables
\usepackage{graphicx}
\usepackage{multicol}
\usepackage{tabularx}

% fixed-width columns
\usepackage{array}
\newcolumntype{L}[1]{>{\raggedright\let\newline\\\arraybackslash\hspace{0pt}}m{#1}}
\newcolumntype{C}[1]{>{\centering\let\newline\\\arraybackslash\hspace{0pt}}m{#1}}
\newcolumntype{R}[1]{>{\raggedleft\let\newline\\\arraybackslash\hspace{0pt}}m{#1}}

% include paths
\makeatletter
\def\input@path{{Models/}{../../Models/}}
\graphicspath{{Models/}{../../Models/}}
\makeatother

% colors
\usepackage[svgnames,table]{xcolor}
\definecolor{bgcolor}{HTML}{FAFAFA}
\definecolor{comment}{HTML}{007C00}
\definecolor{keyword}{HTML}{0000FF}
\definecolor{strings}{HTML}{B20000}

% table headers
\newcommand{\tr}{\bf\cellcolor{Yellow!10}}

% syntax highlighting
\usepackage{textcomp}
\usepackage{listings}
\lstset{
    basicstyle=\ttfamily,
    backgroundcolor=\color{bgcolor},
    numberstyle=\scriptsize\color{comment},
    commentstyle=\color{comment},
    keywordstyle=\color{keyword},
    stringstyle=\color{strings},
    columns=fullflexible,
    keepspaces=true,
    showstringspaces=false,
    upquote=true
}

% code environments
\newcommand{\java}[1]{\lstinline[language=java]{#1}}%[
\lstnewenvironment{javalst}{\lstset{language=java,backgroundcolor=}}{}
\lstnewenvironment{javabox}{\lstset{language=java,frame=single,numbers=left}\quote}{\endquote}

% PDF properties
\usepackage[pdftex]{hyperref}
\urlstyle{same}
\makeatletter
\hypersetup{
  pdftitle={\@title},
  pdfauthor={\@author},
  pdfsubject={\@date},
  pdfkeywords={},
  bookmarksopen=false,
  colorlinks=true,
  citecolor=black,
  filecolor=black,
  linkcolor=black,
  urlcolor=blue
}
\makeatother

% titles
\makeatletter
\renewcommand{\maketitle}{\begin{center}\LARGE\@title\end{center}}
\makeatother

% boxes
\newcommand{\emptybox}[1][10em]{
\vspace{1em}
\begin{tabularx}{\linewidth}{|X|}
\hline\\[#1]\hline
\end{tabularx}}

% models
\newcommand{\model}[1]{\section{#1}\nopagebreak}
\renewcommand{\thesection}{Model~\arabic{section}}

% questions
\newcommand{\quest}[1]{\subsection*{Questions~ (#1)}}
\newcounter{question}
\newcommand{\Q}{\vspace{1em}\refstepcounter{question}\arabic{question}.~ }
\renewcommand{\thequestion}{\#\arabic{question}}

% sub-question lists
\usepackage{enumitem}
\setenumerate[1]{label=\alph*)}
\setlist{itemsep=1em,after=\vspace{1ex}}

% inline answers
\definecolor{answers}{HTML}{C0C0C0}
\newcommand{\ans}[1]{%
\ifdefined\Student
    \phantom{~~\textcolor{answers}{#1}}
\else
    ~~\textcolor{answers}{#1}
\fi}

% longer answers [optional height]
\newsavebox{\ansbox}
\newenvironment{answer}[1][4em]{
\nopagebreak
\begin{lrbox}{\ansbox}
\begin{minipage}[t][#1]{\linewidth}
\color{answers}
}{
\end{minipage}
\end{lrbox}
\ifdefined\Student
    \phantom{\usebox{\ansbox}}%
\else
    \usebox{\ansbox}%
\fi}


\begin{document}

\maketitle

The \java{String} class provides methods for working with text.
The \java{Random} class provides methods for generating random numbers.
In this activity, you'll learn how to make your own classes that represent everyday objects.

\rolenames

\guide{
%  \item Discuss benefits of POGIL for student learning.
  \item Explain the purpose of constructor, accessor, and mutator methods.
  \item Implement the equals and toString methods for a given class design.
  \item Design a new class (UML diagram) based on a general description.
}{
  \item Identifying attributes and data types that model a real-world object. (Problem Solving)
}{
%The meta activities are designed to reestablish student buy-in later in the course.
%They are an opportunity to explain why you are using POGIL and how it benefits students.
%The figures and sample solutions are from Moog (2014), found in Chapter 8 of \href{https://dx.doi.org/10.7936/K7PN93HC}{Integrating Cognitive Science with Innovative Teaching in STEM Disciplines}.

\ref{common-methods.tex} begins with questions about constructors, getters, and setters.
Report out as soon as the majority teams have finished.
You may find it helpful to project the source code for {\it Color.java} while students work through the questions.
Reinforce the concept of immutable objects, and point out that the \java{String} class is designed this way.

The questions at the beginning of \ref{equals-tostring.tex} require students to understand the source code of \java{equals} and \java{toString}.
If this is the first time they have seen language features like \java{Object}, \texttt{instanceof}, and \java{String.format}, you might want to give a 5-minute lecture (but avoid giving away the answers).

When reporting out \ref{credit-card.tex}, have presenters write their designs on the board.
Compare the trade-offs of their different designs.
For example, to store credit card numbers some teams may use strings, others may use arrays of integers, and some may use a \texttt{long} variable (\texttt{int} won't work because of the range).

Key questions: \ref{key1}, \ref{key2}, \ref{key3}

Source files: \src{Act11}{Color.java}, \src{Act11}{Point.java}
}

%\section*{Meta Activity: POGIL Research 1}

\textit{Process-Oriented Guided Inquiry Learning} (see \href{https://pogil.org/}{pogil.org}) is a student-centered, group-learning instructional strategy and philosophy developed through research on how students learn best.
The following figure is from a peer-reviewed research article about POGIL:

\begin{center}
% image from Chapter 8 of http://dx.doi.org/10.7936/K7PN93HC
\includegraphics[width=0.60\linewidth]{pogil-grades.png}
\end{center}


\quest{7.5 min}


\Q Based on the figure above:

\begin{enumerate}
\item How many years were considered? \ans[2em]{8}
\item How many instructors were involved? \ans[2em]{3}
\item How many students were involved? \ans[2em]{905}
\end{enumerate}
\vspace{-1em}


\Q Which grade categories improved after the instructors switched to POGIL?

\begin{answer}
D/W/F decreased from 22\% to 10\%. This result is arguably the most important.
B's increased from 33\% to 40\%, and A's increased from 19\% to 24\%.
In other words, more students passed, and most students' grades increased.
\end{answer}


\Q What does the research suggest about POGIL's impact on student success?

\begin{answer}[5em]
``Students in courses employing a POGIL instructional strategy achieved a significantly higher success rate (defined as receiving an A, B, or C in the course, as compared to a D, F or withdrawal) than students who had been taught by the same instructors in previous years using a more traditional lecture-oriented approach.'' (Moog 2014)
\end{answer}

%\section*{Meta Activity: POGIL Research 2}

Many research studies have been conducted about POGIL.
In the following example, students were given an unannounced quiz on the first day of class (based on the previous semester).
About half of them had been taught in lecture sections, and half in POGIL sections.

\begin{center}
% image from Chapter 8 of http://dx.doi.org/10.7936/K7PN93HC
\includegraphics[width=0.625\linewidth]{pogil-prequiz.png}
\end{center}


\quest{7.5 min}


\Q How large were the classes in the previous semester?

\begin{answer}[2em]
More than 150 students per section.
\end{answer}

\vspace{1ex}


\Q About what percentage of the \ldots

\begin{multicols}{2}
\begin{enumerate}

\item Lecture students scored below 60? \ans[2em]{75}
%\item Lecture students scored above 60? \ans[2em]{25}
\item Lecture students scored above 80? \ans[2em]{2}

\item POGIL students scored below 60? \ans[2em]{38}
%\item POGIL students scored above 60? \ans[2em]{62}
\item POGIL students scored above 80? \ans[2em]{30}

\end{enumerate}
\end{multicols}


\Q What does the research suggest about students' retention of knowledge?

\begin{answer}[5em]
Students in courses that use POGIL are more likely to remember what they learn:
``In contrast, fewer than a quarter of the students from the POGIL section scored below 50 percent on this quiz, and about 30 percent of the students scored above 80 percent, with over one-fifth of the POGIL students scoring above 90 percent.'' (Moog 2014)
\end{answer}


\model{Common Methods}

Classes are often used to represent abstract data types, such as \java{Color} or \java{Point}:

\begin{center}
\includegraphics{Color.pdf}  % immutable
~~~~~
\includegraphics{Point.pdf}  % mutable
\end{center}

As shown in the UML diagrams, classes generally include the following kinds of methods (in addition to others):

\begin{itemize}[itemsep=0pt]
\item \textbf{constructor} methods that initialize new objects
\item \textbf{accessor} methods (getters) that return attributes
\item \textbf{mutator} methods (setters) that modify attributes
%\item \textbf{object} methods such as \java{equals} and \java{toString}
%\item \textbf{utility} methods which are generally static
\end{itemize}

%Note: The \java{Color} class does not have getters and setters.


\quest{15 min}


\Q Identify the constructors for the \java{Color} class.
What is the difference between them?
%What arguments do they take?

\begin{answer}[3em]
There are two constructors: one that takes no parameters (the default constructor), and one that takes three integers for the RGB values.
\end{answer}


\Q What kind of constructor does the \java{Point} class have that the \java{Color} class does not?
%Explain the purpose of such a constructor.

\begin{answer}[3em]
The \java{Point} class also has a copy constructor: one that ``copies'' the values of another object.
\end{answer}


\Q Identify an accessor method in the \java{Point} class.

\begin{enumerate}
\item What is the name of the method? \ans{{\tt getX} or {\tt getY}}
\item Which instance variable does it get? \ans{{\tt this.x} or {\tt this.y}}
\item What arguments does the method take? \ans{none}
\item What does the method return? \ans{The value of \java{x} or \java{y}}
\end{enumerate}


\Q Identify a mutator method in the \java{Point} class.

\begin{enumerate}
\item What is the name of the method? \ans{{\tt setX} or {\tt setY}}
\item Which instance variable does it set? \ans{{\tt this.x} or {\tt this.y}}
\item What arguments does the method take? \ans{The value of \java{x} or \java{y}}
\item What does the method return? \ans{nothing}
\end{enumerate}


\Q How would you define accessor methods for each attribute of the \java{Color} class?
Write your answer using UML syntax.

\begin{answer}[5em]
\begin{javaans}
+getRed(): int
+getGreen(): int
+getBlue(): int
\end{javaans}
\end{answer}


\Q How would you define mutator methods for each attribute of the \java{Color} class?
Write your answer using UML syntax.

\begin{answer}[5em]
\begin{javaans}
+setRed(red:int)
+setGreen(green:int)
+setBlue(blue:int)
\end{javaans}
\end{answer}


\Q \label{key1}
The \java{Color} class does not provide any accessors or mutators.
Instead, it provides methods that return new \texttt{Color} objects.
Why do you think the class was designed this way?

\begin{answer} [5em]
Other than the constructor, there are no methods that change the \texttt{red}, \texttt{green}, and \texttt{blue} values.
This design makes the class immutable, which means that objects can be reused.
The \java{String} class is also designed this way.
\end{answer}

\newpage
\model{Object Methods}

In addition to providing constructors, getters, and setters, classes often provide \java{equals} and \java{toString} methods.
These methods make it easier to work with objects of the class.

\vspace{1em}

As a team, review the provided \textit{Color.java} and \textit{Point.java} files.
Run each program to see how it works.
Then answer the following questions using the source code (don't just guess).


\quest{15 min}


\Q Based on the output of \textit{Color.java}, what is the value of each expression below?

\begin{javalst}
Color black = new Color();
Color other = new Color(0, 0, 0);
Color gold = new Color(255, 215, 0);
\end{javalst}

\begin{multicols}{2}
\setlength{\defaultwidth}{5em}
\begin{enumerate}[itemsep=1pt]
\item \java{black == other} \ans{false}
\item \java{black == gold} \ans{false}
\item \java{black.toString()} \ans{"\#000000"}
\item \java{black.equals(other)} \ans{true}
\item \java{black.equals(gold)} \ans{false}
\item \java{gold.toString()} \ans{"\#ffd700"}
\end{enumerate}
\end{multicols}


\Q What is the purpose of the \texttt{toString} method?

\begin{answer}
It returns a \texttt{String} representation of the \texttt{Color} (in HTML/CSS format).
The \java{toString} method makes it easier to examine and debug objects.
\end{answer}


\Q \label{expr}
Based on the output of \textit{Point.java}, what is the value of each expression below?
\begin{javalst}
Point p1 = new Point();
Point p2 = new Point(0, 0);
Point p3 = new Point(3, 3);
\end{javalst}

\begin{multicols}{2}
\setlength{\defaultwidth}{5em}
\begin{enumerate}[itemsep=1pt]
\item \java{p1 == p2} \ans{false}
\item \java{p1.toString()} \ans{"(0, 0)"}
\item \java{p3.toString()} \ans{"(3, 3)"}
\item \java{p1.equals(p2)} \ans{true}
\item \java{p1.equals("(0, 0)")} \ans{false}
\item \java{p3.equals("(3, 3)")} \ans{false}
\end{enumerate}
\end{multicols}


\Q \label{key2}
What is the purpose of the \texttt{equals} method?

\begin{answer}
It determines whether two objects have the same attribute values.
The \java{equals} method is useful for testing with \java{assertEquals}.
\end{answer}


\Q Examine {\it Point.java} again.
What is the purpose of the \java{if}-statement in the \texttt{equals} method?

\begin{answer}
Since \java{equals} can take any type of \java{Object}, you need to check if the argument is a \java{Color} or \java{Point} instance before using it as such.
\end{answer}


\Q How could you modify the \java{equals} method to cause both \ref{expr}e and \ref{expr}f to return \java{true}?

\begin{answer}[5em]
Change the last line to ~\jans{return this.toString().equals(obj);}

\vspace{1em}
You could instead add ~\texttt{if (obj instanceof String)}, but since the \texttt{String.equals} method takes an \texttt{Object}, there's no need to convert the \java{obj} parameter before calling \texttt{String.equals}.
\end{answer}

\vspace{1em}
\model{Credit Card}
% based on Model 2 of "Activity 10 - Class Design" by Helen Hu

Classes often represent objects in the real world.
In this section, you will design a new class that represents a \java{CreditCard} like the one below:

\begin{center}
% https://www.bankofamerica.com/credit-cards/
\includegraphics{credit-card.png}
\end{center}


\quest{15 min}


\Q Identify two or more attributes that would be necessary for the \java{CreditCard} class.
For each attribute, indicate what data type would be most appropriate.

\begin{answer}
Answers may include ~\verb|number:long|, ~\verb|expire:Date|, ~\verb|name:String|, ~\verb|code:int|, ~etc.
\end{answer}


\Q Using UML syntax, define two or more constructors for the \java{CreditCard} class.

\begin{answer}
\begin{javaans}
+CreditCard()
+CreditCard(number:long, name:String)
\end{javaans}
\end{answer}


\Q Define two or more accessor methods for the \java{CreditCard} class.
Include arguments and return values, using the same format as a UML diagram.

\begin{answer}[5em]
\begin{verbatim}
+getNumber(): long
+getExpire(): Date
+getName(): String
+getCode(): int
\end{verbatim}
\end{answer}


\Q Define two or more mutator methods for the \java{CreditCard} class.
Include arguments and return values, using the same format as a UML diagram.

\begin{answer}[5em]
\begin{verbatim}
+setNumber(number:long): void
+setExpire(expire:Date): void
+setName(name:String): void
+setCode(code:int): void
\end{verbatim}
\end{answer}


\Q \label{key3}
Describe how you would implement the \java{equals} method of the \java{CreditCard} class.

\begin{answer}
Two credit cards would be considered equal if they have the same account number, assuming there are no duplicates in the bank.
\end{answer}


\Q Describe how you would implement the \java{toString} method of the \java{CreditCard} class.

\begin{answer}
The \java{toString} would print the account number, expiration date, and cardholder's name, each separated by a comma.
\end{answer}


\Q When constructing (or updating) a \java{CreditCard} object, which arguments would you need to validate?
What are the valid ranges of values for each attribute?

\begin{answer}[5em]
The number should have 16 digits, dates need to have valid months and days, names should be at most 22 letters and not contain digits or other characters, code should be 3--4 digits, etc.
\end{answer}


\end{document}
