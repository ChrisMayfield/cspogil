\def\Teacher{}
\documentclass[12pt]{article}

\title{Activity 11: Designing Classes}
\author{Chris Mayfield and Helen Hu}
\date{July 2017}

%\ProvidesPackage{cspogil}

% fonts
\usepackage[utf8]{inputenc}
\usepackage[T1]{fontenc}
\usepackage{mathpazo}

% spacing
\usepackage[margin=2cm]{geometry}
\renewcommand{\arraystretch}{1.4}
\setlength{\parindent}{0pt}

% orphans and widows
\clubpenalty=10000
\widowpenalty=10000
\pagestyle{empty}

% figures and tables
\usepackage{graphicx}
\usepackage{multicol}
\usepackage{tabularx}

% fixed-width columns
\usepackage{array}
\newcolumntype{L}[1]{>{\raggedright\let\newline\\\arraybackslash\hspace{0pt}}m{#1}}
\newcolumntype{C}[1]{>{\centering\let\newline\\\arraybackslash\hspace{0pt}}m{#1}}
\newcolumntype{R}[1]{>{\raggedleft\let\newline\\\arraybackslash\hspace{0pt}}m{#1}}

% include paths
\makeatletter
\def\input@path{{Models/}{../../Models/}}
\graphicspath{{Models/}{../../Models/}}
\makeatother

% colors
\usepackage[svgnames,table]{xcolor}
\definecolor{bgcolor}{HTML}{FAFAFA}
\definecolor{comment}{HTML}{007C00}
\definecolor{keyword}{HTML}{0000FF}
\definecolor{strings}{HTML}{B20000}

% table headers
\newcommand{\tr}{\bf\cellcolor{Yellow!10}}

% syntax highlighting
\usepackage{textcomp}
\usepackage{listings}
\lstset{
    basicstyle=\ttfamily,
    backgroundcolor=\color{bgcolor},
    numberstyle=\scriptsize\color{comment},
    commentstyle=\color{comment},
    keywordstyle=\color{keyword},
    stringstyle=\color{strings},
    columns=fullflexible,
    keepspaces=true,
    showstringspaces=false,
    upquote=true
}

% code environments
\newcommand{\java}[1]{\lstinline[language=java]{#1}}%[
\lstnewenvironment{javalst}{\lstset{language=java,backgroundcolor=}}{}
\lstnewenvironment{javabox}{\lstset{language=java,frame=single,numbers=left}\quote}{\endquote}

% PDF properties
\usepackage[pdftex]{hyperref}
\urlstyle{same}
\makeatletter
\hypersetup{
  pdftitle={\@title},
  pdfauthor={\@author},
  pdfsubject={\@date},
  pdfkeywords={},
  bookmarksopen=false,
  colorlinks=true,
  citecolor=black,
  filecolor=black,
  linkcolor=black,
  urlcolor=blue
}
\makeatother

% titles
\makeatletter
\renewcommand{\maketitle}{\begin{center}\LARGE\@title\end{center}}
\makeatother

% boxes
\newcommand{\emptybox}[1][10em]{
\vspace{1em}
\begin{tabularx}{\linewidth}{|X|}
\hline\\[#1]\hline
\end{tabularx}}

% models
\newcommand{\model}[1]{\section{#1}\nopagebreak}
\renewcommand{\thesection}{Model~\arabic{section}}

% questions
\newcommand{\quest}[1]{\subsection*{Questions~ (#1)}}
\newcounter{question}
\newcommand{\Q}{\vspace{1em}\refstepcounter{question}\arabic{question}.~ }
\renewcommand{\thequestion}{\#\arabic{question}}

% sub-question lists
\usepackage{enumitem}
\setenumerate[1]{label=\alph*)}
\setlist{itemsep=1em,after=\vspace{1ex}}

% inline answers
\definecolor{answers}{HTML}{C0C0C0}
\newcommand{\ans}[1]{%
\ifdefined\Student
    \phantom{~~\textcolor{answers}{#1}}
\else
    ~~\textcolor{answers}{#1}
\fi}

% longer answers [optional height]
\newsavebox{\ansbox}
\newenvironment{answer}[1][4em]{
\nopagebreak
\begin{lrbox}{\ansbox}
\begin{minipage}[t][#1]{\linewidth}
\color{answers}
}{
\end{minipage}
\end{lrbox}
\ifdefined\Student
    \phantom{\usebox{\ansbox}}%
\else
    \usebox{\ansbox}%
\fi}


\begin{document}

\maketitle

Previously we explored how classes define attributes and methods.
Static variables and methods apply to the whole class, whereas non-static variables and methods apply to specific objects.

\guide{
  \item Discuss benefits of POGIL for student learning.
  \item Explain the purpose of constructor, accessor, and mutator methods.
  \item Implement the equals and toString methods for a given class design.
  \item Design a new class (UML diagram) based on a general description.
}{
  \item Identifying key attributes and data types that model a real-world object. (Problem Solving)
}{
TODO

When reporting out \ref{credit-card.tex}, have presenters write their designs on the board.
Compare the trade-offs of their different designs.
For example, to store credit card numbers some teams may use strings, others may use arrays of integers, and some may use a \texttt{long} variable (\texttt{int} won't work because of the range).
}

%TODO add new model for equals and toString

\section*{Meta Activity: POGIL Research}

\textit{Process-Oriented Guided Inquiry Learning} (see \href{https://pogil.org/}{pogil.org}) is a student-centered, group-learning instructional strategy and philosophy developed through research on how students learn best.
The following two figures are from peer-reviewed articles published in education journals.

% images from Chapter 8 of http://dx.doi.org/10.7936/K7PN93HC

\vspace{1em}
\includegraphics[width=0.47\linewidth]{pogil-grades.png}
\hfill
\includegraphics[width=0.50\linewidth]{pogil-prequiz.png}


\quest{10 min}


\Q How large were the classes at each of the universities shown above?

\begin{answer}[2em]
The left university had classes of about 24, and the right had over 150 students/section.
\end{answer}


\Q What are the measures of performance shown in each of the figures?

\begin{answer}[2em]
The left figure shows grade distributions, and the right figure shows pre-quiz scores.
\end{answer}


\Q What does the figure on the left suggest about POGIL's impact on student success?

\begin{answer}[6em]
\small
``Students in courses employing a POGIL instructional strategy achieved a significantly higher success rate (defined as receiving an A, B, or C in the course, as compared to a D, F or withdrawal) than students who had been taught by the same instructors in previous years using a more traditional lecture-oriented approach.
In both cases, the students were in classes of about 24 students each, and similar exams were used for both groups of students.'' (Moog 2014)
\end{answer}


\Q What does the figure on the right suggest about students' retention of knowledge?

\begin{answer}[6em]
\small
``A majority of the students (60\%) in the lecture section scored below 50 percent on the quiz, and none of the students achieved a score above 90 percent.
Less than five percent scored above 80 percent.
In contrast, fewer than a a quarter of the students from the POGIL section scored below 50 percent on this quiz, and about 30 percent of the students scored above 80 percent, with over one-fifth of the POGIL
students scoring above 90 percent.'' (Moog 2014)
\end{answer}

\newpage
\model{Common Methods}

Classes are often used to represent abstract data types, such as \java{Color} or \java{Point}:

\begin{center}
\includegraphics{Color.pdf}  % immutable
~~~~~
\includegraphics{Point.pdf}  % mutable
\end{center}

As shown in the UML diagrams, classes generally include the following kinds of methods (in addition to others):

\begin{itemize}[itemsep=0pt]
\item \textbf{constructor} methods that initialize new objects
\item \textbf{accessor} methods (getters) that return attributes
\item \textbf{mutator} methods (setters) that modify attributes
%\item \textbf{object} methods such as \java{equals} and \java{toString}
%\item \textbf{utility} methods which are generally static
\end{itemize}

%Note: The \java{Color} class does not have getters and setters.


\quest{15 min}


\Q Identify the constructors for the \java{Color} class.
What is the difference between them?
%What arguments do they take?

\begin{answer}[3em]
There are two constructors: one that takes no parameters (the default constructor), and one that takes three integers for the RGB values.
\end{answer}


\Q What kind of constructor does the \java{Point} class have that the \java{Color} class does not?
%Explain the purpose of such a constructor.

\begin{answer}[3em]
The \java{Point} class also has a copy constructor: one that ``copies'' the values of another object.
\end{answer}


\Q Identify an accessor method in the \java{Point} class.

\begin{enumerate}
\item What is the name of the method? \ans{{\tt getX} or {\tt getY}}
\item Which instance variable does it get? \ans{{\tt this.x} or {\tt this.y}}
\item What arguments does the method take? \ans{none}
\item What does the method return? \ans{The value of \java{x} or \java{y}}
\end{enumerate}


\Q Identify a mutator method in the \java{Point} class.

\begin{enumerate}
\item What is the name of the method? \ans{{\tt setX} or {\tt setY}}
\item Which instance variable does it set? \ans{{\tt this.x} or {\tt this.y}}
\item What arguments does the method take? \ans{The value of \java{x} or \java{y}}
\item What does the method return? \ans{nothing}
\end{enumerate}


\Q How would you define accessor methods for each attribute of the \java{Color} class?
Write your answer using UML syntax.

\begin{answer}[5em]
\begin{javaans}
+getRed(): int
+getGreen(): int
+getBlue(): int
\end{javaans}
\end{answer}


\Q How would you define mutator methods for each attribute of the \java{Color} class?
Write your answer using UML syntax.

\begin{answer}[5em]
\begin{javaans}
+setRed(red:int)
+setGreen(green:int)
+setBlue(blue:int)
\end{javaans}
\end{answer}


\Q \label{key1}
The \java{Color} class does not provide any accessors or mutators.
Instead, it provides methods that return new \texttt{Color} objects.
Why do you think the class was designed this way?

\begin{answer} [5em]
Other than the constructor, there are no methods that change the \texttt{red}, \texttt{green}, and \texttt{blue} values.
This design makes the class immutable, which means that objects can be reused.
The \java{String} class is also designed this way.
\end{answer}

\model{Credit Card}
% based on Model 2 of "Activity 10 - Class Design" by Helen Hu

Classes often represent objects in the real world.
In this section, you will design a new class that represents a \java{CreditCard} like the one below:

\begin{center}
% https://www.bankofamerica.com/credit-cards/
\includegraphics{credit-card.png}
\end{center}


\quest{15 min}


\Q Identify two or more attributes that would be necessary for the \java{CreditCard} class.
For each attribute, indicate what data type would be most appropriate.

\begin{answer}
Answers may include ~\verb|number:long|, ~\verb|expire:Date|, ~\verb|name:String|, ~\verb|code:int|, ~etc.
\end{answer}


\Q Using UML syntax, define two or more constructors for the \java{CreditCard} class.

\begin{answer}
\begin{javaans}
+CreditCard()
+CreditCard(number:long, name:String)
\end{javaans}
\end{answer}


\Q Define two or more accessor methods for the \java{CreditCard} class.
Include arguments and return values, using the same format as a UML diagram.

\begin{answer}[5em]
\begin{verbatim}
+getNumber(): long
+getExpire(): Date
+getName(): String
+getCode(): int
\end{verbatim}
\end{answer}


\Q Define two or more mutator methods for the \java{CreditCard} class.
Include arguments and return values, using the same format as a UML diagram.

\begin{answer}[5em]
\begin{verbatim}
+setNumber(number:long): void
+setExpire(expire:Date): void
+setName(name:String): void
+setCode(code:int): void
\end{verbatim}
\end{answer}


\Q \label{key3}
Describe how you would implement the \java{equals} method of the \java{CreditCard} class.

\begin{answer}
Two credit cards would be considered equal if they have the same account number, assuming there are no duplicates in the bank.
\end{answer}


\Q Describe how you would implement the \java{toString} method of the \java{CreditCard} class.

\begin{answer}
The \java{toString} would print the account number, expiration date, and cardholder's name, each separated by a comma.
\end{answer}


\Q When constructing (or updating) a \java{CreditCard} object, which arguments would you need to validate?
What are the valid ranges of values for each attribute?

\begin{answer}[5em]
The number should have 16 digits, dates need to have valid months and days, names should be at most 22 letters and not contain digits or other characters, code should be 3--4 digits, etc.
\end{answer}


\end{document}
