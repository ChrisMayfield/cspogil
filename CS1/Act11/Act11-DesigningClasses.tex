\documentclass[12pt]{article}

\title{Activity 11: Designing Classes}
\author{Chris Mayfield and Helen Hu}
\date{July 2017}

%\ProvidesPackage{cspogil}

% fonts
\usepackage[utf8]{inputenc}
\usepackage[T1]{fontenc}
\usepackage{mathpazo}

% spacing
\usepackage[margin=2cm]{geometry}
\renewcommand{\arraystretch}{1.4}
\setlength{\parindent}{0pt}

% orphans and widows
\clubpenalty=10000
\widowpenalty=10000
\pagestyle{empty}

% figures and tables
\usepackage{graphicx}
\usepackage{multicol}
\usepackage{tabularx}

% fixed-width columns
\usepackage{array}
\newcolumntype{L}[1]{>{\raggedright\let\newline\\\arraybackslash\hspace{0pt}}m{#1}}
\newcolumntype{C}[1]{>{\centering\let\newline\\\arraybackslash\hspace{0pt}}m{#1}}
\newcolumntype{R}[1]{>{\raggedleft\let\newline\\\arraybackslash\hspace{0pt}}m{#1}}

% include paths
\makeatletter
\def\input@path{{Models/}{../../Models/}}
\graphicspath{{Models/}{../../Models/}}
\makeatother

% colors
\usepackage[svgnames,table]{xcolor}
\definecolor{bgcolor}{HTML}{FAFAFA}
\definecolor{comment}{HTML}{007C00}
\definecolor{keyword}{HTML}{0000FF}
\definecolor{strings}{HTML}{B20000}

% table headers
\newcommand{\tr}{\bf\cellcolor{Yellow!10}}

% syntax highlighting
\usepackage{textcomp}
\usepackage{listings}
\lstset{
    basicstyle=\ttfamily,
    backgroundcolor=\color{bgcolor},
    numberstyle=\scriptsize\color{comment},
    commentstyle=\color{comment},
    keywordstyle=\color{keyword},
    stringstyle=\color{strings},
    columns=fullflexible,
    keepspaces=true,
    showstringspaces=false,
    upquote=true
}

% code environments
\newcommand{\java}[1]{\lstinline[language=java]{#1}}%[
\lstnewenvironment{javalst}{\lstset{language=java,backgroundcolor=}}{}
\lstnewenvironment{javabox}{\lstset{language=java,frame=single,numbers=left}\quote}{\endquote}

% PDF properties
\usepackage[pdftex]{hyperref}
\urlstyle{same}
\makeatletter
\hypersetup{
  pdftitle={\@title},
  pdfauthor={\@author},
  pdfsubject={\@date},
  pdfkeywords={},
  bookmarksopen=false,
  colorlinks=true,
  citecolor=black,
  filecolor=black,
  linkcolor=black,
  urlcolor=blue
}
\makeatother

% titles
\makeatletter
\renewcommand{\maketitle}{\begin{center}\LARGE\@title\end{center}}
\makeatother

% boxes
\newcommand{\emptybox}[1][10em]{
\vspace{1em}
\begin{tabularx}{\linewidth}{|X|}
\hline\\[#1]\hline
\end{tabularx}}

% models
\newcommand{\model}[1]{\section{#1}\nopagebreak}
\renewcommand{\thesection}{Model~\arabic{section}}

% questions
\newcommand{\quest}[1]{\subsection*{Questions~ (#1)}}
\newcounter{question}
\newcommand{\Q}{\vspace{1em}\refstepcounter{question}\arabic{question}.~ }
\renewcommand{\thequestion}{\#\arabic{question}}

% sub-question lists
\usepackage{enumitem}
\setenumerate[1]{label=\alph*)}
\setlist{itemsep=1em,after=\vspace{1ex}}

% inline answers
\definecolor{answers}{HTML}{C0C0C0}
\newcommand{\ans}[1]{%
\ifdefined\Student
    \phantom{~~\textcolor{answers}{#1}}
\else
    ~~\textcolor{answers}{#1}
\fi}

% longer answers [optional height]
\newsavebox{\ansbox}
\newenvironment{answer}[1][4em]{
\nopagebreak
\begin{lrbox}{\ansbox}
\begin{minipage}[t][#1]{\linewidth}
\color{answers}
}{
\end{minipage}
\end{lrbox}
\ifdefined\Student
    \phantom{\usebox{\ansbox}}%
\else
    \usebox{\ansbox}%
\fi}


\begin{document}

\maketitle

Previously we explored how classes define attributes and methods.
Static variables and methods apply to the whole class, whereas non-static variables and methods apply to specific objects.
%In this activity, we'll take a closer look at what objects look like.

\guide{
  \item Discuss benefits of POGIL for student learning.
  \item Explain the purpose of constructor, accessor, and mutator methods.
  \item Implement the equals and toString methods for a given class design.
  \item Design a new class (UML diagram) based on a general description.
}{
  \item Identifying key attributes and data types that model a real-world object. (Problem Solving)
}{
TODO

Ask the managers to pace their team on Model 1; they will need to work quickly.
Set aside 10 minutes between Models 1 and 2 to step through Circle.java using \href{http://pythontutor.com/java.html}{Java Tutor} or similar tool.
Have the recorder document team misconceptions on his/her activity sheet.

When reporting out Model 2, have presenters write their designs on whiteboards.
Compare the trade-offs of different designs: to store credit card numbers, some teams may use strings, others arrays, some int/long.
}

%TODO add new model for equals and toString
% based on Model 2 of "Activity 10 - Class Design" by Helen Hu

\model{Class Design}

Classes are often used to represent abstract data types, such as \java{Color} or \java{Point}.
They are also used to represent objects in the real world, such as \java{CreditCard} (see next page) or \java{Person}.
%UML class diagrams summarize the attributes and methods of a class.

\begin{center}
\includegraphics{CS1B/Color.pdf}
~~~~~
\includegraphics{CS1B/Point.pdf}
\end{center}

Classes generally include the following kinds of methods:
\begin{itemize}[itemsep=0pt]
\item \textbf{constructor} methods that initialize new objects
\item \textbf{accessor} methods (getters) that return attributes
\item \textbf{mutator} methods (setters) that modify attributes
\item \textbf{object} methods such as \java{equals} and \java{toString}
%\item \textbf{utility} methods which are generally static
\end{itemize}


\quest{15 min}


\Q Identify the constructors for the \java{Color} class. What is the difference between them? What arguments do they take? What do these methods return?

\begin{answer}
\end{answer}


\Q Identify an accessor method in the \java{Point} class. 
\begin{enumerate}[itemsep=0pt]
\item Which instance variable does it get?
\item What arguments does the method take?
\item What does the method return?
\end{enumerate}


\Q Identify a mutator method in the \java{Point} class.
\begin{enumerate}[itemsep=0pt]
\item Which instance variable does it set?
\item What arguments does the method take?
\item What does the method return?
\end{enumerate}


\begin{center}
\textit{For the remaining questions, you will design a class that represents an individual's credit card.}
\bigskip\par
% https://www.bankofamerica.com/credit-cards/
\includegraphics{CS1B/credit-card.png}
\end{center}


\Q List two or more attributes that would be necessary for this \java{CreditCard} class. For each attribute, indicate what data type would be most appropriate.

\begin{enumerate}
\item 
\item 
\end{enumerate}


\Q When constructing (or updating) a \java{CreditCard} object, what values would you need to check? What are the valid ranges of values for each attribute?

\begin{enumerate}
\item 
\item 
\end{enumerate}


\Q List two accessor methods would be appropriate for the \java{CreditCard} class.
Include arguments and return values, using the same format as a UML diagram.

\begin{enumerate}
\item 
\item 
\end{enumerate}


\Q List two mutator methods would be appropriate for the \java{CreditCard} class.
Include arguments and return values, using the same format as a UML diagram.

\begin{enumerate}
\item 
\item 
\end{enumerate}

%TODO add new meta activity: POGIL Research

\end{document}
