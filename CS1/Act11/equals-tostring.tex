\model{Object Methods}

In addition to providing constructors, getters, and setters, classes often provide \java{equals} and \java{toString} methods.
These methods make it easier to work with objects of the class.

\vspace{1em}

As a team, review the provided \textit{Color.java} and \textit{Point.java} files.
Run each program to see how it works.
Then answer the following questions using the source code (don't just guess).


\quest{15 min}


\Q Based on the output of \textit{Color.java}, what is the value of each expression below?

\begin{javalst}
Color black = new Color();
Color other = new Color(0, 0, 0);
Color gold = new Color(255, 215, 0);
\end{javalst}

\begin{multicols}{2}
\setlength{\defaultwidth}{5em}
\begin{enumerate}[itemsep=1pt]
\item \java{black == other} \ans{false}
\item \java{black == gold} \ans{false}
\item \java{black.toString()} \ans{"\#000000"}
\item \java{black.equals(other)} \ans{true}
\item \java{black.equals(gold)} \ans{false}
\item \java{gold.toString()} \ans{"\#ffd700"}
\end{enumerate}
\end{multicols}


\Q What is the purpose of the \texttt{toString} method?

\begin{answer}
It returns a \texttt{String} representation of the \texttt{Color} (in HTML/CSS format).
The \java{toString} method makes it easier to examine and debug objects.
\end{answer}


\Q \label{expr}
Based on the output of \textit{Point.java}, what is the value of each expression below?
\begin{javalst}
Point p1 = new Point();
Point p2 = new Point(0, 0);
Point p3 = new Point(3, 3);
\end{javalst}

\begin{multicols}{2}
\setlength{\defaultwidth}{5em}
\begin{enumerate}[itemsep=1pt]
\item \java{p1 == p2} \ans{false}
\item \java{p1.toString()} \ans{"(0, 0)"}
\item \java{p3.toString()} \ans{"(3, 3)"}
\item \java{p1.equals(p2)} \ans{true}
\item \java{p1.equals("(0, 0)")} \ans{false}
\item \java{p3.equals("(3, 3)")} \ans{false}
\end{enumerate}
\end{multicols}


\Q \label{key2}
What is the purpose of the \texttt{equals} method?

\begin{answer}
It determines whether two objects have the same attribute values.
The \java{equals} method is useful for testing with \java{assertEquals}.
\end{answer}


\Q Examine {\it Point.java} again.
What is the purpose of the \java{if}-statement in the \texttt{equals} method?

\begin{answer}
Since \java{equals} can take any type of \java{Object}, you need to check if the argument is a \java{Color} or \java{Point} instance before using it as such.
\end{answer}


\Q How could you modify the \java{equals} method to cause both \ref{expr}e and \ref{expr}f to return \java{true}?

\begin{answer}[5em]
Change the last line to ~\jans{return this.toString().equals(obj);}

\vspace{1em}
You could instead add ~\texttt{if (obj instanceof String)}, but since the \texttt{String.equals} method takes an \texttt{Object}, there's no need to convert the \java{obj} parameter before calling \texttt{String.equals}.
\end{answer}
