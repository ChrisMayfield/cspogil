% comment out for student version
\ifdefined\Student\relax\else\def\Teacher{}\fi

\documentclass[12pt]{article}

\title{Arrays of Objects}
\author{Chris Mayfield}
\date{Fall 2019}

%\ProvidesPackage{cspogil}

% fonts
\usepackage[utf8]{inputenc}
\usepackage[T1]{fontenc}
\usepackage{mathpazo}

% spacing
\usepackage[margin=2cm]{geometry}
\renewcommand{\arraystretch}{1.4}
\setlength{\parindent}{0pt}

% orphans and widows
\clubpenalty=10000
\widowpenalty=10000
\pagestyle{empty}

% figures and tables
\usepackage{graphicx}
\usepackage{multicol}
\usepackage{tabularx}

% fixed-width columns
\usepackage{array}
\newcolumntype{L}[1]{>{\raggedright\let\newline\\\arraybackslash\hspace{0pt}}m{#1}}
\newcolumntype{C}[1]{>{\centering\let\newline\\\arraybackslash\hspace{0pt}}m{#1}}
\newcolumntype{R}[1]{>{\raggedleft\let\newline\\\arraybackslash\hspace{0pt}}m{#1}}

% include paths
\makeatletter
\def\input@path{{Models/}{../../Models/}}
\graphicspath{{Models/}{../../Models/}}
\makeatother

% colors
\usepackage[svgnames,table]{xcolor}
\definecolor{bgcolor}{HTML}{FAFAFA}
\definecolor{comment}{HTML}{007C00}
\definecolor{keyword}{HTML}{0000FF}
\definecolor{strings}{HTML}{B20000}

% table headers
\newcommand{\tr}{\bf\cellcolor{Yellow!10}}

% syntax highlighting
\usepackage{textcomp}
\usepackage{listings}
\lstset{
    basicstyle=\ttfamily,
    backgroundcolor=\color{bgcolor},
    numberstyle=\scriptsize\color{comment},
    commentstyle=\color{comment},
    keywordstyle=\color{keyword},
    stringstyle=\color{strings},
    columns=fullflexible,
    keepspaces=true,
    showstringspaces=false,
    upquote=true
}

% code environments
\newcommand{\java}[1]{\lstinline[language=java]{#1}}%[
\lstnewenvironment{javalst}{\lstset{language=java,backgroundcolor=}}{}
\lstnewenvironment{javabox}{\lstset{language=java,frame=single,numbers=left}\quote}{\endquote}

% PDF properties
\usepackage[pdftex]{hyperref}
\urlstyle{same}
\makeatletter
\hypersetup{
  pdftitle={\@title},
  pdfauthor={\@author},
  pdfsubject={\@date},
  pdfkeywords={},
  bookmarksopen=false,
  colorlinks=true,
  citecolor=black,
  filecolor=black,
  linkcolor=black,
  urlcolor=blue
}
\makeatother

% titles
\makeatletter
\renewcommand{\maketitle}{\begin{center}\LARGE\@title\end{center}}
\makeatother

% boxes
\newcommand{\emptybox}[1][10em]{
\vspace{1em}
\begin{tabularx}{\linewidth}{|X|}
\hline\\[#1]\hline
\end{tabularx}}

% models
\newcommand{\model}[1]{\section{#1}\nopagebreak}
\renewcommand{\thesection}{Model~\arabic{section}}

% questions
\newcommand{\quest}[1]{\subsection*{Questions~ (#1)}}
\newcounter{question}
\newcommand{\Q}{\vspace{1em}\refstepcounter{question}\arabic{question}.~ }
\renewcommand{\thequestion}{\#\arabic{question}}

% sub-question lists
\usepackage{enumitem}
\setenumerate[1]{label=\alph*)}
\setlist{itemsep=1em,after=\vspace{1ex}}

% inline answers
\definecolor{answers}{HTML}{C0C0C0}
\newcommand{\ans}[1]{%
\ifdefined\Student
    \phantom{~~\textcolor{answers}{#1}}
\else
    ~~\textcolor{answers}{#1}
\fi}

% longer answers [optional height]
\newsavebox{\ansbox}
\newenvironment{answer}[1][4em]{
\nopagebreak
\begin{lrbox}{\ansbox}
\begin{minipage}[t][#1]{\linewidth}
\color{answers}
}{
\end{minipage}
\end{lrbox}
\ifdefined\Student
    \phantom{\usebox{\ansbox}}%
\else
    \usebox{\ansbox}%
\fi}


\begin{document}

\maketitle

With arrays and objects, you can represent pretty much any type of data.
It's not only possible to have arrays of objects, but also objects of arrays, objects of objects, arrays of arrays, arrays of objects of arrays, and so forth.

\guide{
  \item Explain the differences when instantiating an array and an object.
  \item Rewrite a for loop (over an array) using an enhanced for loop.
  \item Use enhanced for loops to construct and search arrays of objects.
}{
  \item Developing algorithms for constructing and searching arrays. (Problem Solving)
}{
The first two objectives apply to \ref{hand-of-cards.tex}.
Students should understand the differences in syntax (i.e., using square brackets vs parentheses) as well as what happens behind the scenes (i.e., \texttt{new} invokes a constructor for objects but not arrays).

Keep an eye on what students write down for \ref{random}.
If their answer is superficial, challenge them to be more specific.
Report out on \ref{foreach}, and ask students why \java{i} is not an appropriate variable name for this loop (\java{i} typically stands for index or iteration).

This activity makes use of a simple \java{Card} class.
It stores both the rank and suit as integers, but it does not perform any validation in the constructor.
Consider projecting the \github{CS1/Act12/Card.java}{source code} while you facilitate \ref{deck-of-cards.tex}.
Question~\ref{cnstr} asks students to implement a default value replacement, and it can be extended by having students call a setter method.

\ref{build} and \ref{search} are the most important questions.
Make sure students have enough time to develop the algorithms, and if possible, write the complete source code.
Consider reporting out after \ref{build} to bring the class back in sync, since there is so much time allocated for \ref{deck-of-cards.tex}.
}

\model{Hand of Cards}

Creating an array of objects is typically a 3-step process:

\begin{center}
\begin{minipage}[t]{155pt}

1. Declare the array \\[1ex]
\java{Card[] hand;}

\end{minipage}
\begin{minipage}[t]{155pt}

2. Instantiate the array \\[1ex]
\java{hand = new Card[5];}

\end{minipage}
\begin{minipage}[t]{155pt}

3. Instantiate each object \\[1ex]
\java{hand[0] = new Card(4, 2);}
\java{hand[1] = new Card(3, 1);}

\end{minipage}
\end{center}

\begin{center}
\includegraphics[width=465pt]{cards-array.pdf}
\end{center}


\quest{20 min}


\Q What is the type of the local variable \java{hand}?
What is the value of \java{hand} {\it before} step 2?
What is the value of \java{hand} {\it after} step 2?

\begin{answer}[3em]
The variable \java{hand} is an array of \java{Card} objects.
Before step 2, it's uninitialized (i.e., you can't read its value).
After step 2, its value is the memory location of the first array element.
\end{answer}


\Q When you create an array (e.g., ~\java{new Card[5]}) what is the initial value of each element?

\begin{answer}
The initial values are automatically set to null (for reference types). For arrays of integers, it's 0; for doubles, it's 0.0; for booleans, it's {\tt false}; for characters, it's the unicode character \chr{\textbackslash u0000}.
\end{answer}


\Q When you construct a new object (e.g., ~\java{new Card(4, 2)}) what are the initial values of its attributes (e.g., ~\java{this.rank})?

\begin{answer}
It depends on the constructor. For the \java{Card} class, the attributes \java{rank} and \java{suit} are initialized from the parameters. If there is no constructor, Java automatically initializes attributes to zero (or equivalent).
\end{answer}


\comment{The \java{new} operator finds a memory location to store an array or object.
Java automatically determines how much memory is needed and initializes the contents of the corresponding memory cells to zero.
That's why array elements and object attributes have default values, whereas local variables (not allocated with \java{new}) must be initialized before they are used.}


\Q \label{random}
Describe in your own words what the following code does.
Be sure to explain how the random part works.

\begin{quote}
\java{int index = (int) (Math.random() * hand.length);} \\
\java{hand[index] = null;}
\end{quote}

\begin{answer}
\java{Math.random()} returns a value in the range [0, 1).
Multiplying that value by \java{hand.length} and then casting it to an integer gives a value in the range 0..5 inclusive.
So this statement randomly sets one of the \java{Card} references to {\tt null}.
\end{answer}


\Q \label{forcard}
What is the result of running the loop below?
Explain why the if-statement is necessary.

\begin{javalst}
for (int i = 0; i < hand.length; i++) {
    if (hand[i] != null) {
        int suit = hand[i].getSuit();
        System.out.println("The suit of #" + i + " is " + Card.SUITS[suit]);
    }
}
\end{javalst}

\begin{answer}[3em]
The loop prints the suits of all cards in the hand.
Because some of the cards are {\tt null}, the if-statement prevents {\tt NullPointerException}.
\end{answer}


\Q The \emph{enhanced for loop} allows you to iterate the elements of an array.
Another name for this structure is the ``for each'' loop.
Rewrite the following example using a standard for loop.

\vspace{1ex}
\begin{javalst}
String[] days = {"Sun", "Mon", "Tue", "Wed", "Thu", "Fri", "Sat"};
for (String day : days) {
    System.out.println(day + " is a great day!");
}
\end{javalst}
\vspace{-1em}

\begin{answer}
\begin{javaans}
for (int i = 0; i < days.length; i++) {
    System.out.println(days[i] + " is a great day!");
}
\end{javaans}
\end{answer}


\Q In contrast to enhanced for loops, what does a standard for loop iterate?
Why would it be misleading to name the enhanced for loop variable \java{i} instead of \java{day}?

\begin{answer}[2em]
Standard for loops typically iterate indexes; that's why the variable is usually named \java{i}.
\end{answer}


\Q \label{foreach}
Rewrite the loop in \ref{forcard} using an enhanced for loop.
Use an appropriate variable name for the \java{Card} object (i.e., not \java{i}).
For simplicity, you may omit the \java{System.out.println} line.

\begin{answer}[6em]
\begin{javaans}
for (Card card : hand) {
    if (card != null) {
        int suit = card.getSuit();
    }
}
\end{javaans}
\end{answer}

\model{Deck of Cards}

There are 52 cards in a standard deck.
Each card has one of \emph{13 ranks} (1=Ace, 2, 3, 4, 5, 6, 7, 8, 9, 10, 11=Jack, 12=Queen, and 13=King) and one of \emph{4 suits} (0=Clubs, 3=Spades, 2=Hearts, and 1=Diamonds).
For example, ~\java{new Card(12, 2)} would construct the Queen of Hearts.

\vspace{1em}

The following deck is represented by an array of \java{Card} objects.
The array is one-dimensional, but the cards are shown in four rows (because of the paper margins).

\begin{center}
% http://www.milefoot.com/math/discrete/counting/cardfreq.htm
\includegraphics[width=\linewidth]{playing-cards1.png}
\end{center}


\quest{25 min}


\Q What is the index (in the array above) of the following cards?

\setlength{\defaultwidth}{3em}

\begin{multicols}{2}
\begin{enumerate}
\item Ace of Clubs     \ans{0}
\item Jack of Clubs    \ans{10}
\item 2 of Spades      \ans{14}
\item Queen of Spades  \ans{24}
\item 7 of Hearts      \ans{32}
\item King of Diamonds \ans{51}
\end{enumerate}
\end{multicols}


\Q Write the following statements using one line of code each.

\begin{enumerate}

\item Declare and initialize a \java{Card} array named \java{deck} that can hold 52 cards.
\begin{answer}[1em]
\begin{javaans}
Card[] deck = new Card[52];
\end{javaans}
\end{answer}

\item Construct the Ace of Clubs, and assign it as the first element in \java{deck}.
\begin{answer}[1em]
\begin{javaans}
deck[0] = new Card(1, 0);
\end{javaans}
\end{answer}

\item Construct the King of Diamonds, and assign it as the last element in \java{deck}.
\begin{answer}[1em]
\begin{javaans}
deck[51] = new Card(13, 1);    // or deck[deck.length - 1]
\end{javaans}
\end{answer}

\end{enumerate}


\Q Describe how you could repeat code from the previous question to construct the entire deck of cards (without having to type 52 statements).

\begin{answer}[3em]
Use nested for loops to iterate each possible \java{rank} and \java{suit}, construct that card, and assign it to the next index in the \jans{deck} array.
\end{answer}


\Q \label{build}
Discuss the following code as a team:

\begin{javalst}
int index = 0;
int[] suits = {0, 3, 2, 1};
for (int suit : suits) {
    for (int rank = 1; rank <= 13; rank++) {
        deck[index] = new Card(rank, suit);
        index++;
    }
}
\end{javalst}

\vspace{-1ex}
\begin{enumerate}

\item What is the overall purpose of the code?
\begin{answer}[1em]
It creates a deck of cards in the order shown in \ref{\currfilename}.
\end{answer}

\item Why is the \java{suits} array not just \{0, 1, 2, 3\}? (See \ref{\currfilename}.)
\begin{answer}[1em]
Because the picture shows the suits in a different order.
\end{answer}

\item Why does the code use an enhanced for loop for \java{suit}?
\begin{answer}[1em]
The code iterates the suits out of order, as specified in the array.
\end{answer}

\item Why does the code use a standard for loop for \java{rank}?
\begin{answer}[1em]
The code iterates the ranks in order; no array is needed for that.
\end{answer}

\item What is the purpose of the \java{index} variable?
\begin{answer}[1em]
To keep track of where to store the next card (not based on rank and suit).
\end{answer}

\end{enumerate}
\vspace{-1ex}


\Q \label{search}
Write a method named \java{inDeck} that takes a \java{Card[]} representing a deck of cards and a \java{Card} object representing a single card, and that returns \java{true} if the card is somewhere in the deck.
%Be sure to use the equals method of the \java{Card} object to make comparisons.
%Your solution must avoid \java{NullPointerException} if any cards are missing.
%(Hint: Start with your solution to \ref{foreach}.)

\begin{answer}[10em]
\begin{javaans}
public static boolean inDeck(Card[] deck, Card card) {
    for (Card c : deck) {
        if (c != null && c.equals(card)) {
            return true;
        }
    }
    return false;
}
\end{javaans}
\end{answer}


\Q \label{sort}
 Describe what the following code does and how it works.
(Note: You've come a long way this semester, to be able to understand this example!)

\begin{javalst}
public static Card[] sort(Card[] deck) {
    if (deck == null) {
        System.err.println("Missing deck!");
        return null;
    }
    Card[] sorted = new Card[deck.length];
    for (Card card : deck) {
        int index = card.position();  // returns suit * 13 + rank - 1
        sorted[index] = card;
    }
    return sorted;
}
\end{javalst}

\vspace{-1ex}
\begin{enumerate}

\item What is the overall purpose of the code?
\begin{answer}[1em]
This example sorts an array of cards.
\end{answer}

\item What is the purpose of the if statement?
\begin{answer}[1em]
It avoids \java{NullPointerException} if \java{deck} is invalid.
\end{answer}

\item Does this method modify the \java{deck} array? Justify your answer.
\begin{answer}[1em]
No; it creates and returns a new array named \java{sorted}.
\end{answer}

\item How does the \java{sort} method know where to put each card?
\begin{answer}[1em]
It computes the \java{position} based on its rank and suit.
\end{answer}

\end{enumerate}
\vspace{-1ex}


\Q Identify the following Java language features in the previous question.

\setlength{\defaultwidth}{15em}

\begin{minipage}{22.5em}
\begin{enumerate}

\item variables
\hfill \ans{deck, sorted, card, index}
\item decisions
\hfill \ans{if (deck == null)}
\item loops
\hfill \ans{for (Card card : deck)}
\item methods
\hfill \ans{sort, println, position}
\item arrays
\hfill \ans{deck, sorted}
\item objects
\hfill \ans{"Missing deck!", card}

\end{enumerate}
\end{minipage}


\end{document}
