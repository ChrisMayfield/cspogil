\model{Example Code}

The following examples are found in \textit{Model1.java}.
Open this file on your computer, and run the program.
Record the output of each example in the space below.

\begin{multicols}{2}

\begin{javanum}
int[] nums;
nums = new int[3];

nums[0] = 74;
nums[1] = 11;
nums[2] = 21;

System.out.println(nums.length);
System.out.println(nums);
\end{javanum}

\vspace{1em}
The output is:

\begin{answer}[3em]
\begin{javaans}
3
[I@7440e464
\end{javaans}
\end{answer}

\columnbreak

\begin{javanum}
ArrayList<Integer> nums;
nums = new ArrayList<Integer>();

nums.add(74);
nums.add(11);
nums.add(21);

System.out.println(nums.size());
System.out.println(nums);
\end{javanum}

\vspace{1em}
The output is:

\begin{answer}[3em]
\begin{javaans}
3
[74, 11, 21]
\end{javaans}
\end{answer}

\end{multicols}


\quest{20 min}


\Q Compare the examples line by line, and summarize the differences.

\begin{enumerate}[itemsep=1ex]

\item Line 1: \vspace{-1ex}
\begin{answer}[2em]
Arrays are declared with square brackets; ArrayLists are declared with angle brackets.
The array example uses the primitive type int, but the ArrayList example uses Integer.
\end{answer}

\item Line 2: \vspace{-1ex}
\begin{answer}[2em]
When creating the array, you have to specify the length in brackets.
When creating the ArrayList, you need parentheses after the <>'s.
\end{answer}

\item Lines 3--6: \vspace{-1ex}
\begin{answer}[2em]
Arrays use square brackets to index a specific element.
ArrayLists use the add method, and no indexes are required.
\end{answer}

\item Line 8: \vspace{-1ex}
\begin{answer}[2em]
Arrays have a length attribute.
ArrayLists have a size method.
\end{answer}

\end{enumerate}


\Q What is the main difference in the output of these two examples?

\begin{answer}
Printing the array just displays its memory address.
Printing the ArrayList shows the actual contents.
(The reason why is because ArrayList has a toString method.)
\end{answer}


\Q What happens if you add the following code after Line 6 in the array example?
Verify your answer by editing \textit{Model1.java} and running the program.

\begin{javalst}
nums[3] = 59;
\end{javalst}

\begin{answer}
\begin{javaans}
Exception in thread "main" java.lang.ArrayIndexOutOfBoundsException:
Index 3 out of bounds for length 3
\end{javaans}
\end{answer}


\Q In \textit{Model1.java}, comment out the line you just added in the previous question.
Then add the following line to the ArrayList example.
What is the resulting output?

\begin{javalst}
nums.add(59);
\end{javalst}

\begin{answer}
\begin{javaans}
4
[74, 11, 21, 59]
\end{javaans}
\end{answer}


\Q \label{key1}
Based on your previous answer, what ability do ArrayLists have that arrays do not?

\begin{answer}
Their size can change, i.e., they can ``grow'' as you add new elements.

(They also provide a toString method, so you can print them directly.)
\end{answer}


\Q Add the following line to the ArrayList example. What is the result?

\begin{javalst}
nums[0] = 33;
\end{javalst}

\begin{answer}
Compiler error: \\
The type of the expression must be an array type but it resolved to ArrayList<Integer>.
\end{answer}


\Q In the \java{ArrayList} example, is \java{nums} an \textit{array} or an \textit{object}? Justify your answer.

\begin{answer}
It's an object, because it has methods like add and size.
You can't use it like an array, as shown in the compiler error.
\end{answer}
