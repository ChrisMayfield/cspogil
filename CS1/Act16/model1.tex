\model{Review of Scanner}

The \java{java.util.Scanner} class is useful for reading and parsing text from various sources:

\begin{quote}
\begin{javalst}
// Example 1
Scanner in = new Scanner(System.in);
while (in.hasNextLine()) {
    String line = in.nextLine();
    System.out.println(line);
}

// Example 2
String text = "1 fish 2 fish red fish blue fish";
Scanner sc = new Scanner(text);
System.out.println(sc.nextInt());
System.out.println(sc.next());
System.out.println(sc.nextInt());
System.out.println(sc.next());
\end{javalst}
\end{quote}


\quest{10 min}


\Q For each example above, describe what the \java{Scanner} is scanning.

\begin{enumerate}
\item Example 1:~ \java{new Scanner(System.in)} \ans{keyboard input}
\item Example 2:~ \java{new Scanner(text)} \ans{the ``fish'' string}
\end{enumerate}


\Q Based on the \href{https://docs.oracle.com/en/java/javase/11/docs/api/java.base/java/util/Scanner.html}{documentation for \java{Scanner}}, explain the following:

\begin{enumerate}
\item \java{in.hasNextLine()} \vspace{-1ex}
\begin{answer}[1em]
Returns true if there is another line in the input of this scanner.
\end{answer}

\item \java{in.nextLine()} \vspace{-1ex}
\begin{answer}[1em]
Advances this scanner past the current line and returns the input that was skipped.
\end{answer}

\item \java{s.nextInt()} \vspace{-1ex}
\begin{answer}[1em]
Scans the next token of the input as an int.
\end{answer}

\item \java{s.next()} \vspace{-1ex}
\begin{answer}[1em]
Finds and returns the next complete token from this scanner.
\end{answer}
\end{enumerate}


\newpage

\Q \label{run1}
Open \textit{ScannerDemo.java} in Eclipse, and run the program.
Enter three lines of input, and notice the output.
Then press \textsf{Ctrl+D}, which is the keyboard shortcut for ``end of file'' (EOF).

\begin{enumerate}

\item In the Console, what color was the user's input? \ans[8em]{blue/green}

\item In the Console, what color was the program's output? \ans[8em]{black}

\item What was the complete output of the program? (Note: Do not include the input lines.)

\begin{answer}[12em]
\begin{javaans}
==== Example 1 ====
Line 1
Line 2
Line 3
==== Example 2 ====
1
fish
2
fish
\end{javaans}
\end{answer}

\end{enumerate}
\vspace{-1em}


\Q What effect did pressing \textsf{Ctrl+D} have on the program?
Explain how you think EOF works.

\begin{answer}[5em]
The first example repeated while the input had a next line.
EOF caused the program to exit the while loop and move on to Example 2.
Once the keyboard input ``file'' ended, \java{hasNextLine} returned false.
\end{answer}


\Q \label{key1}
Rewrite the code for Example 2 to output each \textit{word} of the string using a \java{while} loop.
Run your code to make sure it works.

\begin{answer}[6em]
\begin{javaans}
Scanner sc = new Scanner(text);
while (sc.hasNext()) {
    System.out.println(sc.next());
}
\end{javaans}
\end{answer}
