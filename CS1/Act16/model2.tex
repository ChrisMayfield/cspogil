\model{Reading from a File}

The Internet Movie Database (IMDb) maintains information about movies, television shows, video games, and more---including their cast, production crew, trivia, ratings, etc.

\vspace{1em}

Download the \textit{title2020.tsv} file into your Eclipse project.
It's a subset of all titles from the year 2020 and is based on the data available at \url{https://www.imdb.com/interfaces/}.

\vspace{1em}

Create a new program named \textit{IMDb.java}, and add the following lines to the \java{main} method:

\begin{quote}
\begin{javalst}
File file = new File("title2020.tsv");
Scanner in = new Scanner(file);
for (int i = 0; i < 3; i++) {
    System.out.println(in.nextLine());
}
in.close();
\end{javalst}
\end{quote}

Note: You will need to import \java{java.io.File} and \java{java.util.Scanner}.

\quest{20 min}


\Q Explain the compiler error when attempting to construct the \java{Scanner}.

\begin{answer}[3em]
Unhandled exception type FileNotFoundException.

(The constructor throws FileNotFoundException.)
\end{answer}


\Q Explain two ways you can modify the code to handle this error.
(\textit{Hint:} Eclipse offers them as ``quick fixes''.)
Which way is better?

\begin{answer}[5em]
You can (1) add a throws declaration, or (2) surround with try/catch.
The first option is simpler because it changes the least amount of code.
But it causes the entire method to ignore file not found exceptions.
The latter option is better, because it isolates the problem.
\end{answer}


\Q \label{FNF}
Modify the program so that it compiles:
1) surround the ``\java{new Scanner}'' line with try/catch;
2) remove the auto-generated \java{TODO} comment; and
3) initialize the \java{in} variable to \java{null} (before the \java{try} block).
Copy and paste the beginning on your \java{main} method (from the ``\java{File file}'' line to the end of the \java{catch} block) in the box below.

\begin{answer}[10em]
\begin{javaans}
File file = new File("title2020.tsv");
Scanner in = null;
try {
    in = new Scanner(file);
} catch (FileNotFoundException e) {
    e.printStackTrace();
}
\end{javaans}
\end{answer}


\Q Run the program. What is the output of the \java{for} loop?

\begin{answer}[4em]
\begin{javaans}
tid	type	title	original	isadult	year	ended	runtime	genres
60366	short	A Embalagem de Vidro	A Embalagem de Vidro	0	2020		11	Documentary,Short
62336	movie	El tango del viudo y su espejo deformante	El tango del viudo y su espejo deformante	0	2020		70	Drama
\end{javaans}
\end{answer}


\Q TSV stands for ``tab-separated values''.
Explain the format of the \textit{title2020.tsv} file:

\begin{enumerate}
\item What does the first line represent? \ans{the attribute names}
\item What do the remaining lines represent? \ans{movies / TV shows}
\item How are ``column breaks'' represented? \ans{by tab characters}
\end{enumerate}
\vspace{-1em}


%\Q Open \textit{title2020.tsv} using a spreadsheet application (like Excel).
%
%\begin{enumerate}
%\item How many lines does the file have? \ans[5em]{34,698}
%\item How many ``columns'' does it have? \ans[5em]{9}
%\end{enumerate}
%\vspace{-1em}


\Q By default, \java{Scanner.next()} separates input by whitespace.
Replace the \java{for} loop in your \java{main} method with the following code.
What is the resulting output?

\begin{quote}
\begin{javalst}
in.useDelimiter("\t");
in.nextLine();

for (int i = 0; i < 3; i++) {
    System.out.println(in.next());
}
\end{javalst}
\end{quote}
\vspace{-1em}

\begin{answer}
\begin{javaans}
60366
short
A Embalagem de Vidro
\end{javaans}
\end{answer}
        
\Q \label{key2}
Remove the \java{for} loop from your program.
Write code that outputs the \java{tid}, \java{type}, and \java{title} of the first 5 titles that start with \str{A}.
(\textit{Hint:} Use a \java{while} loop, call \java{in.next()} to get each of the three values, and call \java{in.nextLine()} to advance to the next line.)
Paste your code below, starting from the \java{while} loop.

\begin{answer}[15em]
\begin{javaans}
int count = 0;
while (count < 5) {
    String tid = in.next();
    String type = in.next();
    String title = in.next();
    if (title.startsWith("A")) {
        System.out.println(tid + "\t" + type + "\t" + title);
        count++;
    }
    in.nextLine();
}
\end{javaans}
\end{answer}
