\model{Writing to a File}

The \java{java.io.PrintWriter} class is useful for writing text files:

\medskip

\begin{javalst}
File file = new File("results.tsv");
PrintWriter out = new PrintWriter(file);
// output text to the file...
out.close();
\end{javalst}


\quest{15 min}


\Q Examine the \href{https://docs.oracle.com/en/java/javase/11/docs/api/java.base/java/io/PrintWriter.html}{documentation for \java{PrintWriter}}.
What methods can be used to output a string to the file?

\begin{answer}[3em]
append, format, print, printf, println, write
\end{answer}


\Q \label{key3}
Modify your code from Question \ref{key2} to output to the \textit{results.tsv} file instead of to the screen.
Summarize your changes below:

\begin{answer}[6em]
1. Add the lines from \ref{\currfilename} to the beginning/end of \java{main}. \\
2. Surround the PrintWriter constructor in a try/catch block. \\
3. Replace each \jans{System.out.println} with just \java{out.println}.
\end{answer}


\Q In general, is it easier to write code that reads a file or writes a file?
Explain your reasoning.

\begin{answer}
Writing a file is much easier, because you don't have to parse the file contents and deal with potentially incorrect formatting.
\end{answer}



\Q Make sure the end of your \java{main} method closes both files.
Why is it important to close files when you are finished with them?

\begin{answer}
Closing a file releases any system resources associated with it.
And when writing files, the output may not be on disk until it's closed.
\end{answer}


\Q (Optional) What is the difference between the \java{print} methods and the \java{write} methods in the \java{PrintWriter} class?

\begin{answer}
The \java{print} methods should be used most of the time for outputting text.
The \java{write} methods are lower level and used for printing characters without encoding them based on the current operating environment.
\end{answer}
