\model{Months of the Year}

Open \textit{JShell} on your computer.
Type (or copy and paste) the following enum definition:

%    JANUARY, FEBRUARY, MARCH, APRIL, MAY, JUNE, JULY, AUGUST, SEPTEMBER,
%    OCTOBER, NOVEMBER, DECEMBER;

\begin{quote}
\begin{javalst}
public enum Month {
    JAN, FEB, MAR, APR, MAY, JUN, JUL, AUG, SEP, OCT, NOV, DEC;
}
\end{javalst}
\end{quote}

%\vspace{-18.5em}
%\hfill \includegraphics[height=16em]{Month1_Initial.png}
%\hspace{5em}
%\vspace{1.5em}

Then type each line of code below in \textit{JShell}, \emph{one at a time}, and record the results.
You only need to record the output to the right of the ``\java{==>}'' symbol.
For example, if \textit{JShell} outputs \verb|$8 ==> true|, then just write \verb|true|.
If an error occurs, summarize the error message.

\setlength{\defaultwidth}{20.8em}

\begin{center}
\begin{tabular}{|l|p{21em}|}
\hline
\multicolumn{1}{|c|}{\tr Java code} &
\multicolumn{1}{ c|}{\tr Shell output}
\\ \hline

\java{Month m = null;}
& \ans{null}
\\ %\hline

\java{m = JUN;}
& \ans{cannot find symbol: variable JUN}
\\ %\hline

\java{m = Month.JUN;}
& \ans{JUN}
\\ %\hline

\java{m.toString()}
& \ans{"JUN"}
\\ \hline

\java{Month spring = Month.MAR;}
& \ans{MAR}
\\ %\hline

\java{Month summer = Month.JUN;}
& \ans{JUN}
\\ %\hline

\java{m == spring}
& \ans{false}
\\ %\hline

\java{m == summer}
& \ans{true}
\\ %\hline

\java{Month.JUL = summer;}
& \ans{cannot assign a value to final variable JUL}
\\ \hline

\java{m.ordinal()}
& \ans{5}
\\ %\hline

\java{spring.ordinal()}
& \ans{2}
\\ %\hline

\java{Month.OCT.ordinal()}
& \ans{9}
\\ %\hline

\java{m.compareTo(spring)}
& \ans{3}
\\ %\hline

\java{m.compareTo(Month.OCT)}
& \ans{-4}
\\ \hline

\java{m = Month.valueOf("Mar");}
& \ans{IllegalArgumentException: No enum constant}
\\ %\hline

\java{m = Month.valueOf("MAR");}
& \ans{MAR}
\\ %\hline

\java{m == spring}
& \ans{true}
\\ %\hline

\java{m = Month.valueOf(5);}
& \ans{no suitable method found for valueOf(int)}
\\ %\hline

\java{m = new Month("HEY");}
& \ans{enum types may not be instantiated}
\\ \hline

\java{Month[] all = Month.values();}
& \ans{Month[12] \{ JAN, FEB, MAR, ... \}}
\\ %\hline

\java{all[0]}
& \ans{JAN}
\\ %\hline

\java{all[11]}
& \ans{DEC}
\\ %\hline

\java{all[12]}
& \ans{ArrayIndexOutOfBoundsException}
\\ \hline

\end{tabular}
\end{center}


\quest{25 min}


\Q Consider the variables \java{JAN}, \java{FEB}, \java{MAR}, etc.
Based on how they were used:

\begin{multicols}{3}
\begin{enumerate}
\item Are they \java{public}? \ans[2em]{Yes}
\item Are they \java{static}? \ans[2em]{Yes}
\item Are they \java{final}?  \ans[2em]{Yes}
\end{enumerate}
\end{multicols}


\Q Is the variable ``\java{m}'' a primitive type or a reference type?
Justify your answer.
(If primitive, what is its value?
If not, what does it reference?)

\begin{answer}[3em]
A reference type, because it can be \jans{null}.
It references one of the constants in \java{Month}.
\end{answer}


\Q What ability do classes have that enums do not have?
(\textit{Hint:} Review the error messages.)

\begin{answer}[3em]
Classes can be instantiated using the \jans{new} operator.
\end{answer}


\Q \label{key1}
Based on your previous answers, explain why it's okay to compare enum variables using the \java{==} operator (as opposed to calling the \java{equals} method).

\begin{answer}
If two enums variables are equal, then they will be referencing the same constant.
Only one of each constant will exist; there will not be multiple equivalent \java{Month} objects.
\end{answer}


\Q What does the \java{ordinal} method return?
Explain the range of possible values.

\begin{answer}
The ``order'' of the constant (i.e., its position in the enum definition).
For \java{Month}, the possible values range from 0 to 11.
\end{answer}


\Q What does the \java{compareTo} method return?
Explain how to interpret the results.

\begin{answer}
The difference between the ordinal values.
Negative results mean ``this'' value comes before the ``other'' value;
positive results mean ``this'' value comes after the ``other'' value.
\end{answer}


\Q What does the \java{valueOf} method return?

\begin{answer}[3em]
The \java{Month} object with the specified name, or it throws \java{IllegalArgumentException} if the name is not found.
\end{answer}


\Q What does the \java{values} method return?

\begin{answer}[3em]
An array of references to all constants in the enum.
This method is useful for iterating the constants.
\end{answer}


\Q Which of the aforementioned methods are \java{static}?
Explain how you can tell.

\begin{answer}
Enums have two static methods: \java{valueOf} and \java{values}.
These methods are invoked on the enum itself, rather than an instance.
\end{answer}


\Q \label{key2}
The following code snippet prompts the user to input their birth month:

\begin{quote}
\begin{javalst}
Scanner in = new Scanner(System.in);
System.out.print("What month were you born? ");
String line = in.nextLine();
\end{javalst}
\end{quote}

\begin{enumerate}

\item Declare a variable named \java{birth} and initialize it to the \java{Month} object that corresponds to the user input.
(\textit{Hint:} Use \java{valueOf}.)

\begin{answer}[2em]
\begin{javaans}
Month birth = Month.valueOf(line);
\end{javaans}
\end{answer}

\item Output a message that tells the user the number of their birth month.
For example, if the user inputs \,\verb|MAY|, output the message \,\verb|You were born in month #5|.
(\textit{Hint:} Use \java{ordinal}.)

\begin{answer}[2em]
\begin{javaans}
System.out.printf("You were born in month #%d\n", birth.ordinal() + 1);
\end{javaans}
\end{answer}

\item Write an enhanced \java{for} loop that outputs each of the \java{Month} names that come before \java{birth}.
(\textit{Hint:} Use \java{values} and \java{compareTo}.)

\begin{answer}[8em]
\begin{javaans}
for (Month m : Month.values()) {
    if (m.compareTo(birth) < 0) {
        System.out.println(m);
    }
}
\end{javaans}
\end{answer}

\end{enumerate}


%\Q Write a \java{switch} statement that prints the season of the \java{birth} month.
%\java{MAR}--\java{MAY} is spring, \java{JUN}--\java{AUG} is summer, \java{SEP}--\java{NOV} is autumn, and \java{DEC}--\java{FEB} is winter.
%
%\begin{answer}[32em]
%\begin{javaans}
%switch (birth) {
%
%    case MAR:
%    case APR:
%    case MAY:
%        System.out.println("spring");
%        break;
%
%    case JUN:
%    case JUL:
%    case AUG:
%        System.out.println("summer");
%        break;
%
%    case SEP:
%    case OCT:
%    case NOV:
%        System.out.println("autumn");
%        break;
%
%    case DEC:
%    case JAN:
%    case FEB:
%        System.out.println("winter");
%        break;
%
%}
%\end{javaans}
%\end{answer}
