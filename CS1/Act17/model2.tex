\model{Attributes and Methods}

Here is a new and improved version of the \java{enum} from \ref{model1.tex}.
Read and discuss the following source code as a team.

\medskip

\begin{javabox}
public enum Month {

    JAN("January", 31),
    FEB("February", 28),
    MAR("March", 31),
    APR("April", 30),
    MAY("May", 31),
    JUN("June", 30),
    JUL("July", 31),
    AUG("August", 31),
    SEP("September", 30),
    OCT("October", 31),
    NOV("November", 30),
    DEC("December", 31);

    private final String name;
    private final int days;

    private Month(String name, int days) {
        this.name = name;
        this.days = days;
    }

    public int length() {
        return days;
    }

    public int number() {
        return ordinal() + 1;
    }

    public static Month parseMonth(String name) {
        String abbr = name.substring(0, 3);
        return valueOf(abbr.toUpperCase());
    }

    public String toString() {
        return name;
    }

}
\end{javabox}


\quest{20 min}


\Q What are the attributes of a \java{Month} object?

\begin{answer}[3em]
The \jans{name} of a month, and the number of \jans{days} in a month.
\end{answer}


\Q Open the provided \textit{Month.java} file.
Try changing the constructor to \java{public}.
What compiler error results?

\begin{answer}[3em]
Illegal modifier for the enum constructor; only private is permitted.
\end{answer}


\Q \label{key3}
Based on what you observed in \ref{model1.tex}, why do you think an \java{enum} constructor must be declared \java{private}?

\begin{answer}
Enum types may not be instantiated in other classes using the \jans{new} operator.
However, they may be instantiated within the enum definition itself.
\end{answer}


\Q On which lines is the \java{Month} constructor called in \ref{\currfilename}?

\begin{answer}[3em]
Lines 3--14, where the \java{Month} constants are defined.
\end{answer}


\Q Other than \java{substring} and \java{toUpperCase}, what methods are called in \ref{\currfilename} that are not explicitly defined in \textit{Month.java}?

\begin{answer}[3em]
The \java{ordinal} method (on Line 29) and the \java{valueOf} method (on Line 34).
\end{answer}


\Q The \java{number} method returns the numeric value of the month (i.e., \java{1} for January or \java{12} for December).
Explain how the implementation works.

\begin{answer}
The \java{ordinal} method returns the month's position in the enum declaration, which is a number from 0 to 11.
Adding one to this number yields the desired result.
\end{answer}


\Q The \java{parseMonth} method returns the \java{Month} that corresponds to the provided name.
Explain how the implementation works.

\begin{answer}
It gets the first three letters of the given name and converts them to uppercase.
Then it uses the \java{valueOf} method to get the corresponding \java{Month} constant.
\end{answer}


\Q \label{HelpV1}
Open the provided \textit{MonthHelp.java} file, and discuss the code as a team.
Write additional code that displays the full name and number of days in the month input by the user.
For example, if the user inputs \,\verb|Sept.|, output the message \,\verb|September has 30 days|.

\begin{answer}[6em]
\begin{javaans}
Month m = Month.parseMonth(line);
System.out.printf("%s has %d days\n", m, m.length());
\end{javaans}
\end{answer}


\Q Implement a new method named \java{parseMonth(int number)} that returns the month for the given integer.
For example, \java{parseMonth(1)} would return \java{JAN}, \java{parseMonth(2)} would return \java{FEB}, and so forth.
(\textit{Hint:} Use \java{values}.)

\begin{answer}[6em]
\begin{javaans}
public static Month parseMonth(int number) {
    return values()[number - 1];
}
\end{javaans}
\end{answer}


\Q \label{key4}
Extend your code from \ref{HelpV1} to use both versions of \java{parseMonth}.
If the user inputs a month name or 3-letter abbreviation, call the string version.
If the user inputs a month number, call the integer version.
(\textit{Hint:} Use \java{line.length()} and \java{Integer.parseInt(line)}.)

\begin{answer}[12em]
\begin{javaans}
Month m;
if (line.length() > 2) {
    m = Month.parseMonth(line);
} else {
    int number = Integer.parseInt(line);
    m = Month.parseMonth(number);
}
System.out.printf("%s has %d days\n", m, m.length());
\end{javaans}
\end{answer}
