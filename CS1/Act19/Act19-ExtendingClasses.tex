% comment out for student version
\ifdefined\Student\relax\else\def\Teacher{}\fi

\documentclass[12pt]{article}

\title{Extending Classes}
\author{Chris Mayfield}
\date{Summer 2021}

%\ProvidesPackage{cspogil}

% fonts
\usepackage[utf8]{inputenc}
\usepackage[T1]{fontenc}
\usepackage{mathpazo}

% spacing
\usepackage[margin=2cm]{geometry}
\renewcommand{\arraystretch}{1.4}
\setlength{\parindent}{0pt}

% orphans and widows
\clubpenalty=10000
\widowpenalty=10000
\pagestyle{empty}

% figures and tables
\usepackage{graphicx}
\usepackage{multicol}
\usepackage{tabularx}

% fixed-width columns
\usepackage{array}
\newcolumntype{L}[1]{>{\raggedright\let\newline\\\arraybackslash\hspace{0pt}}m{#1}}
\newcolumntype{C}[1]{>{\centering\let\newline\\\arraybackslash\hspace{0pt}}m{#1}}
\newcolumntype{R}[1]{>{\raggedleft\let\newline\\\arraybackslash\hspace{0pt}}m{#1}}

% include paths
\makeatletter
\def\input@path{{Models/}{../../Models/}}
\graphicspath{{Models/}{../../Models/}}
\makeatother

% colors
\usepackage[svgnames,table]{xcolor}
\definecolor{bgcolor}{HTML}{FAFAFA}
\definecolor{comment}{HTML}{007C00}
\definecolor{keyword}{HTML}{0000FF}
\definecolor{strings}{HTML}{B20000}

% table headers
\newcommand{\tr}{\bf\cellcolor{Yellow!10}}

% syntax highlighting
\usepackage{textcomp}
\usepackage{listings}
\lstset{
    basicstyle=\ttfamily,
    backgroundcolor=\color{bgcolor},
    numberstyle=\scriptsize\color{comment},
    commentstyle=\color{comment},
    keywordstyle=\color{keyword},
    stringstyle=\color{strings},
    columns=fullflexible,
    keepspaces=true,
    showstringspaces=false,
    upquote=true
}

% code environments
\newcommand{\java}[1]{\lstinline[language=java]{#1}}%[
\lstnewenvironment{javalst}{\lstset{language=java,backgroundcolor=}}{}
\lstnewenvironment{javabox}{\lstset{language=java,frame=single,numbers=left}\quote}{\endquote}

% PDF properties
\usepackage[pdftex]{hyperref}
\urlstyle{same}
\makeatletter
\hypersetup{
  pdftitle={\@title},
  pdfauthor={\@author},
  pdfsubject={\@date},
  pdfkeywords={},
  bookmarksopen=false,
  colorlinks=true,
  citecolor=black,
  filecolor=black,
  linkcolor=black,
  urlcolor=blue
}
\makeatother

% titles
\makeatletter
\renewcommand{\maketitle}{\begin{center}\LARGE\@title\end{center}}
\makeatother

% boxes
\newcommand{\emptybox}[1][10em]{
\vspace{1em}
\begin{tabularx}{\linewidth}{|X|}
\hline\\[#1]\hline
\end{tabularx}}

% models
\newcommand{\model}[1]{\section{#1}\nopagebreak}
\renewcommand{\thesection}{Model~\arabic{section}}

% questions
\newcommand{\quest}[1]{\subsection*{Questions~ (#1)}}
\newcounter{question}
\newcommand{\Q}{\vspace{1em}\refstepcounter{question}\arabic{question}.~ }
\renewcommand{\thequestion}{\#\arabic{question}}

% sub-question lists
\usepackage{enumitem}
\setenumerate[1]{label=\alph*)}
\setlist{itemsep=1em,after=\vspace{1ex}}

% inline answers
\definecolor{answers}{HTML}{C0C0C0}
\newcommand{\ans}[1]{%
\ifdefined\Student
    \phantom{~~\textcolor{answers}{#1}}
\else
    ~~\textcolor{answers}{#1}
\fi}

% longer answers [optional height]
\newsavebox{\ansbox}
\newenvironment{answer}[1][4em]{
\nopagebreak
\begin{lrbox}{\ansbox}
\begin{minipage}[t][#1]{\linewidth}
\color{answers}
}{
\end{minipage}
\end{lrbox}
\ifdefined\Student
    \phantom{\usebox{\ansbox}}%
\else
    \usebox{\ansbox}%
\fi}


\begin{document}

\maketitle

The \java{extends} keyword allows you to define a new class based on an existing class.
This way, you can define new versions of classes without having to copy and paste their source code.

\rolenames

\guide{
  \item Explain what it means for one class to extend another.
  \item Summarize uses of the keywords \java{extends} and \java{super}.
  \item Write a new method for an existing Java library class.
}{
  \item Making conclusions based on IDE hints and program output. (Critical Thinking)
}{
This activity requires some prior knowledge of BigInteger and the \java{String.replace()} method.
All that's needed is a 10-minute mini lecture (e.g., using JShell) that shows how to create and add two BigIntegers, and how to find and replace strings.

BigInteger was originally chosen for this activity, because (1) it's a useful class in the Java library, and (2) it's not declared as \java{final} (so it can be extended).
The new feature this activity is adding to BigInteger is the ability to work with comma separators.

Highlight important aspects of the documentation for BigInteger.
For example, mention that BigInteger extends Number, which extends Object.
Notice the static constant fields.
In addition to the constructors, there is a static \java{valueOf} method.

Prevent students from spending too much time on \ref{count1} and \ref{count2}.
You can tell them the number of methods; they should be able to count the fields and constructors.
Introduce ``is a'' and ''has a'' terminology when reporting out.

Key questions: \ref{key1}, \ref{key2}, \ref{key3}

Source files: \src{Act19}{MyBigInt.java}
}

\model{Repair Shop}

You have been asked to design software for a local automobile repair shop:

\begin{quote}
\slshape
Wrench Craft, owned by Dean George, is an independent automobile repair shop in Harrisonburg, VA, specializing in repair of Asian import vehicles such as Honda, Toyota, Subaru, Nissan, Isuzu, Mitsubishi, Hyundai, and Mazda. If you'd like to arrange a time for us to service your car, just give us a call and we'll set you up with an appointment to get the work done. What could be more simple?

\normalfont
Source: \href{http://www.wrenchcraft.com/}{wrenchcraft.com}
\end{quote}

Among other things, the software needs to keep track of \emph{customers} (e.g., name, address, phone number), \emph{cars} (e.g., make, model, year), and \emph{invoices} (e.g., parts, labor, notes).

\quest{15 min}


\Q Identify all nouns in the description above.
Don't worry about pronouns.
List each noun only once (not every time it appears).
Feel free to use an online tool like \href{https://parts-of-speech.info/}{parts-of-speech.info}.

\begin{answer}[6em]
software, automobile, repair, shop, Wrench, Craft, Dean, George, Harrisonburg, VA, import, vehicles, Honda, Toyota, Subaru, Nissan, Isuzu, Mitsubishi, Hyundai, Mazda, time, car(s), call, appointment, work, Source, wrenchcraft.com, track, customers, name, address, phone, number, make, model, year, invoices, parts, labor, notes
\end{answer}


\Q \label{relevant}
Discuss which nouns are more likely to be relevant for designing software.
For each relevant noun, determine whether it represents a class, an object, or an attribute:

\begin{enumerate}[itemsep=0ex]

\item Classes: \hfill
\ans[33.5em]{Shop, Appointment, Customer, Car, Invoice, \ldots}

\item Objects: \hfill
\ans[33.5em]{Dean George, Harrisonburg, Honda, Toyota, \ldots}

\item Attributes: \hfill
\ans[33.5em]{name, address, phone number, make, model, year, parts, labor, notes}

\end{enumerate}


\Q \label{key1}
Based on your discussion for \ref{relevant}, write a general definition for the following concepts:

\begin{enumerate}[itemsep=0ex]

\item Classes: \hfill
\ans[33.5em]{General categories of real-world things}

\item Objects: \hfill
\ans[33.5em]{Examples (instances) of a particular category}

\item Attributes: \hfill
\ans[33.5em]{Characteristics (or values) of a real-world thing}

\end{enumerate}


\Q What other classes, objects, and attributes (not included in the description above) might be relevant for this software project?

\begin{answer}
Appointment (e.g., date, time, type) \\
Inventory (e.g., part, quantity) \\
Parts (e.g., number, name, cost) \\
\end{answer}

\newpage
\model{Adding New Methods}

Add the following method to the \java{MyBigInt} class:

\begin{quote}
\begin{javalst}
public MyBigInt reverse() {
    String str = super.toString();
    final int N = str.length();

    // reverse the digits in the string
    StringBuilder sb = new StringBuilder(N);
    for (int i = 0; i < N; i++) {
        int j = N - 1 - i;
        sb.append(str.charAt(j));
    }
    return new MyBigInt(sb.toString());
}
\end{javalst}
\end{quote}

Add the following code to the \java{main} method:

\begin{quote}
\begin{javalst}
BigInteger bi1 = new BigInteger("12345678");
MyBigInt bi2 = new MyBigInt("12,345,678");
System.out.println(bi1.reverse());
System.out.println(bi2.reverse());
\end{javalst}
\end{quote}


\quest{20 min}


\Q Attempt to compile and run the program. Explain the error in \java{main}.

\begin{answer}
The method \java{reverse()} is undefined for the type \java{BigInteger}.
It was defined only in the \java{MyBigInt} class.
\end{answer}


\Q Remove the line that caused the error, and run the program.
What is the result?

\begin{answer}[2em]
87,654,321
\end{answer}


\Q Which \java{toString} method (in which class) is invoked on the first line of \java{reverse}?

\begin{answer}[2em]
\java{BigInteger.toString()}
\end{answer}


\Q Explain why \java{reverse()} does not need to worry about the placement of commas.

\begin{answer}
The string returned from \java{BigInteger.toString()} does not contain any commas.
The commas are only added when \java{System.out.println} calls \java{MyBigInt.toString()}.
\end{answer}


\Q Consider a method \java{isPalindrome()} that determines whether a \java{MyBigInt} has the same digits forward and backward.
For example, \java{123,321} and \java{12,321} are palindromes, but \java{123,421} and \java{12,341} are not.
How could you implement this method using one line of code?

\begin{javalst}
public boolean isPalindrome() {
\end{javalst}

\vspace{-1ex}
\begin{answer}[1em]
\begin{javaans}
    return this.equals(this.reverse());  // note: "this." is optional
\end{javaans}
\end{answer}
\vspace{-1ex}

\begin{javalst}
}
\end{javalst}


\Q Why is the one-line implementation inefficient, especially for very large integers?

\begin{answer}
It converts the integer to a string, then it makes a copy of the string, and then it copies the string into a new \java{MyBigInt}.
The two extra copies are unnecessary just for checking digits.
\end{answer}


\Q \label{key3}
Rewrite \java{isPalindrome()} to be more efficient.
(\textit{Hint:} Use the source code of \java{reverse()} as a starting point.)

\begin{javalst}
public boolean isPalindrome() {
\end{javalst}

\begin{answer}[18em]
\begin{javaans}
    String str = super.toString();
    final int N = str.length();

    // check each pair of digits
    for (int i = 0; i < N / 2; i++) {
        int j = N - 1 - i;
        if (str.charAt(i) != str.charAt(j)) {
            return false;
        }
    }
    return true;
\end{javaans}
\end{answer}

\begin{javalst}
}
\end{javalst}


\Q Add your solution to \textit{MyBigInt.java}, and make sure it works.
What code can you add to \java{main} to test the \java{isPalindrome} method?

\begin{answer}[8em]
\begin{javaans}
System.out.println(new MyBigInt("123321").isPalindrome());
System.out.println(new MyBigInt("12321").isPalindrome());
System.out.println(new MyBigInt("123421").isPalindrome());
System.out.println(new MyBigInt("12341").isPalindrome());
\end{javaans}
\end{answer}


\end{document}
