\model{Adding New Methods}

Add the following method to the \java{MyBigInt} class:

\begin{quote}
\begin{javalst}
public MyBigInt reverse() {
    String str = super.toString();
    final int N = str.length();

    // reverse the digits in the string
    StringBuilder sb = new StringBuilder(N);
    for (int i = 0; i < N; i++) {
        int j = N - 1 - i;
        sb.append(str.charAt(j));
    }
    return new MyBigInt(sb.toString());
}
\end{javalst}
\end{quote}

Add the following code to the \java{main} method:

\begin{quote}
\begin{javalst}
BigInteger bi1 = new BigInteger("12345678");
MyBigInt bi2 = new MyBigInt("12,345,678");
System.out.println(bi1.reverse());
System.out.println(bi2.reverse());
\end{javalst}
\end{quote}


\quest{20 min}


\Q Attempt to compile and run the program. Explain the error in \java{main}.

\begin{answer}
The method \java{reverse()} is undefined for the type \java{BigInteger}.
It was defined only in the \java{MyBigInt} class.
\end{answer}


\Q Remove the line that caused the error, and run the program.
What is the result?

\begin{answer}[2em]
87,654,321
\end{answer}


\Q Which \java{toString} method (in which class) is invoked on the first line of \java{reverse}?

\begin{answer}[2em]
\java{BigInteger.toString()}
\end{answer}


\Q Explain why \java{reverse()} does not need to worry about the placement of commas.

\begin{answer}
The string returned from \java{BigInteger.toString()} does not contain any commas.
The commas are only added when \java{System.out.println} calls \java{MyBigInt.toString()}.
\end{answer}


\Q Consider a method \java{isPalindrome()} that determines whether a \java{MyBigInt} has the same digits forward and backward.
For example, \java{123,321} and \java{12,321} are palindromes, but \java{123,421} and \java{12,341} are not.
How could you implement this method using one line of code?

\begin{javalst}
public boolean isPalindrome() {
\end{javalst}

\vspace{-1ex}
\begin{answer}[1em]
\begin{javaans}
    return this.equals(this.reverse());  // note: "this." is optional
\end{javaans}
\end{answer}
\vspace{-1ex}

\begin{javalst}
}
\end{javalst}


\Q Why is the one-line implementation inefficient, especially for very large integers?

\begin{answer}
It converts the integer to a string, then it makes a copy of the string, and then it copies the string into a new \java{MyBigInt}.
The two extra copies are unnecessary just for checking digits.
\end{answer}


\Q \label{key3}
Rewrite \java{isPalindrome()} to be more efficient.
(\textit{Hint:} Use the source code of \java{reverse()} as a starting point.)

\begin{javalst}
public boolean isPalindrome() {
\end{javalst}

\begin{answer}[18em]
\begin{javaans}
    String str = super.toString();
    final int N = str.length();

    // check each pair of digits
    for (int i = 0; i < N / 2; i++) {
        int j = N - 1 - i;
        if (str.charAt(i) != str.charAt(j)) {
            return false;
        }
    }
    return true;
\end{javaans}
\end{answer}

\begin{javalst}
}
\end{javalst}


\Q Add your solution to \textit{MyBigInt.java}, and make sure it works.
What code can you add to \java{main} to test the \java{isPalindrome} method?

\begin{answer}[8em]
\begin{javaans}
System.out.println(new MyBigInt("123321").isPalindrome());
System.out.println(new MyBigInt("12321").isPalindrome());
System.out.println(new MyBigInt("123421").isPalindrome());
System.out.println(new MyBigInt("12341").isPalindrome());
\end{javaans}
\end{answer}
