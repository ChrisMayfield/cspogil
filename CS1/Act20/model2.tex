\model{Variable vs Object Types}

Consider the following program:

\begin{quote}
\begin{javalst}
public static void main(String[] args) {
    Person p1 = new Person("Helen", LocalDate.parse("2000-01-02"));
    Student s1 = new Student("John", LocalDate.parse("2000-03-04"));
    Person poly = new Student("Polly", LocalDate.parse("2000-05-06"));

    System.out.println(p1 instanceof Student);
    System.out.println(s1 instanceof Student);
    System.out.println(poly instanceof Student);
}
\end{javalst}
\end{quote}

The output of the program is:

\begin{quote}
\begin{verbatim}
false
true
true
\end{verbatim}
\end{quote}


\quest{30 min}


\Q Complete the table below based on the source code:

\setlength{\defaultwidth}{6em}

\begin{center}
\begin{tabular}{c|c|c}
\tr Variable & \tr Type of Variable & \tr Type of Object \\
\hline
\java{p1}    & \ans{Person}         & \ans{Person}       \\
\hline
\java{s1}    & \ans{Student}        & \ans{Student}      \\
\hline
\java{poly}  & \ans{Person}         & \ans{Student}      \\
\end{tabular}
\end{center}


\Q Is the \java{instanceof} operator based on the variable's type or object's type?
Justify your answer with a specific example from the program.

\begin{answer}[3em]
It's based on the object's type.
The 3rd line of output is true, because poly is a Student, even though it's declared as a Person.
\end{answer}


\Q Predict the result of the following expressions.
Then run the code on a computer to check your answers.
\note{Note: \jans{instanceof Object} is always true.}

\setlength{\defaultwidth}{4em}

\begin{multicols}{2}
\begin{enumerate}
\item \java{p1 instanceof Person} \ans{true}

\item \java{p1 instanceof Object} \ans{true}

\item \java{s1 instanceof Person} \ans{true}

\item \java{s1 instanceof LocalDate} \ans{false}

\item \java{poly instanceof Person} \ans{true}

\item \java{poly instanceof Teacher} \ans{false}
\end{enumerate}
\end{multicols}


\Q Review your answer to Quesiton \ref{NotStu}.
Explain why the following statement is invalid:

\begin{javalst}
Student s2 = new Person("Chris", LocalDate.parse("2000-07-08"));
\end{javalst}

\begin{answer}[3em]
The variable's type has to be a parent of the object's type (if not the same).
Not all Person objects are Students.
\end{answer}


\Q Open \textit{Model2.java} in your editor.
Answer each question by typing the following code in \java{main} and pressing \textsf{Ctrl+Space} to list possible completions.

\begin{enumerate}
\item Which methods can be called on the \java{s1} variable? \hspace{2em} \java{s1.}
\begin{answer}[1em]
All the methods of Object, Person, and Student.
\end{answer}

\item Which methods can be called on the \java{poly} variable? \hspace{1em} \java{poly.}
\begin{answer}[1em]
Only the methods of Object and Person.
\end{answer}
\end{enumerate}

\vspace{-2em}


\Q Identify a method that is only in the \java{Student} class (and not the \java{Person} class).

\setlength{\defaultwidth}{15em}

\begin{enumerate}
\item Which method did you choose? \ans{getSid, isTA, setSid, or setTA}

\item Write code that calls that method on \java{poly}:
\ans{\jans{poly.setSid(1234);}}

\item What happens when you try to run that code on a computer?
\begin{answer}[2em]
Compiler error: ``The method setSid(int) is undefined for the type Person''
\end{answer}

\item Are the methods that you can call based on the variable's or object's type?
\hfill \ans[6em]{variable's}
\end{enumerate}


\Q Sometimes you need to call a method from the object's class, even though the variable was declared as a different type.
Using your example from the previous question, do the following:

\begin{enumerate}
\item Write an if-statement that checks if a \java{Person} variable ``is a'' \java{Student} object.
\begin{answer}[1em]
\jans{if (poly instanceof Student) \{}
\end{answer}

\item Inside the if-statement block, declare a new variable of type \java{Student}.
Type-cast the \java{Person} variable, and assign the result to the  \java{Student} variable.

\begin{answer}[1em]
\jans{Student stu = (Student) poly;}
\end{answer}

\item Call the \java{Student} method on this new variable.

\begin{answer}[1em]
\jans{stu.setSid(1234);}
\end{answer}
\end{enumerate}


\Q Where in the source code of \textit{Person.java} do you see this 3-step pattern?

\begin{answer}[1em]
In the \java{equals} method.
\end{answer}


\Q \label{key2}
In general, explain why the first two steps (the if statement and type cast) are needed.

\begin{answer}[3em]
The first step is needed to avoid getting a \jans{ClassCastException}.
The second step is needed in order to access fields/methods of the subclass.
\end{answer}


\Q Trace the execution of the following code using a debugger:

\begin{javalst}
LocalDate d = LocalDate.parse("1949-01-17");
Object obj = new Teacher("Anita Borg", d, 123456);
System.out.println(obj.toString());
\end{javalst}

\setlength{\defaultwidth}{6em}

\vspace{-1ex}
\begin{enumerate}
\item What type of variable is \java{obj}?
\ans{Object}

\item What type of object does \java{obj} reference?
\ans{Teacher}

\item Which version of \java{toString} (in which class) is invoked first?
\ans{Teacher}

\item Which version of \java{toString} (in which class) is invoked second?
\ans{Person}
\end{enumerate}
\vspace{-1ex}


\Q Predict which \java{equals} methods will be called in the following example.
Then trace the code using a debugger to check your answer.

\begin{javalst}
Person j = new Student("John", LocalDate.parse("2000-03-04"));
Person m = new Teacher("Mary", LocalDate.parse("2000-09-10"), 100000);
System.out.println(j.equals(m));
System.out.println(m.equals(j));
\end{javalst}

\begin{answer}
In the first \jans{println}, \jans{Student.equals} is called because \jans{j} is a \jans{Student}.
(Then \jans{Person.equals} is called via \jans{super}.)
In the second \jans{println}, \jans{Person.equals} is called because \jans{m} is a \jans{Teacher}, which inherits \jans{equals} from \jans{Person}.
\end{answer}


\Q \label{key3}
Discuss the following questions. Justify your answers using examples from today's activity.

\begin{enumerate}
\item Does the variable's type or object's type determine which methods can be called?

\begin{answer}[3em]
The variable's type.
When the compiler sees a variable, it makes sure the methods are defined in its class (or parent's class).
\end{answer}

\item Does the variable's type or object's type determine which method is actually called?

\begin{answer}[3em]
The object's type.
This behavior is desired when methods are overloaded: we want to call the method specific to that object.
\end{answer}
\end{enumerate}
