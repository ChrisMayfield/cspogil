\model{Abstract Methods}

The \java{abstract} keyword can be used to declare methods that have no body.
%These methods must be overridden in subclasses.
Classes with abstract methods must also be defined as abstract.

\begin{quote}
\begin{javalst}
public abstract class LoudToy {
    private int volume;

    public LoudToy(int volume) {
        this.volume = volume;
    }

    public int getVolume() {
        return volume;
    }

    public void setVolume(int volume) {
        this.volume = volume;
        makeNoise();
    }

    public abstract void makeNoise();
}
\end{javalst}
\end{quote}


\quest{15 min}


\Q Summarize the differences between \ref{\currfilename} and your answer to \ref{LoudToyV1}.

\begin{answer}
The class and the \java{makeNoise} method are declared as abstract.
The definition of \java{makeNoise} ends with a semicolon, rather than an empty body \verb|{}|.
\end{answer}


\Q Open \textit{LoudToy.java} (from \ref{\currfilename}) in your IDE.
Remove the word \java{abstract} from the class definition.
What are the two compiler errors?

\begin{answer}
The type LoudToy must be an abstract class to define abstract methods. \\[1ex]
The abstract method makeNoise in type LoudToy can only be defined by an abstract class.
\end{answer}


\Q Replace the word \java{abstract} in the class definition, and then remove the word \java{abstract} from the method definition.
What is the compiler error now?

\begin{answer}
This method requires a body instead of a semicolon.
\end{answer}


\Q Remove the definition of \java{makeNoise} altogether, and notice the compiler error.
Why is it necessary to declare this method in \java{LoudToy}?

\begin{answer}[3em]
The \java{setVolume} method calls the \java{makeNoise} method.
\end{answer}


\Q Undo all changes in \textit{LoudToy.java}, and add the following \java{main} method.
What is the compiler error message?
Why do you think Java doesn't allow you to construct a \java{LoudToy}?

\begin{javalst}
public static void main(String[] args) {
    LoudToy toy1 = new LoudToy(1);
    toy1.makeNoise();
}
\end{javalst}

\begin{answer}
The compiler says, ``Cannot instantiate the type LoudToy.''
Abstract classes cannot be instantiated, because some of their methods aren't implemented.
\end{answer}


\Q Open \textit{ToySheep.java} and rename \java{makeNoise} to \java{makeNoise2}.
What is the compiler error?

\begin{answer}[3em]
The type ToySheep must implement the inherited abstract method LoudToy.makeNoise().
\end{answer}


\Q Rename the method back to \java{makeNoise}, but change \java{void} to \java{int}.
What is the error now?

\begin{answer}[3em]
The return type is incompatible with LoudToy.makeNoise().
\end{answer}


\Q \label{key2}
Explain how an abstract method is like a contract.

\begin{answer}[5em]
If you inherit an abstract class, you must override the abstract methods exactly as defined.
This is important because they might be called in the code of the abstract class.
\end{answer}
