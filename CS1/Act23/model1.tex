\model{Set of Strings}

Type each line of code below in \textit{JShell}, \emph{one at a time}, and record the results.
You only need to record the output to the right of the ``\java{==>}'' symbol.
For example, if \textit{JShell} outputs \verb|$8 ==> true|, then just write \verb|true|.
If an error occurs, record the error message.

\setlength{\defaultwidth}{20.8em}

\begin{center}
\begin{tabular}{|l|p{21em}|}
\hline
\multicolumn{1}{|c|}{\tr Java code} &
\multicolumn{1}{ c|}{\tr Shell output}
\\ \hline

\java{Set<String> names = new Set<>();}
& \ans{java.util.Set is abstract; cannot be instantiated}
\\ %\hline

\java{Set<String> names = new HashSet<>();}
& \ans{[]}
\\ \hline

\java{names.add("WAS")}
& \ans{true}
\\ %\hline

\java{names.add("BAL")}
& \ans{true}
\\ %\hline

\java{names.add("PHI")}
& \ans{true}
\\ %\hline

\java{names}
& \ans{[PHI, WAS, BAL]}
\\ \hline

\java{names.contains("DEN")}
& \ans{false}
\\ %\hline

\java{names.add("DEN")}
& \ans{true}
\\ %\hline

\java{names.contains("DEN")}
& \ans{true}
\\ %\hline

\java{names.contains("den")}
& \ans{false}
\\ \hline

\java{names.add("DEN")}
& \ans{false}
\\ %\hline

\java{names.add(123)}
& \ans{int cannot be converted to java.lang.String}
\\ %\hline

\java{names.size()}
& \ans{4}
\\ %\hline

\java{names}
& \ans{[PHI, WAS, DEN, BAL]}
\\ \hline

\java{names.remove("WAS")}
& \ans{true}
\\ %\hline

\java{names.remove("IND")}
& \ans{false}
\\ %\hline

\java{names}
& \ans{[PHI, DEN, BAL]}
\\ \hline

\java{names.isEmpty()}
& \ans{false}
\\ %\hline

\java{names.clear()}
& \ans{}
\\ %\hline

\java{names.size()}
& \ans{0}
\\ %\hline

\java{names.isEmpty()}
& \ans{true}
\\ \hline

\end{tabular}
\end{center}


\quest{20 min}


\Q For the collection above:

\setlength{\defaultwidth}{5em}

\begin{multicols}{2}
\begin{enumerate}
\item What is the interface name? \ans{Set}
\item What is the class name? \ans{HashSet}
\item What is the variable name? \ans{names}
\item What is the element type? \ans{String}
\end{enumerate}
\end{multicols}


\Q Based on the shell output, describe what the following methods return:

\setlength{\defaultwidth}{32em}

\begin{enumerate}
\item \java{add} ~\ans{true if this set did not already contain the specified element}
\item \java{remove} ~\ans{true if this set contained the specified element}
\end{enumerate}


\Q Consider the contents of \java{names} just before \java{"WAS"} was removed.

\setlength{\defaultwidth}{3em}

\begin{enumerate}
\item What was the size of \java{names} at this point? \ans{4}
\item How many times was the \java{add} method called? \ans{6}
\item Explain why these two numbers are different.
\begin{answer}[2em]
DEN was added a second time, but it was already in the set. \\
123 could not be added, because its type didn't match.
\end{answer}
\vspace{-1ex}
\end{enumerate}


\Q Continuing the previous question:

\setlength{\defaultwidth}{15em}

\begin{enumerate}
\item In what order were the strings added to the set? \ans{WAS, BAL, PHI, DEN}
\item In what order were they displayed in the output? \ans{PHI, WAS, DEN, BAL}
\item Why do you think the two orders are different?
\begin{answer}[2em]
Sets have no defined order
\end{answer}
\vspace{-1ex}
\end{enumerate}


\Q \label{key1}
In your own words, summarize what a \java{Set} is in Java.
Give an example from everyday life.

\begin{answer}
The \java{java.util.Set} interface models the mathematical notion of a \emph{set}, i.e., a collection that has no duplicates.
For example, you could have the set of all computer science majors.
\end{answer}


\Q In discrete mathematics, sets have three basic operations:

\begin{itemize}[itemsep=2pt]
\item Union ($S \cup T$) : all elements in $S$ or $T$ (or both)
\item Intersection ($S \cap T$) : elements in both $S$ and $T$
\item Difference ($S - T$) : elements in $S$ but not in $T$
\end{itemize}

Based on the \href{https://docs.oracle.com/en/java/javase/11/docs/api/java.base/java/util/Set.html}{documentation} for \java{java.util.Set}, which methods implement these operations?

\begin{answer}
\begin{javaans}
set1.addAll(set2);    // Union
set1.retainAll(set2); // Intersection
set1.removeAll(set2); // Difference
\end{javaans}
\end{answer}
