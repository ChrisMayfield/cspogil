\model{Map of Team Names}

The following abbreviations are for National Football League (NFL) teams:

\begin{center}
\begin{tabular}{l|l}
ATL & Atlanta Falcons \\ \hline
DEN & Denver Broncos \\ \hline
IND & Indianapolis Colts \\ \hline
MIA & Miami Dolphins \\ \hline
SEA & Seattle Seahawks \\
\end{tabular}
\end{center}

Complete the table below using \textit{JShell} (the same way you did for \ref{model1.tex}).

\setlength{\defaultwidth}{20.8em}

\begin{center}
\begin{tabular}{|l|p{21em}|}
\hline
\multicolumn{1}{|c|}{\tr Java code} &
\multicolumn{1}{ c|}{\tr Shell output}
\\ \hline

\java{Map<String, String> teams;}
& \ans{null}
\\ %\hline

\java{teams = new Map<>();}
& \ans{java.util.Map is abstract; cannot be instantiated}
\\ %\hline

\java{teams = new HashMap<>();}
& \ans{\{\}}
\\ %\hline

\java{teams.isEmpty()}
& \ans{true}
\\ \hline

\java{teams.put("MIA", "Miami Dolphins")}
& \ans{null}
\\ %\hline

\java{teams.put("MIA", "Miami")}
& \ans{"Miami Dolphins"}
\\ %\hline

\java{teams.size()}
& \ans{1}
\\ %\hline

\java{teams}
& \ans{\{MIA=Miami\}}
\\ \hline

\java{teams.put("ATL", "Atlanta")}
& \ans{null}
\\ %\hline

\java{teams.put("SEA", "Seattle")}
& \ans{null}
\\ %\hline

\java{teams}
& \ans{\{MIA=Miami, ATL=Atlanta, SEA=Seattle\}}
\\ \hline

\java{teams.containsKey("ATL")}
& \ans{true}
\\ %\hline

\java{teams.containsKey("DEN")}
& \ans{false}
\\ %\hline

\java{teams.containsValue("Miami")}
& \ans{true}
\\ %\hline

\java{teams.containsValue("Dolphins")}
& \ans{false}
\\ \hline

\java{teams.get("SEA")}
& \ans{"Seattle"}
\\ %\hline

\java{teams.get("IND")}
& \ans{null}
\\ %\hline

\java{teams.get(0)}
& \ans{null}
\\ \hline

\java{teams.remove("MIA")}
& \ans{"Miami"}
\\ %\hline

\java{teams.remove("MIA")}
& \ans{null}
\\ %\hline

\java{teams}
& \ans{\{ATL=Atlanta, SEA=Seattle\}}
\\ \hline

\java{teams.keySet()}
& \ans{[ATL, SEA]}
\\ %\hline

\java{teams.values()}
& \ans{[Atlanta, Seattle]}
\\ \hline

\end{tabular}
\end{center}


\quest{25 min}


\Q For the collection above:

\setlength{\defaultwidth}{5em}

\begin{multicols}{2}
\begin{enumerate}
\item What is the interface? \ans{Map}
\item What is the class? \ans{HashMap}
\item What type of keys? \ans{String}
\item What type of values? \ans{String}
\end{enumerate}
\end{multicols}


\Q Based on the shell output, describe what the following methods return:

\setlength{\defaultwidth}{32em}

\begin{enumerate}
\item \java{put} ~\ans{The previous value associated with the key, or null if not mapped.}
\item \java{get} ~\ans{The value to which the specified key is mapped, or null if none.}
\end{enumerate}


\Q What type of object does the \java{keySet} method return? Describe its contents.

\begin{answer}[3em]
In this example, it returns a \java{Set<String>} containing all the abbreviations.
\end{answer}


\Q What type of object does the \java{values} method return? Describe its contents.

\begin{answer}[3em]
In this example, it returns a \java{Collection<String>} containing all the team names.
\end{answer}


\Q \label{key2}
In your own words, summarize what a \java{Map} is in Java.
Give an example from everyday life.

\begin{answer}
An object that ``maps'' keys with values.
A map cannot contain duplicate keys; each key can map to at most one value.
For example, you could maps English words to their definitions.
\end{answer}


\Q Why did \java{teams.get(0)} return null, even though there were values in the map?

\begin{answer}
You cannot use ``indexes'' to access values in a map; only keys.
There is no value mapped to the key of 0.
Besides, keys in the \java{teams} map need to be strings.
\end{answer}


\Q \label{key3}
Write Java code that defines a map named \java{dow} that represents the seven days of the week as follows: Sun=1, Mon=2, Tue=3, etc.
Run your code in \textit{JShell} to make sure it works.

\begin{answer}[12em]
\begin{javaans}
Map<String, Integer> dow = new HashMap<>();
dow.put("Sun", 1);
dow.put("Mon", 2);
dow.put("Tue", 3);
dow.put("Wed", 4);
dow.put("Thu", 5);
dow.put("Fri", 6);
dow.put("Sat", 7);
\end{javaans}
\end{answer}


\Q Print the \java{dow} variable in \textit{JShell}.
What do you notice about the order of its contents?

\begin{answer}
The contents appear to be listed in a random order:

\medskip
%\jans{\{Monday=2, Thursday=5, Friday=6, Sunday=1, Wednesday=4, Tuesday=3, Saturday=7\}}
\jans{\{Thu=5, Tue=3, Wed=4, Sat=7, Fri=6, Sun=1, Mon=2\}}
\end{answer}
