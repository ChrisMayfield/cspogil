\model{Case Study: Panic Attack}

Frank was behind in his programming assignment.
He approached Martin to see if he could get some help.
But he was so far behind and so confused that Martin just gave him his code with the intent that he would ``just look at it to get some ideas.''

\vspace{1em}

In the paraphrased words of Frank: ``I started the assignment three days after you put it up.
But then other assignments came in and I started on them too.
I felt like I was chasing rabbits and began to panic.
It was already past the due date and I got really scared.
That's when I went to Martin to see if he could help.''
Frank copied much of the code and turned it in as his own.


\quest{7.5 min}


\Q Which, if any, of the students were at fault? Why?

\begin{answer}[6em]
Both students are at fault, assuming the assignment was supposed to be done individually.
It's not okay to look at someone else's code and/or give your code to someone else.
\end{answer}


\Q Which specific Honor Code violations occurred?

\begin{answer}[7em]
[Frank]~ Collaborating in an unauthorized manner with one or more students on any work submitted for academic credit.

\bigskip

[Martin]~ Rendering unauthorized assistance to another student by knowingly permitting him to see a portion of work to be submitted for academic credit.
\end{answer}


\Q What should Martin have done in this situation?

\begin{answer}[6em]
This type of situation often occurs when students feel pressure before a deadline.
Martin should have refused to show his code to Frank and reminded him to get help from the instructor and/or a TA.
\end{answer}


\Q What options did Frank have besides cheating?

\begin{answer}[6em]
Frank could have asked the instructor for help during office hours, met with a TA during lab hours, posted a question on Piazza, or asked a question during class.
It's much easier to get help when it's not the night before.
\end{answer}
