\model{Case Study: Oops!}

Emily was working in the lab on her programming assignment.
She finished the program, submitted it, and went on to do some other work.
Shortly thereafter, she left the lab.

\vspace{1em}

Another student, Kyle, was working nearby.
He knew that Emily had successfully submitted the assignment, and he had not been able to get his to work properly.
When Emily left, he noticed that she had not logged out of her computer.
He moved to her workstation, found the work under her Documents directory, and copied it onto his flash drive.
He then logged out, logged in as himself, and copied the code onto his Desktop where he modified the program a bit, then successfully submitted it.


\quest{10 min}


\Q Which, if any, of the students were at fault? Why?

\begin{answer}[6em]
Kyle is certainly at fault, and depending on the lab policy, Emily may also be penalized.
But in general, students are not held accountable when files are stolen without their knowledge.
\end{answer}


\Q Which specific Honor Code violations occurred?

\begin{answer}[8em]
[Kyle]~ Committing the act of plagiarism: copying information, ideas, or phrasing of another person without proper acknowledgment of the true source.

\bigskip

[Kyle]~ Using computing facilities in an academically dishonest manner.
\end{answer}


\Q What should Emily have done in this situation?

\begin{answer}[6em]
To avoid this situation, Emily should have logged out or at least locked her screen when leaving the room.
\end{answer}


\Q What options did Kyle have besides cheating?

\begin{answer}[6em]
Kyle could have asked the instructor for help during class or office hours, met with a TA during lab hours, or posted a question on Piazza.
\end{answer}
