\model{Array Diagrams}

Array elements are stored together in one contiguous block of memory. To show arrays in memory diagrams, we simply draw adjacent boxes.

\begin{center}
\java{int[] nums = {10, 3, 7, -5};}

\vspace{1ex}
\includegraphics[width=225pt]{CS1/array-diagram1.png}
\end{center}


\quest{15 min}


\Q Draw a memory diagram for the following array declarations.

\begin{enumerate}

\item
\begin{javalst}
int[] sizes = new int[5];
sizes[2] = 7;
\end{javalst}

\item
\begin{javalst}
char[] codes = new char[3];
codes[2] = 'X';
\end{javalst}

\item
\begin{javalst}
double[] costs = new double[4];
costs[0] = 0.99;
\end{javalst}

\end{enumerate}


\Q What is the \emph{default} value for uninitialized array elements? (Hint: You should have no empty boxes in your memory diagram.)

\begin{answer}
\end{answer}


\Q Like strings, arrays are reference types. What is the \emph{value} of an array variable?

\begin{answer}
\end{answer}


\Q Draw a memory diagram of the following array.
(Hint: You should have four arrows.)

\begin{javalst}
String[] greek = {"alpha", "beta", "gamma"};
\end{javalst}

\begin{answer}
\end{answer}
