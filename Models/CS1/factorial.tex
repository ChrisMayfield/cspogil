\model{Factorial Function}

''In mathematics, the factorial of a non-negative integer $n$, denoted by $n!$, is the product of all positive integers less than or equal to $n$. For example, $5! = 5 \times 4 \times 3 \times 2 \times 1 = 120$.''

\smallskip\hfill
Source: \url{https://en.wikipedia.org/wiki/Factorial}

\begin{center}
\begin{tabular}{|C{1cm}|C{1cm}|}
\hline
\tr n & \tr n! \\
\hline
1 & 1 \\
\hline
2 & 2 \\
\hline
3 & 6 \\
\hline
4 & 24 \\
\hline
5 & 120 \\
\hline
6 & 720 \\
\hline
\end{tabular}
\end{center}


\quest{15 min}


\Q \label{fact4} Consider two different ways to show how to calculate 4!

\begin{enumerate}
\item Write out all numbers that need to be multiplied

4! = \ans{4 * 3 * 2 * 1}

\item Rewrite the expression using 3!

4! = \ans{4 * 3!}
\end{enumerate}


\Q Write an expression similar to \ref{fact4}b showing how each factorial can be calculated in terms of a simpler factorial.

\begin{enumerate}
\item 3! = \ans{3 * 2!}
\item 2! = \ans{2 * 1!}
\item 100! = \ans{100 * 99!}
\item $n!$ = \ans{$n * (n - 1)!$}
\end{enumerate}


\Q What is the value of 1! based on the model? Does it make sense to define 1! in terms of a simpler factorial?
