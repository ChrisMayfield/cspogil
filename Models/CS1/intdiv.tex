% Based on Model 3 of "Activity 01 Operators" by Helen Hu

\model{Dividing Numbers}

\vspace{-1em}
\begin{center}
\renewcommand{\arraystretch}{1.5}
\begin{tabular}[t]{|C{35pt}|C{65pt}|C{35pt}|}
\hline
 9 / 4 & \textit{evaluates to} & 2 \\
\hline
10 / 4 & \textit{evaluates to} & 2 \\
\hline
11 / 4 & \textit{evaluates to} & 2 \\
\hline
12 / 4 & \textit{evaluates to} & 3 \\
\hline
13 / 4 & \textit{evaluates to} & 3 \\
\hline
14 / 4 & \textit{evaluates to} & 3 \\
\hline
15 / 4 & \textit{evaluates to} & 3 \\
\hline
16 / 4 & \textit{evaluates to} & 4 \\
\hline
\end{tabular}
\hspace{0.5in}
\begin{tabular}[t]{|C{45pt}|C{65pt}|C{45pt}|}
\hline
9.0 / 4.0 & \textit{evaluates to} & 2.25 \\
\hline
10. / 4.  & \textit{evaluates to} & 2.5 \\
\hline
11  / 4.0 & \textit{evaluates to} & 2.75 \\
\hline
12. / 4   & \textit{evaluates to} & 3.0 \\
\hline
13  / 4.0 & \textit{evaluates to} & 3.25 \\
\hline
\end{tabular}
\end{center}


\quest{15 min}

\Q In the first table, which number(s) divided by 4 evaluate to 3? What is significant about the number of answers you have written down?

\begin{answer}
\end{answer}


\Q How do the answers in the first table differ from the mathematically correct answers?

\begin{answer}
\end{answer}


\Q To the right of the second table, round each answer to the closest integer. How do those values compare to what you see in the first table?

\begin{answer}
\end{answer}


\Q Carefully explain the difference between the numbers in the first and second tables.

\begin{answer}
\end{answer}


\Q Complete the table:

\begin{center}
\renewcommand{\arraystretch}{1.5}
\begin{tabular}[t]{|C{45pt}|C{65pt}|C{45pt}|}
\hline
14. / 4. & \textit{evaluates to} & 3.5 \\
\hline
14. / 4  & \textit{evaluates to} & 3.5 \\
\hline
14  / 4. & \textit{evaluates to} & 3.5 \\
\hline
14  / 4  & \textit{evaluates to} & 3 \\
\hline
\end{tabular}
\end{center}


\Q Dividing numbers with fractional parts (known as \textbf{floating-point} numbers) gives you different results from dividing two integers. In the previous question:

\begin{enumerate}
\item Which rows evaluate to an integer? \ans{the first three}
\vspace{1ex}
\item Which rows evaluate to a floating-point number? \ans{the last one}
\end{enumerate}


\Q Imagine you are writing a Java program that requires division.

\begin{enumerate}
\item What must be true about the \textbf{operands} (the numbers around the \textbf{operators}) for you to get the mathematically correct answer?

\vspace{1ex}
\ans{At least one of them needs to be a floating-point number.}
\vspace{1ex}

\item Does it need to be true for both operands? \ans{No}
\vspace{1ex}
\end{enumerate}


\Q Consider what you know about addition. If you add two integers in a Java expression, will the result always be mathematically correct? Justify your answer.

\begin{answer}
\end{answer}


\Q What about subtraction and multiplication? If you subtract or multiply two integers, will the answer always be mathematically correct? Justify your answer.

\begin{answer}
\end{answer}
