% Based on Model 2 of "Activity 01 Operators" by Helen Hu

\model{The \% Operator}

\vspace{-1ex}
\begin{center}
\renewcommand{\arraystretch}{1.4}
\begin{tabular}[t]{|C{35pt}|C{65pt}|C{35pt}|}
\hline
 9 / 4 & \textit{evaluates to} & 2 \\
\hline
10 / 4 & \textit{evaluates to} & 2 \\
\hline
11 / 4 & \textit{evaluates to} & 2 \\
\hline
12 / 4 & \textit{evaluates to} & 3 \\
\hline
13 / 4 & \textit{evaluates to} & 3 \\
\hline
14 / 4 & \textit{evaluates to} & 3 \\
\hline
15 / 4 & \textit{evaluates to} & 3 \\
\hline
16 / 4 & \textit{evaluates to} & 4 \\
\hline
\end{tabular}
\hspace{0.5in}
\begin{tabular}[t]{|C{35pt}|C{65pt}|C{35pt}|}
\hline
 9 \% 4 & \textit{evaluates to} & 1 \\
\hline
10 \% 4 & \textit{evaluates to} & 2 \\
\hline
11 \% 4 & \textit{evaluates to} & 3 \\
\hline
12 \% 4 & \textit{evaluates to} & 0 \\
\hline
13 \% 4 & \textit{evaluates to} & 1 \\
\hline
14 \% 4 & \textit{evaluates to} & 2 \\
\hline
15 \% 4 & \textit{evaluates to} & 3 \\
\hline
16 \% 4 & \textit{evaluates to} & 0 \\
\hline
\end{tabular}
\end{center}


\quest{10 min}


\Q Which numbers \% 4 evaluate to 0 in the table above? If the table were extended to include more rows, which other numbers \% 4 would evaluate to 0?

\begin{answer}
\end{answer}


\Q Look at the expressions in the second table that evaluate to 1. How do the left operands in these expressions (9, 13, 17) differ from those that evaluate to 0?

\begin{answer}
\end{answer}


\Q List three numbers \% 5 that will evaluate to 0 and three numbers \% 5 that will evaluate to 2.

\begin{answer}
\end{answer}


\Q Evaluate the following Java expressions:

\begin{center}
\begin{tabular}{C{1in}C{1in}C{1in}C{1in}}
18 \% 4  &  19 \% 4   &  19 \% 5  &  19 \% 6  \\
\vspace{1ex}
\ans{2}  & \ans{3}    & \ans{4}   & \ans{1}   \\
\end{tabular}
\end{center}


\Q Describe what the \% operator does. How are the / and \% operators related?

\begin{answer}

\end{answer}


\Q Would it make sense to apply the \% operator to floating-point numbers?
Why or why not?

\begin{answer}
Not really, since the concept of a remainder is based on integer arithmetic.

(Floating-point modulus is defined by the IEEE 754 standard, but its use is rare.)
\end{answer}


%TODO introduce the term modulus when reporting out
