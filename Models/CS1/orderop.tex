\model{Order of Operations}

The Java language defines a specific order of execution for math and other operations. For example, multiplication and division take \textbf{precedence} over addition and subtraction. Using parentheses, you can override the order of operations. The following table lists some Java operators from highest precedence to lowest precedence.

\begin{center}
\renewcommand{\arraystretch}{1.5}
\begin{tabular}{|L{3in}|L{1in}|}
\hline
Parenthesis
& \tt ( ) \\
\hline
Unary (positive or negative signs)
& \tt + - \\
\hline
Multiplicative
& \tt * / \\
\hline
Additive
& \tt + - \\
\hline
Assignment
& \tt = \\
\hline
\end{tabular}
\end{center}

For the following questions, assume you have these two variables:

\begin{quote}
\begin{java}
int x;
double y;
\end{java}
\end{quote}


\quest{10 min}

\Q What operator has the lowest precedence? Why do you think Java is designed that way?

\begin{answer}
\end{answer}


\Q The \texttt{+} and \texttt{-} operators show up twice in the above list of operator precedence. In the Java expression ~ \texttt{x = 5 * -3;} ~ explain how you know whether the \texttt{-} operator is being used as an unary or binary operator in this expression.

\begin{answer}
\end{answer}


\Q Give the order of operations in the Java expression: ~ \texttt{x = 5 * -3;}

\begin{enumerate}
\item First operator to be executed:
\item Second operator:
\item Third operator:
\end{enumerate}



\Q Give the order of operations in the Java expression: ~ \texttt{y = 9 / 2;}

\begin{enumerate}
\item First operator to be executed:
\item Second operator:
\end{enumerate}


\Q Based on your answer to the previous question, explain why the variable y will be changed to 4.0 (and not 4 or 4.5).

\begin{answer}
\end{answer}


\Q Rewrite the assignment for y so that it will be set correctly to 4.5. (Hint: you'll need to recall what you learned about division in \ref{CS1/intdiv}.)

\begin{answer}
\end{answer}
