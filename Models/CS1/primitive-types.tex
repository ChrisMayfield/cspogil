\model{Primitive Types}
\label{CS1/primitive-types}

\vspace{-1ex}
\begin{table}[h!]
\begin{tabularx}{\linewidth}{|X|X|X|X|}
\hline
\tr Keyword    & \tr Size & \tr Min Value & \tr Max Value \\
\hline
\java{byte}    & 1 byte   & $-128$    & $127$ \\
\hline
\java{short}   & 2 bytes  & $-32,768$ & $32,767$ \\
\hline
\java{int}     & 4 bytes  & $-2^{31}$ & $2^{31}-1$ \\
\hline
\java{long}    & 8 bytes  & $-2^{63}$ & $2^{63}-1$ \\
\hline
\java{float}   & 4 bytes  & $\pm 3.4 \times 10^{-38}$  & $\pm 3.4 \times 10^{38}$ \\
\hline
\java{double}  & 8 bytes  & $\pm 1.7 \times 10^{-308}$ & $\pm 1.7 \times 10^{308}$ \\
\hline
\java{boolean} & N/A      & \java{false}     & \java{true} \\
\hline
\java{char}    & 2 bytes  & \java{'\\u0000'} & \java{'\\uffff'} \\
\hline
\end{tabularx}
\end{table}

Note that 1 byte is 8 bits, i.e., eight ``ones and zeros'' in computer memory.
Since there are only two options for each bit, with 8 bits you can represent $2^8 = 256$ possible values.


\quest{10 min}


\Q Which of the primitive types are integers? Which are floating-point? Which are not numeric?

\begin{answer}
\end{answer}


\Q Why do primitive types have ranges of values? What determines the range of the data type?

\begin{answer}
\end{answer}


\Q Why can't computers represent every possible number in mathematics? Will they ever be able to do so?

\begin{answer}
\end{answer}


\Q Since a \java{byte} can represent 256 different numbers, why is its max value 127 and not 128?

\begin{answer}
\end{answer}


\Q What is the data type for each of the following values?

\begin{quote}
\begin{multicols}{2}
1.14159 \ans{double} \\[1ex]
0       \ans{int} \\[1ex]
-1.0F   \ans{float} \\[1ex]
123     \ans{int}

7.2E-4  \ans{double} \\[1ex]
0.0     \ans{double} \\[1ex]
-13L    \ans{long} \\[1ex]
'0'     \ans{char}
\end{multicols}
\end{quote}


\Q Given the following variable declarations, which of the assignments are not allowed?

\begin{quote}
\begin{multicols}{2}

\begin{javalst}
byte miles;
short minutes;
int checking;
long days;
float total;
double sum;
boolean flag;
char letter;
\end{javalst}

\begin{javalst}
checking = 56000;
total = 0;
sum = total;
total = sum;
checking = miles;
sum = checking;
sum = days;
days = "0";
\end{javalst}

\end{multicols}
\end{quote}


\Q \label{allow} In general, when does Java allow you to assign one type of numeric variable to another?

\begin{answer}
\end{answer}


\Q Based on your answer to \ref{allow}, list all possible assignments in this format:
\java{int} $\gets$ \java{short}

\begin{answer}
\end{answer}
