% based on Model 1 of "Improved Scope" activity by Helen Hu

\model{Variable Scope}

It's important for programmers to be aware of a variable's \emph{scope}: where the variable is accessible and visible. In Java, we have four levels of scope:

\newsavebox{\localVars}
\begin{lrbox}{\localVars}
\begin{javalst}
for (int i = 0; i<5; i++) {
    System.out.println(i);
}
// i is not valid here
\end{javalst}
\end{lrbox}

\begin{center}
\begin{tabularx}{\textwidth}{|X|X|X|X|}
\hline
\tr &
\tr \textbf{Where declared?} &
\tr \textbf{Where used?} &
\tr \textbf{Example} \\
\hline
\textbf{static variables} \par (``class variables'') &
declared outside of all methods (typically at start of file) &
anywhere in class &
\java{PI} from the \java{Math} class \\
\hline
\textbf{instance variables} \par (``attributes'') &
declared outside of all methods (typically at start of file) &
any non-static method, only in static methods if another object has been created (this) &
\java{radius} from the \java{Circle} class \\
\hline
\textbf{parameters} &
declared inside ()'s in method header &
anywhere within method where they are declared &
\java{radius} in \java{Circle.setRadius} \\
\hline
\textbf{local variables} &
declared inside the method (possibly inside a code block) &
anywhere within method or code block after they are declared &
\usebox{\localVars} \\
\hline
\end{tabularx}
\end{center}

\quest{15 min}


\Q TODO

\begin{answer}
\end{answer}
