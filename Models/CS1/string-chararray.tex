\model{Character Arrays}

The primitive type \java{char} is used to store a single \emph{character}, which can be a letter, a number, or a symbol.
In contrast, the reference type \java{java.lang.String} represents a sequence of characters.

\begin{quote}
\begin{javalst}
char letter;           String word;
letter = 'a';          word = "food";
\end{javalst}
\end{quote}

Internally, a string is stored as an \emph{array} of characters.
Each character is indexed by a position starting at zero:

\begin{quote}
\begin{tabular}{cccc}
\hline
\multicolumn{1}{|c|}{\java{'f'}} &
\multicolumn{1}{ c|}{\java{'o'}} &
\multicolumn{1}{ c|}{\java{'o'}} &
\multicolumn{1}{ c|}{\java{'d'}} \\
\hline
\fs 0 & \fs 1 & \fs 2 & \fs 3 \\
\end{tabular}
\end{quote}


\quest{15 min}


\Q How is the syntax of character literals and string literals different?

\begin{answer}[3em]
\end{answer}


\Q Why can you use the \java{String} class in Java programs without having to import it first?

\begin{answer}[3em]
\end{answer}


\Q What is the index of \java{'d'} in the string above?
In general, what is the index of the last character of a string?

\begin{answer}[3em]
\end{answer}


\Q Sketch the underlying array for the string \java{"hello world"} (with indexes as shown above).

\begin{answer}
\end{answer}


\Q What is the \emph{value} of a \java{char} variable? What is the \emph{value} of a \java{String} variable?

\begin{answer}
\end{answer}


\Q Draw a memory diagram for the given code. Each variable should be a name next to a box containing its \emph{value}.

\begin{javalst}
String str;
str = "Hi!";

char let;
let = 'X';

short num;
num = -17;

int foo;
foo = num;

String hmm;
hmm = str;
\end{javalst}


\Q Recall that the == operator compares the value of two variables. What does it mean for two \java{char} variables to be ==? What does it mean for two \java{String} variables to be ==?

\begin{answer}
\end{answer}


\Q To compare strings (and other objects), you should use either the equals or compareTo method. Predict the output of the following code.

\begin{javalst}
String name1 = "Mark";    String name2 = "Ma" + "rk";    String name3 = "Mary";

// compare name1 and name2
if (name1 == name2) {
    System.out.println("name1 and name2 are identical");
} else {
    System.out.println("name1 and name2 are NOT identical");
}
// compare "Mark" and "Mark"
if (name1.equals(name2)) {
    System.out.println("name1 and name2 are equal");
} else {
    System.out.println("name1 and name2 are NOT equal");
}
// compare "Mark" and "Mary"
if (name1.equals(name3)) {
    System.out.println("name1 and name3 are equal");
} else {
    System.out.println("name1 and name3 are NOT equal");
}
\end{javalst}
