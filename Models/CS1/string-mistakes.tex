\model{Common Mistakes}

\newsavebox{\programA}
\begin{lrbox}{\programA}
\begin{javalst}
String greeting = "hello world";
greeting.toUpperCase();
System.out.println(greeting);

\end{javalst}
\end{lrbox}

\newsavebox{\programB}
\begin{lrbox}{\programB}
\begin{javalst}
Scanner in = new Scanner(System.in);
String line = in.nextLine();
char letter = line.charAt(1);
System.out.println(letter);
\end{javalst}
\end{lrbox}

\vspace{-1ex}
\begin{table}[h!]
\begin{tabularx}{\linewidth}{|X|X|}
\hline
\tr Program A & \tr Program B \\
\hline
\usebox{\programA} & \usebox{\programB} \\
\hline
\end{tabularx}
\end{table}


\quest{15 min}


\Q Write the output of each program in the space under the table above. What is the logic error you see when you run Program A?

\begin{answer}
\end{answer}


\Q \label{errorA} In Program A, what is returned by the string method? What happens to the return value?

\begin{answer}
\end{answer}


\Q List two different ways you can fix the logic error in \ref{errorA}.

\begin{answer}
\end{answer}


%TODO \ref to Model 2
\Q In general, what happens to \java{this} string when calling the methods in Model 2?

\begin{answer}
\end{answer}


\Q \label{errorB} In what cases will Program B crash? What is the error message displayed?

\begin{answer}
\end{answer}


\Q List two different ways you can fix the runtime error in \ref{errorB}.

\begin{answer}[3em]
\end{answer}


%\Q Solve the \java{stringYak} problem (see attached page) using \java{charAt}.


\newpage

[CodingBat] Given two strings, return the number of positions where they contain the same substring of length two. So \java{"xxcaazz"} and \java{"xxbaaz"} yields 3, since the \java{"xx"}, \java{"aa"}, and \java{"az"} substrings appear in the same place in both strings.

\medskip
\begin{javalst}
public int stringMatch(String a, String b) {
\end{javalst}

\vspace{-1em}
\begin{answer}[18em]
\begin{javalst}
    // figure which string is shorter
    int len = Math.min(a.length(), b.length());
    int count = 0;
    
    // Look at both substrings starting at i
    for (int i = 0; i < len - 1; i++) {
        String aSub = a.substring(i, i + 2);
        String bSub = b.substring(i, i + 2);
        if (aSub.equals(bSub)) {
            count++;
        }
    }
    
    return count;
\end{javalst}
\end{answer}

\begin{javalst}
}
\end{javalst}

\vfill

[CodingBat] Suppose the string \java{"yak"} is unlucky. Given a string, return a version where all the \java{"yak"} are removed, but the \java{'a'} can be any character. The \java{"yak"} strings will not overlap.

\bigskip

\java{stringYak("yakpak")} $\rightarrow$ \java{"pak"}

\java{stringYak("pakyak")} $\rightarrow$ \java{"pak"}

\java{stringYak("yak123ya")} $\rightarrow$ \java{"123ya"}

\medskip
\begin{javalst}
public String stringYak(String str) {
\end{javalst}

\vspace{-1em}
\begin{answer}[16em]
\begin{javalst}
    String result = "";
    
    for (int i = 0; i < str.length(); i++) {
        if (i + 2 < str.length() && str.charAt(i) == 'y'
                                 && str.charAt(i + 2) == 'k') {
            i = i + 2;
        } else {
            result += str.charAt(i);
        }
    }
    
    return result;
\end{javalst}
\end{answer}

\begin{javalst}
}
\end{javalst}
