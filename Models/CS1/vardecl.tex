\model{Variable Declarations}
\label{CS1/vardecl}

In addition to \textbf{literal} numbers, most Java programs will involve \textbf{variables} (named values than can be changed). The following code \textbf{declares} and \textbf{assigns} three variables:

\begin{quote}
\begin{java}
int dollars;        // line 1
int cents;
double grams;
dollars = 1;        // line 4
cents = 90;
grams = 3;
\end{java}
\end{quote}


\quest{10 min}

\Q Identify the Java \textbf{keyword} used in a variable declaration to indicate

\begin{enumerate}
\item an integer: \ans{int}
\item a floating-point number: \ans{double}
\end{enumerate}


\Q Consider numbers of dollar bills, cents, and grams. Which of these units only makes sense as an integer, and why?

\begin{answer}
\end{answer}


\Q What would you expect the following statements to print out? (Hint: Refer to \ref{CS1/intdiv}.)

\begin{enumerate}
\item \texttt{System.out.println(dollars);}
\item \texttt{System.out.println(cents);}
\item \texttt{System.out.println(grams);}
\end{enumerate}


\Q What do you think is the purpose of a variable declaration?

\begin{answer}
\end{answer}


\Q Consider the statement: ~ \texttt{cents = dollars;}

\begin{enumerate}
\item Compare this code to lines 4-6 in \ref{CS1/vardecl}.
What value do you think cents and dollars will have after running this statement?

\item Which side of the equals sign (left or right) was assigned a new value?
\end{enumerate}


\Q Examples of Java operators include \texttt{+} and \texttt{-}; they describe an operation to be executed (e.g., addition or subtraction).

\begin{enumerate}
\item Do you consider the equals sign in Java an operator (an operation to be executed)?
\\ If so, explain the operation. If not, explain why not.

\item Do you consider the equals sign in mathematics an operator (an operation to be executed)?
\\ If so, explain the operation. If not, explain why not.
\end{enumerate}


\Q In your own words, explain how you should read the \texttt{=} sign in Java.
For example, the Java statement ~ \texttt{x = a + b;} ~ should be read as ``x \_\_\_\_\_ a plus b.''

\begin{answer}
\end{answer}
