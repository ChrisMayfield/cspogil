% Based on Model 1 of "Activity 02 Declaration" by Helen Hu

\model{Variable Declarations}
\label{CS1/variable-declare}

In addition to \textbf{literal} numbers like 1 or 2.3, most Java programs involve \textbf{variables} (named values than can be changed).
The following code \textbf{declares} and \textbf{assigns} three variables:

\begin{javabox}
int dollars;
int cents;
double grams;

dollars = 1;
cents = 90;
grams = 3;
\end{javabox}


\quest{10 min}


\Q Identify the Java \textbf{keyword} used in a variable declaration to indicate

\begin{enumerate}
\item an integer: \ans{int}
\item a floating-point number: \ans{double}
\end{enumerate}


\Q Consider numbers of dollar bills, cents, and grams. Which of these units only makes sense as an integer, and why?

\begin{answer}
Cents makes sense (ha ha) only as an integer, because at the end of the day you can't pay with a fractional amount.
\end{answer}


\Q What would you expect the following statements to print out? (Hint: Refer to \ref{CS1/dividing-numbers}.)

\begin{enumerate}
\item \java{System.out.println(dollars);} \ans{1}
\item \java{System.out.println(cents);} \ans{90}
\item \java{System.out.println(grams);} \ans{3.0}
\end{enumerate}


\Q What do you think the purpose of a variable declaration is?

\begin{answer}
It tells the computer how to interpret and display the value.
\end{answer}


\Q Consider the statement: ~ \java{cents = dollars;}

\begin{enumerate}

\item Compare this code to lines 4-6 in \ref{CS1/variable-declare}.
What value do you think cents and dollars will have after running this statement?

\ans{The variable \java{cents} will be 1, and \java{dollars} will remain unchanged.}

\item Which side of the equals sign (left or right) was assigned a new value?
\ans{The left side.}

\end{enumerate}


\Q Examples of Java operators include \java{+} and \java{-}; they describe an operation to be executed (e.g., addition or subtraction).

\begin{enumerate}

\item Do you consider the equals sign in Java an operator (an operation to be executed)?
\\ If so, explain the operation. If not, explain why not.

\ans{Yes; it executes the assignment operation which stores a value in memory.}

\item Do you consider the equals sign in mathematics an operator (an operation to be executed)?
\\ If so, explain the operation. If not, explain why not.

\ans{No; it simply states the proposition that two values are equal.}

\end{enumerate}


\Q In your own words, explain how you should read the \java{=} sign in Java.
For example, the Java statement ~ \java{x = a + b;} ~ should be read as ``x \_\_\_\_\_ a plus b.''

\begin{answer}
Answers may include ``x \emph{gets} a plus b'', ``x \emph{becomes} a plus b'', etc.
\end{answer}
