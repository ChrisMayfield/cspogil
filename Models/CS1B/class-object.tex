%\item Memory diagram of Color149
%\item Memory allocation (Polygon) draw UML and memory diagram

\model{Drawing Objects}

\begin{javalst}
import java.awt.Color;

public class Polygon {
    
    public static final int MIN_SIDES = 3;
    public static final int MAX_SIDES = 100;
    
    private String name;
    private int sides;
    private Color color;
    
    public Polygon(String name, int sides, Color color) {
        this.name = name;
        this.sides = sides;
        this.color = color;
    }
    
    public String toString() {
        return this.name + " (" + this.sides + " sides)";
    }
    
    public static void main(String[] args) {
        Polygon p1 = new Polygon("triangle", 3, Color.RED);
        Polygon p2 = new Polygon("rectangle", 4, Color.BLUE);
        System.out.println(p1);
        System.out.println(p2);
    }
    
}
\end{javalst}

\quest{30 min}

\Q Review and discuss the attached source code for Polygon.
For each variable declaration, indicate where its contents will be stored: in the data (D), on the stack (S), or in the heap (H).

\begin{center}
Number of variable declarations in Polygon.java: \blank
\end{center}

\Q In the space below, draw a memory diagram to show the state of the \java{Polygon} program at the end of the main method.
Make sure you put each box in the right segment.

\Q Draw a UML diagram for the \java{Polygon} class.
Refer to examples from last week's activity and/or the current programming assignment.

\Q What information does the UML diagram convey?
In contrast, what information does the memory diagram convey?

\Q Based on your answer to \textbf{TODO}, what is the difference between a class and an object?

\Q Based on your answer to \textbf{TODO}, explain what the following compiler error means.

\begin{quote}
\begin{javalst}
public static void main(String[] args) {
    Polygon.sides = 4;
}
\end{javalst}

\it Error: non-static variable sides cannot be referenced from a static context
\end{quote}

\Q In your own words, explain what \java{static} means with respect to classes and objects.

