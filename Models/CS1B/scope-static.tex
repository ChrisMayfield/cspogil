% based on Model 1 of "Improved Scope" activity by Helen Hu

\model{Variable Scope}

As a team, review and discuss the \java{Circle} and \java{SwapDriver} classes found on the next page at the end of the questions.
Identify the \emph{scope} of each variable based on the table below.

\begin{center}
\small
\begin{tabular}{|L{95pt}|L{125pt}|L{115pt}|L{105pt}|}
\hline
\tr &
\tr \textbf{Where declared?} &
\tr \textbf{Where used?} &
\tr \textbf{Example} \\
\hline
\textbf{static variables} \par (``class variables'') &
declared outside of all methods (typically at the start of the class) &
anywhere in the class (or in other classes if also \java{public}) &
\java{circleCount} in the \java{Circle} class \\
\hline
\textbf{instance variables} \par (``attributes'') &
declared outside of all methods (typically after any static variables) &
any non-static method (or in static methods when another object has been created) &
\java{radius} in the \java{Circle} class \\
\hline
\textbf{parameters} &
declared inside the ()'s of a method signature &
anywhere within the method where they are declared &
\java{radius} in the \java{Circle} constructor \\
\hline
\textbf{local variables} &
declared inside a method (or inside another block of code, like a \java{for} loop) &
anywhere within the method or code block after they are declared &
\java{temp} in the \java{swapInts} method \\
\hline
\end{tabular}
\end{center}


\quest{20 min}


\Q Identify one static variable from the \java{Circle} class.
\begin{enumerate}
\item What is the name and purpose of the variable?
\item What is the scope of the variable?
\item What is one example of somewhere it cannot be used?
\end{enumerate}


\Q Identify one instance variable from the \java{Circle} class.
\begin{enumerate}
\item What is the name and purpose of the variable?
\item What is the scope of the variable?
\item What is one example of somewhere it cannot be used?
\end{enumerate}


\Q Identify an example of where an instance variable is used within a static method.
\begin{enumerate}
\item In which method does this occur?
\item Why is the method able to access these instance variables, even though they are private?
\item Explain how this method is not a violation of the rule that instance variables cannot be accessed inside a static method.
\end{enumerate}


\Q Identify one parameter from the \java{Circle} class.
\begin{enumerate}
\item What is the name and purpose of the variable?
\item What is the scope of the variable?
\item Where can the variable not be used?
\end{enumerate}


\Q Identify one local variable from the \java{Circle} class.
\begin{enumerate}
\item What is the name and purpose of the variable?
\item What is the scope of the variable?
\item Where can the variable not be used?
\end{enumerate}

\Q In the space below, predict the output of the \java{SwapDriver} program. Why are the results different when swapping integers and swapping Circle objects?

\begin{answer}[20em]
\begin{javaans}
BEFORE swap:
a = 7  b = 4
    Inside swapInts
    swapping integers 7 and 4
    finished swapping 4 and 7
AFTER swap:
a = 7  b = 4

BEFORE swap:
first = Circle(7)  second = Circle(4)
    Inside swapCircles
    swapping circles Circle(7) and Circle(4)
    finished swapping Circle(4) and Circle(7)
AFTER swap:
first = Circle(4)  second = Circle(7)

This program created 2 circles
\end{javaans}
\end{answer}

\newpage

\begin{javabox}
public class Circle {
    
    private static int circleCount = 0;
    
    private double radius;
    
    public Circle(double radius) {
        circleCount++;
        if (radius > 0) {
            this.radius = radius;
        } else {
            this.radius = 1;
        }
    }
    
    public static int getCircleCount() {
        return circleCount;
    }
    
    public double getRadius() {
        return this.radius;
    }
    
    public static void swapInts(int x, int y) {
        System.out.println("\tInside swapInts");
        System.out.println("\tswapping integers " + x + " and " + y);
        int temp = x;
        x = y;
        y = temp;
        System.out.println("\tfinished swapping " + x + " and " + y);
    }
    
    public static void swapCircles(Circle c1, Circle c2) {
        System.out.println("\tInside swapCircles");
        System.out.println("\tswapping circles " + c1 + " and " + c2);
        double r = c1.radius;
        c1.radius = c2.radius;
        c2.radius = r;
        System.out.println("\tfinished swapping " + c1 + " and " + c2);
    }
    
    public String toString() {
        return String.format("Circle(%.0f)", this.radius);
    }
}
\end{javabox}


\newpage

\begin{javabox}
public class SwapDriver {
    
    public static void main(String[] args) {
        
        // first try swapping integers
        int a = 7, b = 4;
        System.out.println("BEFORE swap:");
        System.out.println("a = " + a);
        System.out.println("b = " + b);
        Circle.swapInts(a, b);
        System.out.println("AFTER swap:");
        System.out.println("a = " + a);
        System.out.println("b = " + b);
        System.out.println();
        
        // next try swapping Circle radii
        Circle first, second;
        first = new Circle(7);
        second = new Circle(4);
        System.out.println("BEFORE swap:");
        System.out.println("first = " + first);
        System.out.println("second = " + second);
        Circle.swapCircles(first, second);
        System.out.println("AFTER swap:");
        System.out.println("first = " + first);
        System.out.println("second = " + second);
        System.out.println();
        
        System.out.printf("This program created %d circles",
                          Circle.getCircleCount());
        System.out.println();
    }
}
\end{javabox}
