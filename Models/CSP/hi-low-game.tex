% Based on Models 1 and 2 in "Searching" by Clif Kussmaul

\model{Hi-Lo Game}

\begin{wrapfigure}{r}{0.4\textwidth}
\centering
\vspace{-1em}
\includegraphics[width=\linewidth]{CSP/hi-low1.png}
\vspace*{-4em}
\end{wrapfigure}

Hi-Lo is a number guessing game with simple rules, played by school children.
\vspace{1ex}
\begin{enumerate}[nosep]
\item There are two players -- $A$ and $B$.
\item Player $A$ thinks of a number from 1 to 100.
\item Player $B$ guesses a number.
\item Player $A$ responds with \\ ``too high'', ``too low'', or ``you win''.
\item Players $B$ and $A$ continue to guess and \\ respond until $B$ wins (or gives up).
\end{enumerate}


\quest{12 min}


\Q How many different answers can player $A$ give?

\vspace{1ex}


\Q When does the game end?

\begin{answer}[2em]
\end{answer}


\Q Play the game a few times to ensure that everyone understands the rules.

\vspace{1ex}


\begin{center}
\begin{tabularx}{\linewidth}{|X|}
\hline
\it
In computing, we often must search for a particular item in a set.
Computer scientists are particularly interested in searching very large sets, with thousands or millions of values.
For example, the Harvard University Library has roughly 16,000,000 volumes, and the US Library of Congress has roughly 22 million cataloged books and over 100,000,000 total items.
\\
\hline
\end{tabularx}
\end{center}


\Q Identify 4--5 different guessing strategies that Player $B$ could use.
Each strategy should describe a \textbf{different approach} to the game.
For example: \textit{Start at 1, and count up until the correct answer is found.}
In computer science, we call such strategies \textbf{algorithms}.
Try to have a mixture of simple and clever algorithms, including ones that young children could use.

\begin{enumerate}
\item 
\item 
\item 
\item 
\item 
\end{enumerate}


\Q Rank order the algorithms with regard to how \textbf{fast} they will find the right answer.
Write 1 for the fastest algorithm (fewest guesses) and 5 for the slowest one (most guesses).

\begin{answer}
\end{answer}


\Q Rank order the algorithms with regard to how \textbf{easy} they are to describe or specify.
(Suppose you had to explain them to a first-grader so that he/she could play the game.)
Write 1 for the algorithm that is easiest to describe and 5 for the one that is hardest.

\begin{answer}
\end{answer}


\Q For each algorithm ($a$ to $e$), plot its fast and easy values on the graph:

\begin{table}[h]
\centering
\renewcommand{\arraystretch}{1.6}
\begin{tabularx}{275pt}{X C{28pt} | C{28pt} C{28pt} C{28pt} C{28pt} C{28pt}}
Hard & 5 \\
     & 4 \\
     & 3 \\
     & 2 \\
Easy & 1 \\
\hline
     & & 1    & 2 & 3 & 4 & 5    \\
     & & Fast &   &   &   & Slow \\
\end{tabularx}
\end{table}


\Q In complete sentences, describe the relationship between the fast and easy rankings, including what you see from the graph. 

\begin{answer}
\end{answer}
