% Based on Model 2 in "Activity 13 Encryption" by Helen Hu

\model{Random Substitution}

You have likely decoded ``secret messages'' that simply used a different letter for each letter of the alphabet.
These types of encryption schemes can be broken easily using frequency analysis.
For example, we know that the letter E typically appears most frequently in English, followed by the letter T.
Consider the following quotation, encrypted using a random substitution:

\begin{center}
\bf PXL QLHP PXABCH AB OAGL KML GMLL
\end{center}


\quest{10 min}


\Q Count the frequency of each letter in the above quotation.

\begin{enumerate}

\item Which letter appears the most often? \ans{L (6 times)}

\item Which letter(s) appears the second most often? \ans{A and P (3 times)}

\item Which letter(s) appears the third most often? \ans{B, G, H, M, X (2 times)}

\end{enumerate}


\Q Now consider commonly used English words.

\begin{enumerate}

\item What are some commonly used three-letter words? \ans{and, for, the}

\item What are some commonly used two-letter words? \ans{in, of, to}

\item Based on your answers to the above two questions, and using trial and error, decrypt the above quotation.

\begin{answer}[2em]
\begin{center}
THE BEST THINGS IN LIFE ARE FREE
\end{center}
\end{answer}

\item Discuss as a group the process you just used to decrypt the message, and describe it here.

\begin{answer}[5em]
We first guessed that L was E, since that's the most common letter.
Then we thought PXL was THE, and filled in the other T's, H's, and E's.
We then guessed that AB was IN.
At that point we guessed the entire sentence, and all the letters matched up.
\end{answer}

\end{enumerate}
