\model{Team Roles}

Decide who will be what role for today.
We will rotate roles throughout the course.
If you have three people, one may have two roles.
If you have five people, two may share the same role.

\begin{table}[h!]
\renewcommand{\arraystretch}{1.6}
\begin{tabular}{|p{0.4\linewidth}|p{0.5\linewidth}}
\cline{1-1}
Manager:   \ans{Helen Hu}       & \ans{keeps track of time, all voices are heard} \\
\cline{1-1}
Presenter: \ans{Clif Kussmaul}  & \ans{asks questions, gives the team's answers} \\
\cline{1-1}
Recorder:  \ans{Chris Mayfield} & \ans{quality control and consensus building} \\
\cline{1-1}
Reflector: \ans{Aman Yadav}     & \ans{team dynamics, suggest improvements} \\
\cline{1-1}
\end{tabular}
\end{table}


\quest{12 min}


\Q What is the difference between \textbf{bold} and \textit{italics} on the role cards?

\begin{answer}
The bold points describe what the responsibilities are.
Examples of what that person would say are in italics.
\end{answer}


\Q Manager: invite each person to explain their role to the team.
Recorder: make sure all team members take notes by writing down key phrases next to the table above.

\vspace{1ex}


\Q What responsibilities do two or more roles have in common?

\begin{answer}
Both the presenter and the recorder help the team reach consensus.
The manager and reflector both monitor how the team is working.
\end{answer}


\Q For each role, give an example of how someone observing your group would know that a person is \underline{not} doing their job well.

\begin{itemize}

\item Manager: \ans{The team is constantly getting behind.}

\item Presenter: \ans{The student doesn't know what to say.}

\item Recorder: \ans{Some team members aren't taking good notes.}

\item Reflector: \ans{The student never comments on team dynamics.}

\end{itemize}
