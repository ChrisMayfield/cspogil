% was the last question of CS1/Act06

\Q \label{nested}
A \emph{nested loop} is one that exists within the scope of another loop.
This construct is often used when there are two variables for which all combinations must be examined.

\begin{javalst}
    for (int x = 1; x < 4; x++) {
        for (int y = 1; y < 4; y++) {
            System.out.printf("The product of %d and %d is %d\n", x, y, x * y);
        }
        System.out.println();
    }
\end{javalst}

Write nested loops that compute and display the factorial of each integer from 1 to 20.
(Reuse your code from the previous question.)
Your output should be in this format:

\vspace{-1ex}
\begin{verbatim}
    The factorial of 1 is 1
    The factorial of 2 is 2
    The factorial of 3 is 6
    The factorial of 4 is 24
\end{verbatim}
\vspace{-1ex}

\begin{answer}[9em]
\begin{javaans}
for (int n = 1; n <= 20; n++) {
    long fact = 1;  // not int
    for (int i = n; i > 1; i--) {
        fact *= i;
    }
    System.out.printf("The factorial of %d is %d\n", n, fact);
}
\end{javaans}
\end{answer}


% removed these questions too, because they were too big of a jump

\Q \label{forchar}
Write a \java{for} loop that prints each character of a string on a separate line.
You will need to invoke the \java{length()} and \java{charAt()} methods.
Assume the string variable is named \java{word}.

\begin{answer}
\begin{javaans}
for (int i = 0; i < word.length(); i++) {
    System.out.println(word.charAt(i));
}
\end{javaans}
\end{answer}


\Q Rewrite your \java{for} loop in \ref{forchar} as a \java{while} loop.

\begin{answer}[6em]
\begin{javaans}
int i = 0;
while (i < word.length()) {
    System.out.println(word.charAt(i));
    i++;
}
\end{javaans}
\end{answer}


% corresponding text from the instructor guide:

You might need to guide students on \ref{forchar} when applying string methods in the context of a loop.
For example, you could write \java{word.length()} and \java{word.charAt(i)} on the board.
On \ref{nested}, students might not think to use a {\tt long} variable to store the factorial.
If time permits, show the output of the solution with {\tt fact} declared as an {\tt int}, and ask them to fix the code.
